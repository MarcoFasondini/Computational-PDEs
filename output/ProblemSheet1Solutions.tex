\documentclass[12pt,a4paper]{article}

\usepackage[a4paper,text={16.5cm,25.2cm},centering]{geometry}
\usepackage{lmodern}
\usepackage{amssymb,amsmath}
\usepackage{bm}
\usepackage{graphicx}
\usepackage{microtype}
\usepackage{hyperref}
\setlength{\parindent}{0pt}
\setlength{\parskip}{1.2ex}

\hypersetup
       {   pdfauthor = { Marco Fasondini },
           pdftitle={ foo },
           colorlinks=TRUE,
           linkcolor=black,
           citecolor=blue,
           urlcolor=blue
       }




\usepackage{upquote}
\usepackage{listings}
\usepackage{xcolor}
\lstset{
    basicstyle=\ttfamily\footnotesize,
    upquote=true,
    breaklines=true,
    breakindent=0pt,
    keepspaces=true,
    showspaces=false,
    columns=fullflexible,
    showtabs=false,
    showstringspaces=false,
    escapeinside={(*@}{@*)},
    extendedchars=true,
}
\newcommand{\HLJLt}[1]{#1}
\newcommand{\HLJLw}[1]{#1}
\newcommand{\HLJLe}[1]{#1}
\newcommand{\HLJLeB}[1]{#1}
\newcommand{\HLJLo}[1]{#1}
\newcommand{\HLJLk}[1]{\textcolor[RGB]{148,91,176}{\textbf{#1}}}
\newcommand{\HLJLkc}[1]{\textcolor[RGB]{59,151,46}{\textit{#1}}}
\newcommand{\HLJLkd}[1]{\textcolor[RGB]{214,102,97}{\textit{#1}}}
\newcommand{\HLJLkn}[1]{\textcolor[RGB]{148,91,176}{\textbf{#1}}}
\newcommand{\HLJLkp}[1]{\textcolor[RGB]{148,91,176}{\textbf{#1}}}
\newcommand{\HLJLkr}[1]{\textcolor[RGB]{148,91,176}{\textbf{#1}}}
\newcommand{\HLJLkt}[1]{\textcolor[RGB]{148,91,176}{\textbf{#1}}}
\newcommand{\HLJLn}[1]{#1}
\newcommand{\HLJLna}[1]{#1}
\newcommand{\HLJLnb}[1]{#1}
\newcommand{\HLJLnbp}[1]{#1}
\newcommand{\HLJLnc}[1]{#1}
\newcommand{\HLJLncB}[1]{#1}
\newcommand{\HLJLnd}[1]{\textcolor[RGB]{214,102,97}{#1}}
\newcommand{\HLJLne}[1]{#1}
\newcommand{\HLJLneB}[1]{#1}
\newcommand{\HLJLnf}[1]{\textcolor[RGB]{66,102,213}{#1}}
\newcommand{\HLJLnfm}[1]{\textcolor[RGB]{66,102,213}{#1}}
\newcommand{\HLJLnp}[1]{#1}
\newcommand{\HLJLnl}[1]{#1}
\newcommand{\HLJLnn}[1]{#1}
\newcommand{\HLJLno}[1]{#1}
\newcommand{\HLJLnt}[1]{#1}
\newcommand{\HLJLnv}[1]{#1}
\newcommand{\HLJLnvc}[1]{#1}
\newcommand{\HLJLnvg}[1]{#1}
\newcommand{\HLJLnvi}[1]{#1}
\newcommand{\HLJLnvm}[1]{#1}
\newcommand{\HLJLl}[1]{#1}
\newcommand{\HLJLld}[1]{\textcolor[RGB]{148,91,176}{\textit{#1}}}
\newcommand{\HLJLs}[1]{\textcolor[RGB]{201,61,57}{#1}}
\newcommand{\HLJLsa}[1]{\textcolor[RGB]{201,61,57}{#1}}
\newcommand{\HLJLsb}[1]{\textcolor[RGB]{201,61,57}{#1}}
\newcommand{\HLJLsc}[1]{\textcolor[RGB]{201,61,57}{#1}}
\newcommand{\HLJLsd}[1]{\textcolor[RGB]{201,61,57}{#1}}
\newcommand{\HLJLsdB}[1]{\textcolor[RGB]{201,61,57}{#1}}
\newcommand{\HLJLsdC}[1]{\textcolor[RGB]{201,61,57}{#1}}
\newcommand{\HLJLse}[1]{\textcolor[RGB]{59,151,46}{#1}}
\newcommand{\HLJLsh}[1]{\textcolor[RGB]{201,61,57}{#1}}
\newcommand{\HLJLsi}[1]{#1}
\newcommand{\HLJLso}[1]{\textcolor[RGB]{201,61,57}{#1}}
\newcommand{\HLJLsr}[1]{\textcolor[RGB]{201,61,57}{#1}}
\newcommand{\HLJLss}[1]{\textcolor[RGB]{201,61,57}{#1}}
\newcommand{\HLJLssB}[1]{\textcolor[RGB]{201,61,57}{#1}}
\newcommand{\HLJLnB}[1]{\textcolor[RGB]{59,151,46}{#1}}
\newcommand{\HLJLnbB}[1]{\textcolor[RGB]{59,151,46}{#1}}
\newcommand{\HLJLnfB}[1]{\textcolor[RGB]{59,151,46}{#1}}
\newcommand{\HLJLnh}[1]{\textcolor[RGB]{59,151,46}{#1}}
\newcommand{\HLJLni}[1]{\textcolor[RGB]{59,151,46}{#1}}
\newcommand{\HLJLnil}[1]{\textcolor[RGB]{59,151,46}{#1}}
\newcommand{\HLJLnoB}[1]{\textcolor[RGB]{59,151,46}{#1}}
\newcommand{\HLJLoB}[1]{\textcolor[RGB]{102,102,102}{\textbf{#1}}}
\newcommand{\HLJLow}[1]{\textcolor[RGB]{102,102,102}{\textbf{#1}}}
\newcommand{\HLJLp}[1]{#1}
\newcommand{\HLJLc}[1]{\textcolor[RGB]{153,153,119}{\textit{#1}}}
\newcommand{\HLJLch}[1]{\textcolor[RGB]{153,153,119}{\textit{#1}}}
\newcommand{\HLJLcm}[1]{\textcolor[RGB]{153,153,119}{\textit{#1}}}
\newcommand{\HLJLcp}[1]{\textcolor[RGB]{153,153,119}{\textit{#1}}}
\newcommand{\HLJLcpB}[1]{\textcolor[RGB]{153,153,119}{\textit{#1}}}
\newcommand{\HLJLcs}[1]{\textcolor[RGB]{153,153,119}{\textit{#1}}}
\newcommand{\HLJLcsB}[1]{\textcolor[RGB]{153,153,119}{\textit{#1}}}
\newcommand{\HLJLg}[1]{#1}
\newcommand{\HLJLgd}[1]{#1}
\newcommand{\HLJLge}[1]{#1}
\newcommand{\HLJLgeB}[1]{#1}
\newcommand{\HLJLgh}[1]{#1}
\newcommand{\HLJLgi}[1]{#1}
\newcommand{\HLJLgo}[1]{#1}
\newcommand{\HLJLgp}[1]{#1}
\newcommand{\HLJLgs}[1]{#1}
\newcommand{\HLJLgsB}[1]{#1}
\newcommand{\HLJLgt}[1]{#1}



\def\qqand{\qquad\hbox{and}\qquad}
\def\qqfor{\qquad\hbox{for}\qquad}
\def\qqas{\qquad\hbox{as}\qquad}
\def\half{ {1 \over 2} }
\def\D{ {\rm d} }
\def\I{ {\rm i} }
\def\E{ {\rm e} }
\def\C{ {\mathbb C} }
\def\R{ {\mathbb R} }
\def\bbR{ {\mathbb R} }
\def\H{ {\mathbb H} }
\def\Z{ {\mathbb Z} }
\def\CC{ {\cal C} }
\def\FF{ {\cal F} }
\def\HH{ {\cal H} }
\def\LL{ {\cal L} }
\def\vc#1{ {\mathbf #1} }
\def\bbC{ {\mathbb C} }



\def\fR{ f_{\rm R} }
\def\fL{ f_{\rm L} }

\def\qqqquad{\qquad\qquad}
\def\qqwhere{\qquad\hbox{where}\qquad}
\def\Res_#1{\underset{#1}{\rm Res}\,}
\def\sech{ {\rm sech}\, }
\def\acos{ {\rm acos}\, }
\def\asin{ {\rm asin}\, }
\def\atan{ {\rm atan}\, }
\def\Ei{ {\rm Ei}\, }
\def\upepsilon{\varepsilon}


\def\Xint#1{ \mathchoice
   {\XXint\displaystyle\textstyle{#1} }%
   {\XXint\textstyle\scriptstyle{#1} }%
   {\XXint\scriptstyle\scriptscriptstyle{#1} }%
   {\XXint\scriptscriptstyle\scriptscriptstyle{#1} }%
   \!\int}
\def\XXint#1#2#3{ {\setbox0=\hbox{$#1{#2#3}{\int}$}
     \vcenter{\hbox{$#2#3$}}\kern-.5\wd0} }
\def\ddashint{\Xint=}
\def\dashint{\Xint-}
% \def\dashint
\def\infdashint{\dashint_{-\infty}^\infty}




\def\addtab#1={#1\;&=}
\def\ccr{\\\addtab}
\def\ip<#1>{\left\langle{#1}\right\rangle}
\def\dx{\D x}
\def\dt{\D t}
\def\dz{\D z}
\def\ds{\D s}

\def\rR{ {\rm R} }
\def\rL{ {\rm L} }

\def\norm#1{\left\| #1 \right\|}

\def\pr(#1){\left({#1}\right)}
\def\br[#1]{\left[{#1}\right]}

\def\abs#1{\left|{#1}\right|}
\def\fpr(#1){\!\pr({#1})}

\def\sopmatrix#1{ \begin{pmatrix}#1\end{pmatrix} }

\def\endash{–}
\def\emdash{—}
\def\mdblksquare{\blacksquare}
\def\lgblksquare{\blacksquare}
\def\scre{\E}
\def\mapengine#1,#2.{\mapfunction{#1}\ifx\void#2\else\mapengine #2.\fi }

\def\map[#1]{\mapengine #1,\void.}

\def\mapenginesep_#1#2,#3.{\mapfunction{#2}\ifx\void#3\else#1\mapengine #3.\fi }

\def\mapsep_#1[#2]{\mapenginesep_{#1}#2,\void.}


\def\vcbr[#1]{\pr(#1)}


\def\bvect[#1,#2]{
{
\def\dots{\cdots}
\def\mapfunction##1{\ | \  ##1}
	\sopmatrix{
		 \,#1\map[#2]\,
	}
}
}



\def\vect[#1]{
{\def\dots{\ldots}
	\vcbr[{#1}]
} }

\def\vectt[#1]{
{\def\dots{\ldots}
	\vect[{#1}]^{\top}
} }

\def\Vectt[#1]{
{
\def\mapfunction##1{##1 \cr}
\def\dots{\vdots}
	\begin{pmatrix}
		\map[#1]
	\end{pmatrix}
} }

\def\addtab#1={#1\;&=}
\def\ccr{\\\addtab}

\def\questionequals{= \!\!\!\!\!\!{\scriptstyle ? \atop }\,\,\,}

\def\Ei{\rm Ei\,}

\begin{document}

\section{Problem Sheet 1 Solutions}
\begin{itemize}
\item[1. ] \textbf{[5 marks]} What are the Fourier coefficients $c_k$ of $\sin^4 x$?

\end{itemize}

\begin{eqnarray*}
(\sin x)^4 &=& \left({\exp({\rm i} x) - \exp(-{\rm i} x) \over 2 {\rm i}}\right)^4 \\
&=& \left({\exp(2{\rm i} x) -2 + \exp(-2{\rm i} x) \over -4}\right)^2 \\
&=& {\exp(4{\rm i} x) -4\exp(2{\rm i} x) +6 -4 \exp(-2{\rm i} x)+\exp(-4{\rm i} x) \over 16}
\end{eqnarray*}
hence, $c_{-4} = c_{4} = 1/16$, $c_{-2} = c_2 = -1/4$, $c_0 = 3/8$ and $c_k = 0$ otherwise.

\begin{itemize}
\item[2. ] \textbf{[5 marks]} Show for $0 \leq k,\ell \leq n-1$

\end{itemize}
\[
{1 \over n} \sum_{j=1}^n \cos k \theta_j \cos \ell \theta_j = \begin{cases} 1 & k = \ell = 0 \\
                                                  1/2 & k = \ell \\
                                                  0 & \hbox{otherwise}
                                                  \end{cases}
\]
for $\theta_j = \pi (j-1/2)/n$. Hint:     You may consider replacing $\cos$ with     complex exponentials:

\[
\cos \theta = {{\rm e}^{{\rm i}\theta} + {\rm e}^{-{\rm i}\theta} \over 2}.
\]
We have,


\begin{align*}
\frac{1}{n}\sum_{j=1}^n \cos(k\theta_j)\cos(l\theta_j) &= \frac{1}{4n}\sum_{j=1}^n\left( e^{i(k+l)\theta_j} + e^{-i(k+l)\theta_j} + e^{i(k-l)\theta_j} + e^{-i(k-l)\theta_j}\right) \\
&=	\frac{1}{4n}\sum_{j=1}^n \left( e^{ia_{kl}\theta_j} + e^{-ia_{kl}\theta_j} + e^{ib_{kl}\theta_j} + e^{-ib_{kl}\theta_j}\right),
\end{align*}
where we have defined $a_{kl} = k+l$ and $b_{kl} = k-l$. Now consider, for $a \in \mathbb{Z}$, a $\neq 2kn$ for some $k \in \mathbb{Z}$,


\begin{align*}
	\sum_{j=1}^n e^{ia\theta_j} &=\sum_{j=1}^n e^{ia\pi(j-\frac{1}{2})/n} \\
	&=e^{-ia\pi/2n}\sum_{j=1}^n e^{iaj\pi/n} \\
	&=e^{-ia\pi/2n}\sum_{j=1}^n (e^{ia\pi/n})^j \\
	&=e^{-ia\pi/2n} e^{ia\pi/n}\frac{(e^{ia\pi /n})^n - 1}{e^{ia\pi/n} - 1} \\
	&= e^{ia\pi/2n} \frac{e^{ia\pi} - 1 }{e^{ia\pi / n} - 1},
\end{align*}
where our assumptions on $a$ ensure that we are not dividing by $0$. Then we have, for $a$ as above,


\begin{align*}
	\sum_{j=1}^n \left(e^{ia\theta_j} + e^{-ia\theta_j}\right) &= e^{ia\pi/2n} \frac{e^{ia\pi} - 1 }{e^{ia\pi / n} - 1} + e^{-ia\pi/2n} \frac{e^{-ia\pi} - 1 }{e^{-ia\pi / n} - 1} \\
	&= e^{ia\pi/2n} \frac{e^{ia\pi} - 1 }{e^{ia\pi / n} - 1} + e^{-ia\pi/2n} \cdot\frac{e^{ia\pi/n}}{e^{ia\pi/n}}\cdot \frac{e^{-ia\pi} - 1 }{e^{-ia\pi / n} - 1} \\
	&= e^{ia\pi/2n} \frac{e^{ia\pi} - 1 }{e^{ia\pi / n} - 1} + e^{ia\pi/2n}\frac{e^{-ia\pi} - 1}{1 - e^{ia\pi/n}} \\
	&=e^{ia\pi/2n} \frac{e^{ia\pi} - 1 }{e^{ia\pi / n} - 1} - e^{ia\pi/2n}\frac{e^{-ia\pi} - 1}{e^{ia\pi/n} - 1} \\
	&= \frac{e^{ia\pi/2n}}{e^{ia\pi/n-1}}\left(e^{ia\pi} - 1 - e^{-ia\pi} + 1 \right) \\
	&= \frac{e^{ia\pi/2n}}{e^{ia\pi/n-1}}\frac{1}{2i}\sin(a\pi),
\end{align*}
which is $0$ for $a$ an integer.

Now, when $k = l = 0$, we have $a_{kl} = b_{kl} = 0$, and,

\[
\frac{1}{n}\sum_{j=1}^n \cos(k\theta_j)\cos(l\theta_j) = \frac{1}{4n}\sum_{j=1}^n (1 + 1 + 1 + 1) = 1.
\]
When $k = l \neq 0$, we have $0 < a_{kl} = 2k < 2n$, and $b_{kl} = 0$. Hence,


\begin{align*}
	\frac{1}{n}\sum_{j=1}^n \cos(k\theta_j)\cos(l\theta_j) &= \frac{1}{4n}\sum_{j=1}^n (e^{ia_{kl}\theta_j} + e^{-ia_{kl}\theta_j} + 1 + 1) \\
	&= \frac{1}{4n}\left[ \sum_{j=1}^n\left( e^{ia_{kl}\theta_j} + e^{-ia_{kl}\theta_j}\right) +2n\right] \\
	&= \frac{1}{2},
\end{align*}
since $a_{kl}$ meets the conditions for the sum considered above.

When $k \neq l$, we have, $-2n < a_{kl}, b_{kl} <  2n$ and $a_{kl}, b_{kl} \neq 0$, hence,


\begin{align*}
	\frac{1}{n}\sum_{j=1}^n \cos(k\theta_j)\cos(l\theta_j) &= \frac{1}{4n}\sum_{j=1}^n (e^{ia_{kl}\theta_j} + e^{-ia_{kl}\theta_j} + e^{ib_{kl}\theta_j} + e^{-ib_{kl}\theta_j}) \\
	&= \frac{1}{4n} \left[\sum_{j=1}^n (e^{ia_{kl}\theta_j} + e^{-ia_{kl}\theta_j})  + \sum_{j=1}^n (e^{ib_{kl}\theta_j} + e^{-ib_{kl}\theta_j} )\right] \\
	&= 0.
\end{align*}
\begin{itemize}
\item[3. ] \textbf{[5 marks]} Consider the Discrete Cosine Transform (DCT)

\end{itemize}
\[
C_n := \begin{bmatrix}
\sqrt{1/n} \\
 & \sqrt{2/n} \\ 
 && \ddots \\
 &&& \sqrt{2/n}
 \end{bmatrix}
\begin{bmatrix}
    1 & \cdots & 1\\
    \cos \theta_1 & \cdots & \cos \theta_n \\
    \vdots & \ddots & \vdots \\
    \cos (n-1)\theta_1 & \cdots & \cos (n-1)\theta_n
\end{bmatrix}
\]
for $\theta_j = \pi(j-1/2)/n$. Prove that $C_n$ is orthogonal: $C_n^{\top} C_n = C_n C_n^{\top} = I$. Hint: $C_n C_n^{\top} = I$ might be easier to show than $C_n^{\top} C_n = I$ using the previous problem.

The components of $C$ without the diagonal matrix, which we may call $\hat{C}$ are

\[
\hat{C}_{ij} = \cos((j-1)\ensuremath{\theta}_{i-1}),
\]
where $\ensuremath{\theta}_j = \ensuremath{\pi}(j-1/2)/n$. Recalling that the elements of matrix multiplication are given by

\[
(ab)_{ij} := \sum_{k=1}^n a_{ik} b_{kj}
\]
we find that

\[
(\hat{C}_n \hat{C}_n^\ensuremath{\top})_{ij} = \sum_{k=1}^n \cos((i-1)\ensuremath{\theta}_{k-1}) \cos((k-1)\ensuremath{\theta}_{j-1}).
\]
By using the previous problem and the terms on the diagonal matrices which ensure that the $1/2$ terms become $1$ we know how to compute all of these entries and find that it is the identity.

Here is a computer-based demonstration:


\begin{lstlisting}
(*@\HLJLk{using}@*) (*@\HLJLn{LinearAlgebra}@*)(*@\HLJLp{,}@*) (*@\HLJLn{Plots}@*)(*@\HLJLp{,}@*) (*@\HLJLn{FFTW}@*)
(*@\HLJLn{n}@*) (*@\HLJLoB{=}@*) (*@\HLJLni{5}@*)
(*@\HLJLn{\ensuremath{\theta}}@*) (*@\HLJLoB{=}@*) (*@\HLJLnf{range}@*)(*@\HLJLp{(}@*)(*@\HLJLn{\ensuremath{\pi}}@*)(*@\HLJLoB{/}@*)(*@\HLJLp{(}@*)(*@\HLJLni{2}@*)(*@\HLJLn{n}@*)(*@\HLJLp{);}@*) (*@\HLJLn{step}@*)(*@\HLJLoB{=}@*)(*@\HLJLn{\ensuremath{\pi}}@*)(*@\HLJLoB{/}@*)(*@\HLJLn{n}@*)(*@\HLJLp{,}@*) (*@\HLJLn{length}@*)(*@\HLJLoB{=}@*)(*@\HLJLn{n}@*)(*@\HLJLp{)}@*) (*@\HLJLcs{{\#}}@*) (*@\HLJLcs{n}@*) (*@\HLJLcs{evenly}@*) (*@\HLJLcs{spaced}@*) (*@\HLJLcs{points}@*) (*@\HLJLcs{starting}@*) (*@\HLJLcs{at}@*) (*@\HLJLcs{\ensuremath{\pi}/(2n)}@*) (*@\HLJLcs{with}@*) (*@\HLJLcs{step}@*) (*@\HLJLcs{size}@*) (*@\HLJLcs{\ensuremath{\pi}/n}@*)
(*@\HLJLn{C}@*) (*@\HLJLoB{=}@*) (*@\HLJLnf{Diagonal}@*)(*@\HLJLp{([}@*)(*@\HLJLni{1}@*)(*@\HLJLoB{/}@*)(*@\HLJLnf{sqrt}@*)(*@\HLJLp{(}@*)(*@\HLJLn{n}@*)(*@\HLJLp{);}@*) (*@\HLJLnf{fill}@*)(*@\HLJLp{(}@*)(*@\HLJLnf{sqrt}@*)(*@\HLJLp{(}@*)(*@\HLJLni{2}@*)(*@\HLJLoB{/}@*)(*@\HLJLn{n}@*)(*@\HLJLp{),}@*) (*@\HLJLn{n}@*)(*@\HLJLoB{-}@*)(*@\HLJLni{1}@*)(*@\HLJLp{)])}@*) (*@\HLJLoB{*}@*) (*@\HLJLp{[}@*)(*@\HLJLnf{cos}@*)(*@\HLJLp{((}@*)(*@\HLJLn{k}@*)(*@\HLJLoB{-}@*)(*@\HLJLni{1}@*)(*@\HLJLp{)}@*)(*@\HLJLoB{*}@*)(*@\HLJLn{\ensuremath{\theta}}@*)(*@\HLJLp{[}@*)(*@\HLJLn{j}@*)(*@\HLJLp{])}@*) (*@\HLJLk{for}@*) (*@\HLJLn{k}@*)(*@\HLJLoB{=}@*)(*@\HLJLni{1}@*)(*@\HLJLoB{:}@*)(*@\HLJLn{n}@*)(*@\HLJLp{,}@*) (*@\HLJLn{j}@*)(*@\HLJLoB{=}@*)(*@\HLJLni{1}@*)(*@\HLJLoB{:}@*)(*@\HLJLn{n}@*)(*@\HLJLp{]}@*)
(*@\HLJLn{C}@*)(*@\HLJLoB{{\textquotesingle}}@*)(*@\HLJLn{C}@*)
\end{lstlisting}

\begin{lstlisting}
5(*@\ensuremath{\times}@*)5 Matrix(*@{{\{}}@*)Float64(*@{{\}}}@*):
  1.0          -4.85266e-18  -2.82901e-17   1.68455e-17   3.95658e-17
 -4.85266e-18   1.0           4.68569e-17  -6.18283e-17   4.0019e-17
 -2.82901e-17   4.68569e-17   1.0           6.05122e-18  -1.76076e-16
  1.68455e-17  -6.18283e-17   6.05122e-18   1.0           1.03351e-16
  3.95658e-17   4.0019e-17   -1.76076e-16   1.03351e-16   1.0
\end{lstlisting}


\begin{itemize}
\item[4. ] \textbf{[10 marks]}  Consider the variable-coefficient advection equation

\end{itemize}
\[
   u_t + c(x)u_x = 0, \qquad c(x) = \frac{1}{5} + \sin^2(x-1), \qquad x \in [0, 2\pi), \qquad t \in [0, T],
\]
with $u(x,0) = f(x) = {\rm e}^{-100(x-1)^2}$, which we approximate with the leapfrog method

\[
\mathbf{u}^{i+1} = \mathbf{u}^{i-1} - 2\tau \mathcal{F}^{-1}\left\lbrace {\rm i}(-m\!:\!m)\cdot\mathcal{F}\lbrace \mathbf{u}^{i} \rbrace\right\rbrace, \qquad i = 0, \ldots, n_t-1.
\]
Note that one needs $\mathbf{u}^{-1}$ to initialise the leapfrog method.  Let the entries of $\mathbf{u}^{-1}$ be $u^{-1}_j = f(x_j + \tau/5)$, $j = 0, \ldots, n_x-1$.  The exact solution is periodic in time, i.e.,

\[
   u(x,t+T) = u(x,t)
\]
where

\[
   T = \int_{0}^{2\pi}\frac{1}{c(x)}{\rm d}x = 12.8254983016186\ldots
\]
Compute $T$ using the Trapezoidal rule and confirm that you get the value stated above.  Then compute

\[
   e(n_x) = \max_{j = 0, \ldots, n_x-1} \vert u^{0}_j - u^{n_t}_{j} \vert
\]
where $n_t = 8n_x$ and $\tau n_t = T$  and plot $e(n_x)$ for $n_x = 2^k + 1$, with $k = 5, 6, \ldots, 10$.  Comment on the behaviour of $e(n_x)$.

As discussed in the notes, the trapezoidal rule approximation to an integral is

\[
\int_{a}^{b} g(x)\,{\rm d}x \approx \frac{h}{2}\left(g(x_0) + 2g(x_1) + 2g(x_2) + \cdots + 2g(x_{n-1}) + g(x_{n})    \right).
\]
To compute $T$, we set $a = 0$, $b = 2\pi$ and $g(x) = 1/c(x)$ and note that since $c(x)$ is $2\pi$-periodic, $g(x_0) = g(x_n)$, hence

\[
\int_0^{2\pi}\frac{1}{c(x)}\,{\rm d}x \approx \frac{2\pi}{n}\sum_{j=0}^{n-1}\frac{1}{c(x_j)}
\]

\begin{lstlisting}
(*@\HLJLn{c}@*) (*@\HLJLoB{=}@*) (*@\HLJLn{x}@*) (*@\HLJLoB{->}@*) (*@\HLJLnfB{0.2}@*) (*@\HLJLoB{+}@*) (*@\HLJLnf{sin}@*)(*@\HLJLp{(}@*)(*@\HLJLn{x}@*) (*@\HLJLoB{-}@*) (*@\HLJLni{1}@*)(*@\HLJLp{)}@*)(*@\HLJLoB{{\textasciicircum}}@*)(*@\HLJLni{2}@*)
(*@\HLJLn{n}@*) (*@\HLJLoB{=}@*) (*@\HLJLni{80}@*)
(*@\HLJLn{xx}@*) (*@\HLJLoB{=}@*) (*@\HLJLnf{range}@*)(*@\HLJLp{(}@*)(*@\HLJLni{0}@*)(*@\HLJLp{,}@*)(*@\HLJLni{2}@*)(*@\HLJLn{\ensuremath{\pi}}@*)(*@\HLJLp{;}@*)(*@\HLJLn{length}@*)(*@\HLJLoB{=}@*)(*@\HLJLn{n}@*)(*@\HLJLoB{+}@*)(*@\HLJLni{1}@*)(*@\HLJLp{)[}@*)(*@\HLJLni{1}@*)(*@\HLJLoB{:}@*)(*@\HLJLk{end}@*)(*@\HLJLoB{-}@*)(*@\HLJLni{1}@*)(*@\HLJLp{]}@*)
(*@\HLJLn{T}@*) (*@\HLJLoB{=}@*) (*@\HLJLni{2}@*)(*@\HLJLn{\ensuremath{\pi}}@*)(*@\HLJLoB{/}@*)(*@\HLJLn{n}@*)(*@\HLJLoB{*}@*)(*@\HLJLnf{sum}@*)(*@\HLJLp{(}@*)(*@\HLJLni{1}@*) (*@\HLJLoB{./}@*)(*@\HLJLn{c}@*)(*@\HLJLoB{.}@*)(*@\HLJLp{(}@*)(*@\HLJLn{xx}@*)(*@\HLJLp{))}@*)
\end{lstlisting}

\begin{lstlisting}
12.825498301618637
\end{lstlisting}


\begin{lstlisting}
(*@\HLJLn{c}@*) (*@\HLJLoB{=}@*) (*@\HLJLn{x}@*) (*@\HLJLoB{->}@*) (*@\HLJLnfB{0.2}@*) (*@\HLJLoB{+}@*) (*@\HLJLnf{sin}@*)(*@\HLJLp{(}@*)(*@\HLJLn{x}@*) (*@\HLJLoB{-}@*) (*@\HLJLni{1}@*)(*@\HLJLp{)}@*)(*@\HLJLoB{{\textasciicircum}}@*)(*@\HLJLni{2}@*)
(*@\HLJLn{f}@*) (*@\HLJLoB{=}@*) (*@\HLJLn{x}@*) (*@\HLJLoB{->}@*) (*@\HLJLnf{exp}@*)(*@\HLJLp{(}@*)(*@\HLJLoB{-}@*)(*@\HLJLni{100}@*)(*@\HLJLoB{*}@*)(*@\HLJLp{(}@*)(*@\HLJLn{x}@*)(*@\HLJLoB{-}@*)(*@\HLJLni{1}@*)(*@\HLJLp{)}@*)(*@\HLJLoB{{\textasciicircum}}@*)(*@\HLJLni{2}@*)(*@\HLJLp{)}@*)
(*@\HLJLn{nxv}@*) (*@\HLJLoB{=}@*) (*@\HLJLni{2}@*) (*@\HLJLoB{.{\textasciicircum}}@*)(*@\HLJLp{(}@*)(*@\HLJLni{5}@*)(*@\HLJLoB{:}@*)(*@\HLJLni{10}@*)(*@\HLJLp{)}@*) (*@\HLJLoB{.+}@*) (*@\HLJLni{1}@*)
(*@\HLJLn{maxerr}@*) (*@\HLJLoB{=}@*) (*@\HLJLp{[]}@*)
(*@\HLJLk{for}@*) (*@\HLJLn{n\ensuremath{\_x}}@*) (*@\HLJLoB{=}@*) (*@\HLJLn{nxv}@*) 
    (*@\HLJLn{m}@*) (*@\HLJLoB{=}@*) (*@\HLJLp{(}@*)(*@\HLJLn{n\ensuremath{\_x}}@*)(*@\HLJLoB{-}@*)(*@\HLJLni{1}@*)(*@\HLJLp{)}@*)(*@\HLJLoB{\ensuremath{\div}}@*)(*@\HLJLni{2}@*)
    (*@\HLJLn{h}@*) (*@\HLJLoB{=}@*) (*@\HLJLni{2}@*)(*@\HLJLn{\ensuremath{\pi}}@*)(*@\HLJLoB{/}@*)(*@\HLJLn{n\ensuremath{\_x}}@*)
    (*@\HLJLn{x}@*) (*@\HLJLoB{=}@*) (*@\HLJLn{h}@*)(*@\HLJLoB{*}@*)(*@\HLJLp{(}@*)(*@\HLJLni{0}@*)(*@\HLJLoB{:}@*)(*@\HLJLn{n\ensuremath{\_x}}@*)(*@\HLJLoB{-}@*)(*@\HLJLni{1}@*)(*@\HLJLp{)}@*)
    (*@\HLJLn{tv}@*) (*@\HLJLoB{=}@*) (*@\HLJLnf{range}@*)(*@\HLJLp{(}@*)(*@\HLJLni{0}@*)(*@\HLJLp{,}@*)(*@\HLJLn{T}@*)(*@\HLJLp{;}@*)(*@\HLJLn{length}@*)(*@\HLJLoB{=}@*)(*@\HLJLni{8}@*)(*@\HLJLn{n\ensuremath{\_x}}@*)(*@\HLJLoB{+}@*)(*@\HLJLni{1}@*)(*@\HLJLp{)}@*)
    (*@\HLJLn{\ensuremath{\tau}}@*) (*@\HLJLoB{=}@*) (*@\HLJLn{tv}@*)(*@\HLJLp{[}@*)(*@\HLJLni{2}@*)(*@\HLJLp{]}@*)(*@\HLJLoB{-}@*)(*@\HLJLn{tv}@*)(*@\HLJLp{[}@*)(*@\HLJLni{1}@*)(*@\HLJLp{]}@*)
    (*@\HLJLn{u}@*) (*@\HLJLoB{=}@*) (*@\HLJLn{f}@*)(*@\HLJLoB{.}@*)(*@\HLJLp{(}@*)(*@\HLJLn{x}@*)(*@\HLJLp{)}@*)
    (*@\HLJLn{uold}@*) (*@\HLJLoB{=}@*) (*@\HLJLn{f}@*)(*@\HLJLoB{.}@*)(*@\HLJLp{(}@*)(*@\HLJLn{x}@*) (*@\HLJLoB{.+}@*) (*@\HLJLnfB{0.2}@*)(*@\HLJLn{\ensuremath{\tau}}@*)(*@\HLJLp{)}@*)
    (*@\HLJLnd{@time}@*) (*@\HLJLk{begin}@*)
    (*@\HLJLk{for}@*) (*@\HLJLn{n}@*) (*@\HLJLoB{=}@*) (*@\HLJLni{1}@*)(*@\HLJLoB{:}@*)(*@\HLJLni{8}@*)(*@\HLJLn{n\ensuremath{\_x}}@*)
        (*@\HLJLn{unew}@*) (*@\HLJLoB{=}@*) (*@\HLJLn{real}@*)(*@\HLJLoB{.}@*)(*@\HLJLp{(}@*)(*@\HLJLn{uold}@*) (*@\HLJLoB{-}@*) (*@\HLJLni{2}@*)(*@\HLJLn{\ensuremath{\tau}}@*)(*@\HLJLoB{*}@*)(*@\HLJLn{c}@*)(*@\HLJLoB{.}@*)(*@\HLJLp{(}@*)(*@\HLJLn{x}@*)(*@\HLJLp{)}@*)(*@\HLJLoB{.*}@*)(*@\HLJLnf{ifft}@*)(*@\HLJLp{(}@*)(*@\HLJLnf{ifftshift}@*)(*@\HLJLp{(}@*)(*@\HLJLn{im}@*)(*@\HLJLoB{*}@*)(*@\HLJLp{(}@*)(*@\HLJLoB{-}@*)(*@\HLJLn{m}@*)(*@\HLJLoB{:}@*)(*@\HLJLn{m}@*)(*@\HLJLp{))}@*)(*@\HLJLoB{.*}@*)(*@\HLJLnf{fft}@*)(*@\HLJLp{(}@*)(*@\HLJLn{u}@*)(*@\HLJLp{)))}@*)
        (*@\HLJLn{uold}@*) (*@\HLJLoB{=}@*) (*@\HLJLn{u}@*)
        (*@\HLJLn{u}@*) (*@\HLJLoB{=}@*) (*@\HLJLn{unew}@*)
    (*@\HLJLk{end}@*)
    (*@\HLJLk{end}@*)
    (*@\HLJLkd{global}@*) (*@\HLJLn{maxerr}@*)
    (*@\HLJLn{maxerr}@*) (*@\HLJLoB{=}@*) (*@\HLJLp{[}@*)(*@\HLJLn{maxerr}@*)(*@\HLJLp{;}@*)(*@\HLJLnf{norm}@*)(*@\HLJLp{(}@*)(*@\HLJLn{u}@*)(*@\HLJLoB{-}@*)(*@\HLJLn{f}@*)(*@\HLJLoB{.}@*)(*@\HLJLp{(}@*)(*@\HLJLn{x}@*)(*@\HLJLp{),}@*)(*@\HLJLn{Inf}@*)(*@\HLJLp{)]}@*)
(*@\HLJLk{end}@*)
\end{lstlisting}

\begin{lstlisting}
0.096903 seconds (159.31 k allocations: 9.735 MiB, 61.45(*@{{\%}}@*) compilation tim
e)
  0.126623 seconds (38.50 k allocations: 6.919 MiB)
  0.151179 seconds (77.41 k allocations: 22.644 MiB)
  0.765419 seconds (155.24 k allocations: 80.015 MiB, 6.99(*@{{\%}}@*) gc time)
  1.672286 seconds (310.88 k allocations: 295.999 MiB, 3.96(*@{{\%}}@*) gc time)
  5.037372 seconds (679.58 k allocations: 1.105 GiB, 3.22(*@{{\%}}@*) gc time)
\end{lstlisting}


\begin{lstlisting}
(*@\HLJLnf{scatter}@*)(*@\HLJLp{(}@*)(*@\HLJLn{nxv}@*)(*@\HLJLp{,}@*)(*@\HLJLn{maxerr}@*)(*@\HLJLp{;}@*)(*@\HLJLn{yscale}@*)(*@\HLJLoB{=:}@*)(*@\HLJLn{log10}@*)(*@\HLJLp{,}@*)(*@\HLJLn{xscale}@*)(*@\HLJLoB{=:}@*)(*@\HLJLn{log10}@*)(*@\HLJLp{,}@*)(*@\HLJLn{legend}@*)(*@\HLJLoB{=}@*)(*@\HLJLkc{false}@*)(*@\HLJLp{,}@*)
(*@\HLJLn{xlabel}@*)(*@\HLJLoB{=}@*)(*@\HLJLs{"{}n\ensuremath{\_x}"{}}@*)(*@\HLJLp{,}@*)(*@\HLJLn{ylabel}@*)(*@\HLJLoB{=}@*)(*@\HLJLs{"{}e(n\ensuremath{\_x})"{}}@*)(*@\HLJLp{)}@*)
(*@\HLJLnf{plot!}@*)(*@\HLJLp{(}@*)(*@\HLJLn{nxv}@*)(*@\HLJLp{,}@*)(*@\HLJLni{170}@*) (*@\HLJLoB{./}@*)(*@\HLJLn{nxv}@*)(*@\HLJLoB{.{\textasciicircum}}@*)(*@\HLJLni{2}@*)(*@\HLJLp{)}@*)
\end{lstlisting}

\includegraphics[width=\linewidth]{/figures/ProblemSheet1Solutions_4_1.pdf}

The plot indicates that $e(n_x) = \mathcal{O}(n_x^{-2})$, $n_x \to \infty$.



\end{document}
