\documentclass[12pt,a4paper]{article}

\usepackage[a4paper,text={16.5cm,25.2cm},centering]{geometry}
\usepackage{lmodern}
\usepackage{amssymb,amsmath}
\usepackage{bm}
\usepackage{graphicx}
\usepackage{microtype}
\usepackage{hyperref}
\setlength{\parindent}{0pt}
\setlength{\parskip}{1.2ex}

\hypersetup
       {   pdfauthor = {  },
           pdftitle={  },
           colorlinks=TRUE,
           linkcolor=black,
           citecolor=blue,
           urlcolor=blue
       }




\usepackage{upquote}
\usepackage{listings}
\usepackage{xcolor}
\lstset{
    basicstyle=\ttfamily\footnotesize,
    upquote=true,
    breaklines=true,
    breakindent=0pt,
    keepspaces=true,
    showspaces=false,
    columns=fullflexible,
    showtabs=false,
    showstringspaces=false,
    escapeinside={(*@}{@*)},
    extendedchars=true,
}
\newcommand{\HLJLt}[1]{#1}
\newcommand{\HLJLw}[1]{#1}
\newcommand{\HLJLe}[1]{#1}
\newcommand{\HLJLeB}[1]{#1}
\newcommand{\HLJLo}[1]{#1}
\newcommand{\HLJLk}[1]{\textcolor[RGB]{148,91,176}{\textbf{#1}}}
\newcommand{\HLJLkc}[1]{\textcolor[RGB]{59,151,46}{\textit{#1}}}
\newcommand{\HLJLkd}[1]{\textcolor[RGB]{214,102,97}{\textit{#1}}}
\newcommand{\HLJLkn}[1]{\textcolor[RGB]{148,91,176}{\textbf{#1}}}
\newcommand{\HLJLkp}[1]{\textcolor[RGB]{148,91,176}{\textbf{#1}}}
\newcommand{\HLJLkr}[1]{\textcolor[RGB]{148,91,176}{\textbf{#1}}}
\newcommand{\HLJLkt}[1]{\textcolor[RGB]{148,91,176}{\textbf{#1}}}
\newcommand{\HLJLn}[1]{#1}
\newcommand{\HLJLna}[1]{#1}
\newcommand{\HLJLnb}[1]{#1}
\newcommand{\HLJLnbp}[1]{#1}
\newcommand{\HLJLnc}[1]{#1}
\newcommand{\HLJLncB}[1]{#1}
\newcommand{\HLJLnd}[1]{\textcolor[RGB]{214,102,97}{#1}}
\newcommand{\HLJLne}[1]{#1}
\newcommand{\HLJLneB}[1]{#1}
\newcommand{\HLJLnf}[1]{\textcolor[RGB]{66,102,213}{#1}}
\newcommand{\HLJLnfm}[1]{\textcolor[RGB]{66,102,213}{#1}}
\newcommand{\HLJLnp}[1]{#1}
\newcommand{\HLJLnl}[1]{#1}
\newcommand{\HLJLnn}[1]{#1}
\newcommand{\HLJLno}[1]{#1}
\newcommand{\HLJLnt}[1]{#1}
\newcommand{\HLJLnv}[1]{#1}
\newcommand{\HLJLnvc}[1]{#1}
\newcommand{\HLJLnvg}[1]{#1}
\newcommand{\HLJLnvi}[1]{#1}
\newcommand{\HLJLnvm}[1]{#1}
\newcommand{\HLJLl}[1]{#1}
\newcommand{\HLJLld}[1]{\textcolor[RGB]{148,91,176}{\textit{#1}}}
\newcommand{\HLJLs}[1]{\textcolor[RGB]{201,61,57}{#1}}
\newcommand{\HLJLsa}[1]{\textcolor[RGB]{201,61,57}{#1}}
\newcommand{\HLJLsb}[1]{\textcolor[RGB]{201,61,57}{#1}}
\newcommand{\HLJLsc}[1]{\textcolor[RGB]{201,61,57}{#1}}
\newcommand{\HLJLsd}[1]{\textcolor[RGB]{201,61,57}{#1}}
\newcommand{\HLJLsdB}[1]{\textcolor[RGB]{201,61,57}{#1}}
\newcommand{\HLJLsdC}[1]{\textcolor[RGB]{201,61,57}{#1}}
\newcommand{\HLJLse}[1]{\textcolor[RGB]{59,151,46}{#1}}
\newcommand{\HLJLsh}[1]{\textcolor[RGB]{201,61,57}{#1}}
\newcommand{\HLJLsi}[1]{#1}
\newcommand{\HLJLso}[1]{\textcolor[RGB]{201,61,57}{#1}}
\newcommand{\HLJLsr}[1]{\textcolor[RGB]{201,61,57}{#1}}
\newcommand{\HLJLss}[1]{\textcolor[RGB]{201,61,57}{#1}}
\newcommand{\HLJLssB}[1]{\textcolor[RGB]{201,61,57}{#1}}
\newcommand{\HLJLnB}[1]{\textcolor[RGB]{59,151,46}{#1}}
\newcommand{\HLJLnbB}[1]{\textcolor[RGB]{59,151,46}{#1}}
\newcommand{\HLJLnfB}[1]{\textcolor[RGB]{59,151,46}{#1}}
\newcommand{\HLJLnh}[1]{\textcolor[RGB]{59,151,46}{#1}}
\newcommand{\HLJLni}[1]{\textcolor[RGB]{59,151,46}{#1}}
\newcommand{\HLJLnil}[1]{\textcolor[RGB]{59,151,46}{#1}}
\newcommand{\HLJLnoB}[1]{\textcolor[RGB]{59,151,46}{#1}}
\newcommand{\HLJLoB}[1]{\textcolor[RGB]{102,102,102}{\textbf{#1}}}
\newcommand{\HLJLow}[1]{\textcolor[RGB]{102,102,102}{\textbf{#1}}}
\newcommand{\HLJLp}[1]{#1}
\newcommand{\HLJLc}[1]{\textcolor[RGB]{153,153,119}{\textit{#1}}}
\newcommand{\HLJLch}[1]{\textcolor[RGB]{153,153,119}{\textit{#1}}}
\newcommand{\HLJLcm}[1]{\textcolor[RGB]{153,153,119}{\textit{#1}}}
\newcommand{\HLJLcp}[1]{\textcolor[RGB]{153,153,119}{\textit{#1}}}
\newcommand{\HLJLcpB}[1]{\textcolor[RGB]{153,153,119}{\textit{#1}}}
\newcommand{\HLJLcs}[1]{\textcolor[RGB]{153,153,119}{\textit{#1}}}
\newcommand{\HLJLcsB}[1]{\textcolor[RGB]{153,153,119}{\textit{#1}}}
\newcommand{\HLJLg}[1]{#1}
\newcommand{\HLJLgd}[1]{#1}
\newcommand{\HLJLge}[1]{#1}
\newcommand{\HLJLgeB}[1]{#1}
\newcommand{\HLJLgh}[1]{#1}
\newcommand{\HLJLgi}[1]{#1}
\newcommand{\HLJLgo}[1]{#1}
\newcommand{\HLJLgp}[1]{#1}
\newcommand{\HLJLgs}[1]{#1}
\newcommand{\HLJLgsB}[1]{#1}
\newcommand{\HLJLgt}[1]{#1}


\begin{document}



\section{Fourier methods for periodic functions}
We assume that $f$ is $2\pi$-periodic and its Fourier series converges on $x \in [0, 2\pi)$, i.e.,

\[
f(x) = \sum_{k=-\infty}^{\infty} c_k \mathrm{e}^{\mathrm{i} k x}
\]
where the Fourier coeffients are defined as

\[
%c_k = \frac{1}{2\pi}\int_{-\pi}^{\pi} f(x) \mathrm{e}^{-\mathrm{i}k x} \mathrm{d} x.
c_k = \frac{1}{2\pi}\int_{0}^{2\pi} f(x) \mathrm{e}^{-\mathrm{i}k x} \mathrm{d} x.
\]
Suppose we only know the values of $f(x)$ at the $n$ equally spaced points $x_{j} = jh = \frac{2\pi j}{n}$ for $j = 0, \ldots, n-1$, can we somehow use Fourier series to approximate $f$ and its derivatives?

The idea is to use a truncated Fourier series and approximate Fourier coefficients $\widetilde{c}_k$ to approximate $f$ with

\[
f(x) \approx \sum_{k=-(n-1)/2}^{(n-1)/2} \widetilde{c}_k \mathrm{e}^{\mathrm{i} k x},
\]
assuming $n$ is odd.

\subsection{Trapezoidal rule}
We are going to approximate the Fourier coefficients $c_k$ by using the \emph{trapezoidal rule}.

\textbf{Trapezoidal rule:} Let $x_{j}$, $j = 0, \ldots, n$ be $n+1$ equally spaced points on the interval $[a, b]$. Hence, $x_j = a + jh$ with $h = (b-a)/n$. The $n+1$-point trapezoidal rule for approximating the integral 

\[
I[g] = \int_{a}^{b} g(x) \mathrm{d} x 
\]
is denoted by $I_n[g]$ and defined as

\[
I_n[g] := \frac{h}{2}\left(g(x_0) + 2g(x_1) + 2g(x_2) + \cdots + 2g(x_{n-1}) + g(x_{n})    \right)
\]
The trapezoidal rule is an example of a quadrature method, which are methods to approximate integrals by weighted sums.

Let's use the trapezoidal rule to approximate the Fourier coefficients.  Setting $a = 0$, $b = 2\pi$, $h = 2\pi/n$ and using the fact that $g(x) = f(x)\mathrm{e}^{-\mathrm{i}k x}$ is a $2\pi$-periodic function (since $f(x)$ is assumed to be $2\pi$-periodic), it follows that $g(x_0) = g(x_0 + 2\pi) = g(x_n)$ and we obtain the approximation


\begin{eqnarray}
c_k &=& \frac{1}{2\pi}\int_{0}^{2\pi} f(x) \mathrm{e}^{-\mathrm{i}k x} \mathrm{d} x \\
    &\approx & \frac{1}{2\pi}I_{n}\left[f\mathrm{e}^{-\mathrm{i}k x}\right]  \\
    &=& \frac{1}{n}\sum_{j = 0}^{n-1} f(x_j)\mathrm{e}^{-\mathrm{i}kx_j} \\
    &:=& \widetilde{c}_k
\end{eqnarray}


\end{document}
