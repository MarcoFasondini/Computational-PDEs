\documentclass[12pt,a4paper]{article}

\usepackage[a4paper,text={16.5cm,25.2cm},centering]{geometry}
\usepackage{lmodern}
\usepackage{amssymb,amsmath}
\usepackage{bm}
\usepackage{graphicx}
\usepackage{microtype}
\usepackage{hyperref}
\setlength{\parindent}{0pt}
\setlength{\parskip}{1.2ex}

\hypersetup
       {   pdfauthor = { Marco Fasondini },
           pdftitle={ foo },
           colorlinks=TRUE,
           linkcolor=black,
           citecolor=blue,
           urlcolor=blue
       }




\usepackage{upquote}
\usepackage{listings}
\usepackage{xcolor}
\lstset{
    basicstyle=\ttfamily\footnotesize,
    upquote=true,
    breaklines=true,
    breakindent=0pt,
    keepspaces=true,
    showspaces=false,
    columns=fullflexible,
    showtabs=false,
    showstringspaces=false,
    escapeinside={(*@}{@*)},
    extendedchars=true,
}
\newcommand{\HLJLt}[1]{#1}
\newcommand{\HLJLw}[1]{#1}
\newcommand{\HLJLe}[1]{#1}
\newcommand{\HLJLeB}[1]{#1}
\newcommand{\HLJLo}[1]{#1}
\newcommand{\HLJLk}[1]{\textcolor[RGB]{148,91,176}{\textbf{#1}}}
\newcommand{\HLJLkc}[1]{\textcolor[RGB]{59,151,46}{\textit{#1}}}
\newcommand{\HLJLkd}[1]{\textcolor[RGB]{214,102,97}{\textit{#1}}}
\newcommand{\HLJLkn}[1]{\textcolor[RGB]{148,91,176}{\textbf{#1}}}
\newcommand{\HLJLkp}[1]{\textcolor[RGB]{148,91,176}{\textbf{#1}}}
\newcommand{\HLJLkr}[1]{\textcolor[RGB]{148,91,176}{\textbf{#1}}}
\newcommand{\HLJLkt}[1]{\textcolor[RGB]{148,91,176}{\textbf{#1}}}
\newcommand{\HLJLn}[1]{#1}
\newcommand{\HLJLna}[1]{#1}
\newcommand{\HLJLnb}[1]{#1}
\newcommand{\HLJLnbp}[1]{#1}
\newcommand{\HLJLnc}[1]{#1}
\newcommand{\HLJLncB}[1]{#1}
\newcommand{\HLJLnd}[1]{\textcolor[RGB]{214,102,97}{#1}}
\newcommand{\HLJLne}[1]{#1}
\newcommand{\HLJLneB}[1]{#1}
\newcommand{\HLJLnf}[1]{\textcolor[RGB]{66,102,213}{#1}}
\newcommand{\HLJLnfm}[1]{\textcolor[RGB]{66,102,213}{#1}}
\newcommand{\HLJLnp}[1]{#1}
\newcommand{\HLJLnl}[1]{#1}
\newcommand{\HLJLnn}[1]{#1}
\newcommand{\HLJLno}[1]{#1}
\newcommand{\HLJLnt}[1]{#1}
\newcommand{\HLJLnv}[1]{#1}
\newcommand{\HLJLnvc}[1]{#1}
\newcommand{\HLJLnvg}[1]{#1}
\newcommand{\HLJLnvi}[1]{#1}
\newcommand{\HLJLnvm}[1]{#1}
\newcommand{\HLJLl}[1]{#1}
\newcommand{\HLJLld}[1]{\textcolor[RGB]{148,91,176}{\textit{#1}}}
\newcommand{\HLJLs}[1]{\textcolor[RGB]{201,61,57}{#1}}
\newcommand{\HLJLsa}[1]{\textcolor[RGB]{201,61,57}{#1}}
\newcommand{\HLJLsb}[1]{\textcolor[RGB]{201,61,57}{#1}}
\newcommand{\HLJLsc}[1]{\textcolor[RGB]{201,61,57}{#1}}
\newcommand{\HLJLsd}[1]{\textcolor[RGB]{201,61,57}{#1}}
\newcommand{\HLJLsdB}[1]{\textcolor[RGB]{201,61,57}{#1}}
\newcommand{\HLJLsdC}[1]{\textcolor[RGB]{201,61,57}{#1}}
\newcommand{\HLJLse}[1]{\textcolor[RGB]{59,151,46}{#1}}
\newcommand{\HLJLsh}[1]{\textcolor[RGB]{201,61,57}{#1}}
\newcommand{\HLJLsi}[1]{#1}
\newcommand{\HLJLso}[1]{\textcolor[RGB]{201,61,57}{#1}}
\newcommand{\HLJLsr}[1]{\textcolor[RGB]{201,61,57}{#1}}
\newcommand{\HLJLss}[1]{\textcolor[RGB]{201,61,57}{#1}}
\newcommand{\HLJLssB}[1]{\textcolor[RGB]{201,61,57}{#1}}
\newcommand{\HLJLnB}[1]{\textcolor[RGB]{59,151,46}{#1}}
\newcommand{\HLJLnbB}[1]{\textcolor[RGB]{59,151,46}{#1}}
\newcommand{\HLJLnfB}[1]{\textcolor[RGB]{59,151,46}{#1}}
\newcommand{\HLJLnh}[1]{\textcolor[RGB]{59,151,46}{#1}}
\newcommand{\HLJLni}[1]{\textcolor[RGB]{59,151,46}{#1}}
\newcommand{\HLJLnil}[1]{\textcolor[RGB]{59,151,46}{#1}}
\newcommand{\HLJLnoB}[1]{\textcolor[RGB]{59,151,46}{#1}}
\newcommand{\HLJLoB}[1]{\textcolor[RGB]{102,102,102}{\textbf{#1}}}
\newcommand{\HLJLow}[1]{\textcolor[RGB]{102,102,102}{\textbf{#1}}}
\newcommand{\HLJLp}[1]{#1}
\newcommand{\HLJLc}[1]{\textcolor[RGB]{153,153,119}{\textit{#1}}}
\newcommand{\HLJLch}[1]{\textcolor[RGB]{153,153,119}{\textit{#1}}}
\newcommand{\HLJLcm}[1]{\textcolor[RGB]{153,153,119}{\textit{#1}}}
\newcommand{\HLJLcp}[1]{\textcolor[RGB]{153,153,119}{\textit{#1}}}
\newcommand{\HLJLcpB}[1]{\textcolor[RGB]{153,153,119}{\textit{#1}}}
\newcommand{\HLJLcs}[1]{\textcolor[RGB]{153,153,119}{\textit{#1}}}
\newcommand{\HLJLcsB}[1]{\textcolor[RGB]{153,153,119}{\textit{#1}}}
\newcommand{\HLJLg}[1]{#1}
\newcommand{\HLJLgd}[1]{#1}
\newcommand{\HLJLge}[1]{#1}
\newcommand{\HLJLgeB}[1]{#1}
\newcommand{\HLJLgh}[1]{#1}
\newcommand{\HLJLgi}[1]{#1}
\newcommand{\HLJLgo}[1]{#1}
\newcommand{\HLJLgp}[1]{#1}
\newcommand{\HLJLgs}[1]{#1}
\newcommand{\HLJLgsB}[1]{#1}
\newcommand{\HLJLgt}[1]{#1}



\def\qqand{\qquad\hbox{and}\qquad}
\def\qqfor{\qquad\hbox{for}\qquad}
\def\qqas{\qquad\hbox{as}\qquad}
\def\half{ {1 \over 2} }
\def\D{ {\rm d} }
\def\I{ {\rm i} }
\def\E{ {\rm e} }
\def\C{ {\mathbb C} }
\def\R{ {\mathbb R} }
\def\bbR{ {\mathbb R} }
\def\H{ {\mathbb H} }
\def\Z{ {\mathbb Z} }
\def\CC{ {\cal C} }
\def\FF{ {\cal F} }
\def\HH{ {\cal H} }
\def\LL{ {\cal L} }
\def\vc#1{ {\mathbf #1} }
\def\bbC{ {\mathbb C} }



\def\fR{ f_{\rm R} }
\def\fL{ f_{\rm L} }

\def\qqqquad{\qquad\qquad}
\def\qqwhere{\qquad\hbox{where}\qquad}
\def\Res_#1{\underset{#1}{\rm Res}\,}
\def\sech{ {\rm sech}\, }
\def\acos{ {\rm acos}\, }
\def\asin{ {\rm asin}\, }
\def\atan{ {\rm atan}\, }
\def\Ei{ {\rm Ei}\, }
\def\upepsilon{\varepsilon}


\def\Xint#1{ \mathchoice
   {\XXint\displaystyle\textstyle{#1} }%
   {\XXint\textstyle\scriptstyle{#1} }%
   {\XXint\scriptstyle\scriptscriptstyle{#1} }%
   {\XXint\scriptscriptstyle\scriptscriptstyle{#1} }%
   \!\int}
\def\XXint#1#2#3{ {\setbox0=\hbox{$#1{#2#3}{\int}$}
     \vcenter{\hbox{$#2#3$}}\kern-.5\wd0} }
\def\ddashint{\Xint=}
\def\dashint{\Xint-}
% \def\dashint
\def\infdashint{\dashint_{-\infty}^\infty}




\def\addtab#1={#1\;&=}
\def\ccr{\\\addtab}
\def\ip<#1>{\left\langle{#1}\right\rangle}
\def\dx{\D x}
\def\dt{\D t}
\def\dz{\D z}
\def\ds{\D s}

\def\rR{ {\rm R} }
\def\rL{ {\rm L} }

\def\norm#1{\left\| #1 \right\|}

\def\pr(#1){\left({#1}\right)}
\def\br[#1]{\left[{#1}\right]}

\def\abs#1{\left|{#1}\right|}
\def\fpr(#1){\!\pr({#1})}

\def\sopmatrix#1{ \begin{pmatrix}#1\end{pmatrix} }

\def\endash{–}
\def\emdash{—}
\def\mdblksquare{\blacksquare}
\def\lgblksquare{\blacksquare}
\def\scre{\E}
\def\mapengine#1,#2.{\mapfunction{#1}\ifx\void#2\else\mapengine #2.\fi }

\def\map[#1]{\mapengine #1,\void.}

\def\mapenginesep_#1#2,#3.{\mapfunction{#2}\ifx\void#3\else#1\mapengine #3.\fi }

\def\mapsep_#1[#2]{\mapenginesep_{#1}#2,\void.}


\def\vcbr[#1]{\pr(#1)}


\def\bvect[#1,#2]{
{
\def\dots{\cdots}
\def\mapfunction##1{\ | \  ##1}
	\sopmatrix{
		 \,#1\map[#2]\,
	}
}
}



\def\vect[#1]{
{\def\dots{\ldots}
	\vcbr[{#1}]
} }

\def\vectt[#1]{
{\def\dots{\ldots}
	\vect[{#1}]^{\top}
} }

\def\Vectt[#1]{
{
\def\mapfunction##1{##1 \cr}
\def\dots{\vdots}
	\begin{pmatrix}
		\map[#1]
	\end{pmatrix}
} }

\def\addtab#1={#1\;&=}
\def\ccr{\\\addtab}

\def\questionequals{= \!\!\!\!\!\!{\scriptstyle ? \atop }\,\,\,}

\def\Ei{\rm Ei\,}

\begin{document}

\section{Chapter 3: Exercises Solutions}
\begin{itemize}
\item[1. ] In the method of separation of variables for the wave equation (see the lecture notes for Chapter 3), consider the cases $\lambda > 0$ and $\lambda = 0$.

\end{itemize}
Recall the boundary / initial value problem for the wave equation

\[
u_{tt} = u_{xx}, \qquad t > 0, \qquad -1 < x < 1, \qquad u(\pm 1, t) = 0,
\]
subject to the initial data

\[
u(x,0) = f(x), \qquad u_t(x,0) = g(x), \qquad x \in [-1, 1],
\]
where we assume that $f$ and $g$ are not both identically zero (because then the trivial solution $u(x,t) = 0$ solves the problem).

We showed in the lecture notes that the zero boundary conditions imply that $X(-1) = 0 = X(1)$.

If $\lambda > 0$, then

\[
X = c_1\cosh(\sqrt{\lambda}x) + c_2\sinh(\sqrt{\lambda}x)
\]
where $c_1$ and $c_2$ are arbitrary constants.  We require

\[
X(-1) = c_1\cosh(\sqrt{\lambda}) - c_2\sinh(\sqrt{\lambda}) = 0 =  c_1\cosh(\sqrt{\lambda}) + c_2\sinh(\sqrt{\lambda}).
\]
These equations can only be satisfied for $\lambda > 0$ if $c_1 = c_2 = 0$ (Why? See ($\star$) below).  Hence $X(x) = 0$ for $x \in [-1, 1]$ which implies that $u(x,t) = X(x)T(t) = 0$ for $t > 0$, however $u(x,0) = f(x)$ and $u_t(x,0) = g(x)$ and $f$ and $g$ are not both identically zero.   We conclude that there does not exist a continuous solution to the wave equation if $\lambda > 0$.

($\star$) Formulate the equations as a linear system, then 

\[
\begin{bmatrix}
\cosh(\sqrt{\lambda}) & -\sinh(\sqrt{\lambda}) \\
\cosh(\sqrt{\lambda}) & \sinh(\sqrt{\lambda})
\end{bmatrix}
\begin{bmatrix}
c_1 \\
c_2
\end{bmatrix} = 
\begin{bmatrix}
0 \\
0
\end{bmatrix}
\]
and it follows that $c_1 = c_2 = 0$ because the matrix is invertible; alternatively, it follows  that $c_1 = c_2 = 0$  since  $\sinh (\sqrt{\lambda})$ and $\cosh(\sqrt{\lambda})$ are linearly independent functions for $\lambda > 0$ (this can be seen, for example, by calculating the \href{https://en.wikipedia.org/wiki/Wronskian}{Wronkskian}).

If $\lambda = 0$, then

\[
X(x) = c_1x + c_0
\]
and we require 

\[
X(-1) = c_0 - c_1 = 0 = c_0 + c_1
\]
which implies that $c_1 = c_2 = 0$.   Hence $X(x) = 0$ for $x \in [-1, 1]$ which implies that $u(x,t) = X(x)T(t) = 0$ for $t > 0$, however $u(x,0) = f(x)$ and $u_t(x,0) = g(x)$ and $f$ and $g$ are not both identically zero.   We conclude that there does not exist a continuous solution to the wave equation if $\lambda = 0$.

\begin{itemize}
\item[2. ] We showed, using the method of separation of variables, that 

\end{itemize}
\[
     u(x,t) =  \left( \sin\pi t + \cos\pi t  \right)\sin \pi x
\]
is a solution to the wave equation with zero boundary conditions.  Use this solution to the wave equation to investigate the accuracy of the  numerical method discussed in the notes, i.e., 

\[
   \mathbf{u}^{i+1} = 2\mathbf{u}^{i} - \mathbf{u}^{i-1} + \tau^2 D_n^2\mathbf{u}^{i}, \qquad i = 0, \ldots, n_t-1.
\]
Let $\tau = 8/n_x^2$ and $n_t \tau = 2$, compute the maximum error of the method for $n_x = 2^3, 2^4, \ldots, 2^8$, plot it on a log-log scale and deduce the rate of decay of the error with $n_x$.


\begin{lstlisting}
(*@\HLJLk{using}@*) (*@\HLJLn{Plots}@*)(*@\HLJLp{,}@*) (*@\HLJLn{LinearAlgebra}@*)(*@\HLJLp{,}@*) (*@\HLJLn{FFTW}@*)
\end{lstlisting}


Here is a plot of the solution:


\begin{lstlisting}
(*@\HLJLnf{u}@*)(*@\HLJLp{(}@*)(*@\HLJLn{x}@*)(*@\HLJLp{,}@*)(*@\HLJLn{t}@*)(*@\HLJLp{)}@*) (*@\HLJLoB{=}@*)  (*@\HLJLp{(}@*)(*@\HLJLnf{sin}@*)(*@\HLJLp{(}@*)(*@\HLJLn{\ensuremath{\pi}}@*)(*@\HLJLoB{*}@*)(*@\HLJLn{t}@*)(*@\HLJLp{)}@*) (*@\HLJLoB{+}@*) (*@\HLJLnf{cos}@*)(*@\HLJLp{(}@*)(*@\HLJLn{\ensuremath{\pi}}@*)(*@\HLJLoB{*}@*)(*@\HLJLn{t}@*)(*@\HLJLp{))}@*)(*@\HLJLoB{*}@*)(*@\HLJLnf{sin}@*)(*@\HLJLp{(}@*)(*@\HLJLn{\ensuremath{\pi}}@*)(*@\HLJLoB{*}@*)(*@\HLJLn{x}@*)(*@\HLJLp{)}@*)
(*@\HLJLn{x}@*) (*@\HLJLoB{=}@*) (*@\HLJLnf{range}@*)(*@\HLJLp{(}@*)(*@\HLJLoB{-}@*)(*@\HLJLni{1}@*)(*@\HLJLp{,}@*)(*@\HLJLni{1}@*)(*@\HLJLp{,}@*)(*@\HLJLn{length}@*)(*@\HLJLoB{=}@*)(*@\HLJLni{101}@*)(*@\HLJLp{)}@*)
(*@\HLJLn{t}@*) (*@\HLJLoB{=}@*) (*@\HLJLnf{range}@*)(*@\HLJLp{(}@*)(*@\HLJLni{0}@*)(*@\HLJLp{,}@*)(*@\HLJLni{4}@*)(*@\HLJLp{,}@*)(*@\HLJLn{length}@*)(*@\HLJLoB{=}@*)(*@\HLJLni{101}@*)(*@\HLJLp{)}@*)
(*@\HLJLnf{surface}@*)(*@\HLJLp{(}@*)(*@\HLJLn{x}@*)(*@\HLJLp{,}@*)(*@\HLJLn{t}@*)(*@\HLJLp{,}@*)(*@\HLJLn{u}@*)(*@\HLJLp{,}@*) (*@\HLJLn{camera}@*)(*@\HLJLoB{=}@*)(*@\HLJLp{(}@*)(*@\HLJLni{30}@*)(*@\HLJLp{,}@*)(*@\HLJLni{40}@*)(*@\HLJLp{))}@*)
\end{lstlisting}

\includegraphics[width=\linewidth]{/figures/Chapter3_Exercises_Solutions_2_1.pdf}

\begin{lstlisting}
(*@\HLJLk{function}@*) (*@\HLJLnf{chebfft}@*)(*@\HLJLp{(}@*)(*@\HLJLn{f}@*)(*@\HLJLp{)}@*)
    (*@\HLJLn{n}@*) (*@\HLJLoB{=}@*) (*@\HLJLnf{length}@*)(*@\HLJLp{(}@*)(*@\HLJLn{f}@*)(*@\HLJLp{)}@*)(*@\HLJLoB{-}@*)(*@\HLJLni{1}@*)
    (*@\HLJLn{x}@*) (*@\HLJLoB{=}@*) (*@\HLJLn{cos}@*)(*@\HLJLoB{.}@*)(*@\HLJLp{((}@*)(*@\HLJLni{0}@*)(*@\HLJLoB{:}@*)(*@\HLJLn{n}@*)(*@\HLJLp{)}@*)(*@\HLJLoB{*}@*)(*@\HLJLn{\ensuremath{\pi}}@*)(*@\HLJLoB{/}@*)(*@\HLJLn{n}@*)(*@\HLJLp{)}@*) (*@\HLJLcs{{\#}}@*) (*@\HLJLcs{Chebyshev}@*) (*@\HLJLcs{points}@*)
    (*@\HLJLn{ii}@*) (*@\HLJLoB{=}@*) (*@\HLJLni{0}@*)(*@\HLJLoB{:}@*)(*@\HLJLn{n}@*)(*@\HLJLoB{-}@*)(*@\HLJLni{1}@*)
    (*@\HLJLn{q}@*) (*@\HLJLoB{=}@*) (*@\HLJLp{[}@*)(*@\HLJLn{f}@*)(*@\HLJLp{;}@*) (*@\HLJLn{f}@*)(*@\HLJLp{[}@*)(*@\HLJLn{n}@*)(*@\HLJLoB{:-}@*)(*@\HLJLni{1}@*)(*@\HLJLoB{:}@*)(*@\HLJLni{2}@*)(*@\HLJLp{]]}@*)      (*@\HLJLcs{{\#}}@*) (*@\HLJLcs{transform}@*) (*@\HLJLcs{x}@*) (*@\HLJLcs{->}@*) (*@\HLJLcs{\ensuremath{\theta}}@*)    
    (*@\HLJLcs{{\#}}@*) (*@\HLJLcs{differentiate}@*) (*@\HLJLcs{the}@*) (*@\HLJLcs{interpolant}@*) (*@\HLJLcs{q\ensuremath{\_n}}@*) (*@\HLJLcs{in}@*) (*@\HLJLcs{coefficient}@*) (*@\HLJLcs{space}@*) (*@\HLJLcs{and}@*) (*@\HLJLcs{map}@*) (*@\HLJLcs{back}@*) (*@\HLJLcs{to}@*) (*@\HLJLcs{function}@*) (*@\HLJLcs{values}@*)
    (*@\HLJLn{cq}@*) (*@\HLJLoB{=}@*) (*@\HLJLn{real}@*)(*@\HLJLoB{.}@*)(*@\HLJLp{(}@*)(*@\HLJLnf{fft}@*)(*@\HLJLp{(}@*)(*@\HLJLn{q}@*)(*@\HLJLp{))}@*)
    (*@\HLJLn{dq}@*) (*@\HLJLoB{=}@*) (*@\HLJLn{real}@*)(*@\HLJLoB{.}@*)(*@\HLJLp{(}@*)(*@\HLJLnf{ifft}@*)(*@\HLJLp{(}@*)(*@\HLJLn{im}@*)(*@\HLJLoB{*}@*)(*@\HLJLp{[}@*)(*@\HLJLn{ii}@*)(*@\HLJLp{;}@*) (*@\HLJLni{0}@*)(*@\HLJLp{;}@*) (*@\HLJLni{1}@*)(*@\HLJLoB{-}@*)(*@\HLJLn{n}@*)(*@\HLJLoB{:-}@*)(*@\HLJLni{1}@*)(*@\HLJLp{]}@*) (*@\HLJLoB{.*}@*)(*@\HLJLn{cq}@*)(*@\HLJLp{))}@*)
    (*@\HLJLn{df}@*) (*@\HLJLoB{=}@*) (*@\HLJLnf{zeros}@*)(*@\HLJLp{(}@*)(*@\HLJLn{n}@*)(*@\HLJLoB{+}@*)(*@\HLJLni{1}@*)(*@\HLJLp{,}@*)(*@\HLJLni{1}@*)(*@\HLJLp{)}@*)
    (*@\HLJLcs{{\#}}@*) (*@\HLJLcs{Compute}@*) (*@\HLJLcs{approximations}@*) (*@\HLJLcs{to}@*) (*@\HLJLcs{f{\textquotesingle}}@*) (*@\HLJLcs{at}@*) (*@\HLJLcs{the}@*) (*@\HLJLcs{interior}@*) (*@\HLJLcs{points}@*)
    (*@\HLJLn{df}@*)(*@\HLJLp{[}@*)(*@\HLJLni{2}@*)(*@\HLJLoB{:}@*)(*@\HLJLn{n}@*)(*@\HLJLp{]}@*) (*@\HLJLoB{=}@*) (*@\HLJLoB{-}@*)(*@\HLJLn{dq}@*)(*@\HLJLp{[}@*)(*@\HLJLni{2}@*)(*@\HLJLoB{:}@*)(*@\HLJLn{n}@*)(*@\HLJLp{]}@*) (*@\HLJLoB{./}@*)(*@\HLJLn{sqrt}@*)(*@\HLJLoB{.}@*)(*@\HLJLp{(}@*)(*@\HLJLni{1}@*) (*@\HLJLoB{.-}@*) (*@\HLJLn{x}@*)(*@\HLJLp{[}@*)(*@\HLJLni{2}@*)(*@\HLJLoB{:}@*)(*@\HLJLn{n}@*)(*@\HLJLp{]}@*)(*@\HLJLoB{.{\textasciicircum}}@*)(*@\HLJLni{2}@*)(*@\HLJLp{)}@*)    (*@\HLJLcs{{\#}}@*) (*@\HLJLcs{transform}@*) (*@\HLJLcs{\ensuremath{\theta}}@*) (*@\HLJLcs{->}@*) (*@\HLJLcs{x}@*)   
    (*@\HLJLcs{{\#}}@*) (*@\HLJLcs{At}@*) (*@\HLJLcs{the}@*) (*@\HLJLcs{boundary}@*) (*@\HLJLcs{points}@*)
    (*@\HLJLn{df}@*)(*@\HLJLp{[}@*)(*@\HLJLni{1}@*)(*@\HLJLp{]}@*) (*@\HLJLoB{=}@*) (*@\HLJLnf{sum}@*)(*@\HLJLp{(}@*)(*@\HLJLn{ii}@*)(*@\HLJLoB{.{\textasciicircum}}@*)(*@\HLJLni{2}@*) (*@\HLJLoB{.*}@*)(*@\HLJLn{cq}@*)(*@\HLJLp{[}@*)(*@\HLJLni{1}@*)(*@\HLJLoB{:}@*)(*@\HLJLn{n}@*)(*@\HLJLp{])}@*)(*@\HLJLoB{/}@*)(*@\HLJLn{n}@*) (*@\HLJLoB{+}@*) (*@\HLJLnfB{.5}@*)(*@\HLJLoB{*}@*)(*@\HLJLn{n}@*)(*@\HLJLoB{*}@*)(*@\HLJLn{cq}@*)(*@\HLJLp{[}@*)(*@\HLJLn{n}@*)(*@\HLJLoB{+}@*)(*@\HLJLni{1}@*)(*@\HLJLp{]}@*)     
    (*@\HLJLn{df}@*)(*@\HLJLp{[}@*)(*@\HLJLn{n}@*)(*@\HLJLoB{+}@*)(*@\HLJLni{1}@*)(*@\HLJLp{]}@*) (*@\HLJLoB{=}@*) (*@\HLJLnf{sum}@*)(*@\HLJLp{((}@*)(*@\HLJLoB{-}@*)(*@\HLJLni{1}@*)(*@\HLJLp{)}@*) (*@\HLJLoB{.{\textasciicircum}}@*)(*@\HLJLp{(}@*)(*@\HLJLn{ii}@*) (*@\HLJLoB{.+}@*)(*@\HLJLni{1}@*)(*@\HLJLp{)}@*) (*@\HLJLoB{.*}@*) (*@\HLJLn{ii}@*)(*@\HLJLoB{.{\textasciicircum}}@*)(*@\HLJLni{2}@*) (*@\HLJLoB{.*}@*)(*@\HLJLn{cq}@*)(*@\HLJLp{[}@*)(*@\HLJLni{1}@*)(*@\HLJLoB{:}@*)(*@\HLJLn{n}@*)(*@\HLJLp{])}@*)(*@\HLJLoB{/}@*)(*@\HLJLn{n}@*) (*@\HLJLoB{+}@*)
              (*@\HLJLnfB{.5}@*)(*@\HLJLoB{*}@*)(*@\HLJLp{(}@*)(*@\HLJLoB{-}@*)(*@\HLJLni{1}@*)(*@\HLJLp{)}@*)(*@\HLJLoB{{\textasciicircum}}@*)(*@\HLJLp{(}@*)(*@\HLJLn{n}@*)(*@\HLJLoB{+}@*)(*@\HLJLni{1}@*)(*@\HLJLp{)}@*)(*@\HLJLoB{*}@*)(*@\HLJLn{n}@*)(*@\HLJLoB{*}@*)(*@\HLJLn{cq}@*)(*@\HLJLp{[}@*)(*@\HLJLn{n}@*)(*@\HLJLoB{+}@*)(*@\HLJLni{1}@*)(*@\HLJLp{]}@*)
    (*@\HLJLn{df}@*)
(*@\HLJLk{end}@*)
\end{lstlisting}

\begin{lstlisting}
chebfft (generic function with 1 method)
\end{lstlisting}


First plot the error for on $t \in [0,2]$ to check that the method works as expected:


\begin{lstlisting}
(*@\HLJLn{nx}@*) (*@\HLJLoB{=}@*) (*@\HLJLni{32}@*) (*@\HLJLcs{{\#}}@*) (*@\HLJLcs{should}@*) (*@\HLJLcs{be}@*) (*@\HLJLcs{even}@*) (*@\HLJLcs{because}@*) (*@\HLJLcs{chebfft}@*) (*@\HLJLcs{only}@*) (*@\HLJLcs{works}@*) (*@\HLJLcs{with}@*) (*@\HLJLcs{even}@*) (*@\HLJLcs{inputs}@*)
(*@\HLJLn{x}@*) (*@\HLJLoB{=}@*) (*@\HLJLn{cos}@*)(*@\HLJLoB{.}@*)(*@\HLJLp{(}@*)(*@\HLJLn{\ensuremath{\pi}}@*)(*@\HLJLoB{*}@*)(*@\HLJLp{(}@*)(*@\HLJLni{0}@*)(*@\HLJLoB{:}@*)(*@\HLJLn{nx}@*)(*@\HLJLp{)}@*)(*@\HLJLoB{/}@*)(*@\HLJLn{nx}@*)(*@\HLJLp{)}@*)
(*@\HLJLn{\ensuremath{\tau}}@*) (*@\HLJLoB{=}@*) (*@\HLJLni{8}@*)(*@\HLJLoB{/}@*)(*@\HLJLn{nx}@*)(*@\HLJLoB{{\textasciicircum}}@*)(*@\HLJLni{2}@*)  (*@\HLJLcs{{\#}}@*) (*@\HLJLcs{will}@*) (*@\HLJLcs{see}@*) (*@\HLJLcs{later}@*) (*@\HLJLcs{why}@*) (*@\HLJLcs{we}@*) (*@\HLJLcs{choose}@*) (*@\HLJLcs{this}@*) (*@\HLJLcs{step}@*) (*@\HLJLcs{size}@*)
(*@\HLJLnf{u}@*)(*@\HLJLp{(}@*)(*@\HLJLn{x}@*)(*@\HLJLp{,}@*)(*@\HLJLn{t}@*)(*@\HLJLp{)}@*) (*@\HLJLoB{=}@*)  (*@\HLJLp{(}@*)(*@\HLJLnf{sin}@*)(*@\HLJLp{(}@*)(*@\HLJLn{\ensuremath{\pi}}@*)(*@\HLJLoB{*}@*)(*@\HLJLn{t}@*)(*@\HLJLp{)}@*) (*@\HLJLoB{+}@*) (*@\HLJLnf{cos}@*)(*@\HLJLp{(}@*)(*@\HLJLn{\ensuremath{\pi}}@*)(*@\HLJLoB{*}@*)(*@\HLJLn{t}@*)(*@\HLJLp{))}@*)(*@\HLJLoB{*}@*)(*@\HLJLnf{sin}@*)(*@\HLJLp{(}@*)(*@\HLJLn{\ensuremath{\pi}}@*)(*@\HLJLoB{*}@*)(*@\HLJLn{x}@*)(*@\HLJLp{)}@*) (*@\HLJLcs{{\#}}@*) (*@\HLJLcs{exact}@*) (*@\HLJLcs{solution}@*)
(*@\HLJLn{v}@*) (*@\HLJLoB{=}@*) (*@\HLJLn{u}@*)(*@\HLJLoB{.}@*)(*@\HLJLp{(}@*)(*@\HLJLn{x}@*)(*@\HLJLp{,}@*)(*@\HLJLni{0}@*)(*@\HLJLp{)}@*) 
(*@\HLJLn{vold}@*) (*@\HLJLoB{=}@*) (*@\HLJLn{u}@*)(*@\HLJLoB{.}@*)(*@\HLJLp{(}@*)(*@\HLJLn{x}@*)(*@\HLJLp{,}@*)(*@\HLJLoB{-}@*)(*@\HLJLn{\ensuremath{\tau}}@*)(*@\HLJLp{)}@*)
(*@\HLJLn{tmax}@*) (*@\HLJLoB{=}@*) (*@\HLJLni{4}@*) 
(*@\HLJLn{nt}@*) (*@\HLJLoB{=}@*) (*@\HLJLnf{Int64}@*)(*@\HLJLp{(}@*)(*@\HLJLnf{round}@*)(*@\HLJLp{(}@*)(*@\HLJLn{tmax}@*)(*@\HLJLoB{/}@*)(*@\HLJLn{\ensuremath{\tau}}@*)(*@\HLJLp{))}@*)
(*@\HLJLn{error}@*) (*@\HLJLoB{=}@*) (*@\HLJLnf{zeros}@*)(*@\HLJLp{(}@*)(*@\HLJLn{nt}@*)(*@\HLJLp{)}@*)
(*@\HLJLk{for}@*) (*@\HLJLn{i}@*) (*@\HLJLoB{=}@*) (*@\HLJLni{1}@*)(*@\HLJLoB{:}@*)(*@\HLJLn{nt}@*)
    (*@\HLJLkd{global}@*) (*@\HLJLn{v}@*)
    (*@\HLJLkd{global}@*) (*@\HLJLn{vold}@*)
    (*@\HLJLn{w}@*) (*@\HLJLoB{=}@*) (*@\HLJLnf{chebfft}@*)(*@\HLJLp{(}@*)(*@\HLJLnf{chebfft}@*)(*@\HLJLp{(}@*)(*@\HLJLn{v}@*)(*@\HLJLp{))}@*)  (*@\HLJLcs{{\#}}@*) (*@\HLJLcs{second}@*) (*@\HLJLcs{spatial}@*) (*@\HLJLcs{derivative}@*) (*@\HLJLcs{via}@*) (*@\HLJLcs{FFT}@*)
    (*@\HLJLcs{{\#}}@*) (*@\HLJLcs{zero}@*) (*@\HLJLcs{boundary}@*) (*@\HLJLcs{conditions}@*)
    (*@\HLJLn{w}@*)(*@\HLJLp{[}@*)(*@\HLJLni{1}@*)(*@\HLJLp{]}@*)(*@\HLJLoB{=}@*)(*@\HLJLni{0}@*)(*@\HLJLp{;}@*) (*@\HLJLn{w}@*)(*@\HLJLp{[}@*)(*@\HLJLn{nx}@*)(*@\HLJLoB{+}@*)(*@\HLJLni{1}@*)(*@\HLJLp{]}@*) (*@\HLJLoB{=}@*) (*@\HLJLni{0}@*)
    (*@\HLJLcs{{\#}}@*) (*@\HLJLcs{Leapfrog}@*) (*@\HLJLcs{method}@*)
    (*@\HLJLn{vnew}@*) (*@\HLJLoB{=}@*) (*@\HLJLni{2}@*)(*@\HLJLoB{*}@*)(*@\HLJLn{v}@*) (*@\HLJLoB{-}@*) (*@\HLJLn{vold}@*) (*@\HLJLoB{+}@*) (*@\HLJLn{\ensuremath{\tau}}@*)(*@\HLJLoB{{\textasciicircum}}@*)(*@\HLJLni{2}@*)(*@\HLJLoB{*}@*)(*@\HLJLn{w}@*) 
    (*@\HLJLcs{{\#}}@*) (*@\HLJLcs{update}@*)
    (*@\HLJLn{vold}@*) (*@\HLJLoB{=}@*) (*@\HLJLn{v}@*)(*@\HLJLp{;}@*) (*@\HLJLn{v}@*) (*@\HLJLoB{=}@*) (*@\HLJLn{vnew}@*)
    (*@\HLJLcs{{\#}}@*) (*@\HLJLcs{compute}@*) (*@\HLJLcs{error}@*)
    (*@\HLJLn{error}@*)(*@\HLJLp{[}@*)(*@\HLJLn{i}@*)(*@\HLJLp{]}@*) (*@\HLJLoB{=}@*) (*@\HLJLnf{norm}@*)(*@\HLJLp{((}@*)(*@\HLJLn{v}@*) (*@\HLJLoB{-}@*) (*@\HLJLn{u}@*)(*@\HLJLoB{.}@*)(*@\HLJLp{(}@*)(*@\HLJLn{x}@*)(*@\HLJLp{,}@*)(*@\HLJLn{\ensuremath{\tau}}@*)(*@\HLJLoB{*}@*)(*@\HLJLn{i}@*)(*@\HLJLp{)),}@*)(*@\HLJLn{Inf}@*)(*@\HLJLp{)}@*)
(*@\HLJLk{end}@*)
\end{lstlisting}


\begin{lstlisting}
(*@\HLJLnf{plot}@*)(*@\HLJLp{(}@*)(*@\HLJLn{\ensuremath{\tau}}@*)(*@\HLJLoB{*}@*)(*@\HLJLp{(}@*)(*@\HLJLni{1}@*)(*@\HLJLoB{:}@*)(*@\HLJLn{nt}@*)(*@\HLJLp{),}@*)(*@\HLJLn{error}@*)(*@\HLJLp{;}@*)(*@\HLJLn{yscale}@*)(*@\HLJLoB{=:}@*)(*@\HLJLn{log10}@*)(*@\HLJLp{,}@*)(*@\HLJLn{label}@*)(*@\HLJLoB{=}@*)(*@\HLJLkc{false}@*)(*@\HLJLp{)}@*)
\end{lstlisting}

\includegraphics[width=\linewidth]{/figures/Chapter3_Exercises_Solutions_5_1.pdf}

\begin{lstlisting}
(*@\HLJLn{Nv}@*) (*@\HLJLoB{=}@*) (*@\HLJLni{2}@*) (*@\HLJLoB{.{\textasciicircum}}@*)(*@\HLJLp{(}@*)(*@\HLJLni{3}@*)(*@\HLJLoB{:}@*)(*@\HLJLni{8}@*)(*@\HLJLp{)}@*)
(*@\HLJLn{tmax}@*) (*@\HLJLoB{=}@*) (*@\HLJLni{2}@*) 
(*@\HLJLn{maxerr}@*) (*@\HLJLoB{=}@*) (*@\HLJLp{[]}@*)
(*@\HLJLk{for}@*) (*@\HLJLn{nx}@*) (*@\HLJLoB{=}@*) (*@\HLJLn{Nv}@*)
    (*@\HLJLn{x}@*) (*@\HLJLoB{=}@*) (*@\HLJLn{cos}@*)(*@\HLJLoB{.}@*)(*@\HLJLp{(}@*)(*@\HLJLn{\ensuremath{\pi}}@*)(*@\HLJLoB{*}@*)(*@\HLJLp{(}@*)(*@\HLJLni{0}@*)(*@\HLJLoB{:}@*)(*@\HLJLn{nx}@*)(*@\HLJLp{)}@*)(*@\HLJLoB{/}@*)(*@\HLJLn{nx}@*)(*@\HLJLp{)}@*)
    (*@\HLJLn{\ensuremath{\tau}}@*) (*@\HLJLoB{=}@*) (*@\HLJLni{8}@*)(*@\HLJLoB{/}@*)(*@\HLJLn{nx}@*)(*@\HLJLoB{{\textasciicircum}}@*)(*@\HLJLni{2}@*)  (*@\HLJLcs{{\#}}@*) (*@\HLJLcs{will}@*) (*@\HLJLcs{see}@*) (*@\HLJLcs{later}@*) (*@\HLJLcs{why}@*) (*@\HLJLcs{we}@*) (*@\HLJLcs{choose}@*) (*@\HLJLcs{this}@*) (*@\HLJLcs{step}@*) (*@\HLJLcs{size}@*)
    (*@\HLJLn{v}@*) (*@\HLJLoB{=}@*) (*@\HLJLn{u}@*)(*@\HLJLoB{.}@*)(*@\HLJLp{(}@*)(*@\HLJLn{x}@*)(*@\HLJLp{,}@*)(*@\HLJLni{0}@*)(*@\HLJLp{)}@*) 
    (*@\HLJLn{vold}@*) (*@\HLJLoB{=}@*) (*@\HLJLn{u}@*)(*@\HLJLoB{.}@*)(*@\HLJLp{(}@*)(*@\HLJLn{x}@*)(*@\HLJLp{,}@*)(*@\HLJLoB{-}@*)(*@\HLJLn{\ensuremath{\tau}}@*)(*@\HLJLp{)}@*)
    (*@\HLJLn{nt}@*) (*@\HLJLoB{=}@*) (*@\HLJLnd{@show}@*) (*@\HLJLnf{Int64}@*)(*@\HLJLp{(}@*)(*@\HLJLnf{round}@*)(*@\HLJLp{(}@*)(*@\HLJLn{tmax}@*)(*@\HLJLoB{/}@*)(*@\HLJLn{\ensuremath{\tau}}@*)(*@\HLJLp{))}@*)
    (*@\HLJLn{error}@*) (*@\HLJLoB{=}@*) (*@\HLJLnf{zeros}@*)(*@\HLJLp{(}@*)(*@\HLJLn{nt}@*)(*@\HLJLp{)}@*)
(*@\HLJLnd{@time}@*) (*@\HLJLk{begin}@*)
    (*@\HLJLk{for}@*) (*@\HLJLn{i}@*) (*@\HLJLoB{=}@*) (*@\HLJLni{1}@*)(*@\HLJLoB{:}@*)(*@\HLJLn{nt}@*)
        (*@\HLJLcs{{\#}global}@*) (*@\HLJLcs{x,}@*) (*@\HLJLcs{\ensuremath{\tau},}@*) (*@\HLJLcs{v,}@*) (*@\HLJLcs{vold}@*)
        (*@\HLJLn{w}@*) (*@\HLJLoB{=}@*) (*@\HLJLnf{chebfft}@*)(*@\HLJLp{(}@*)(*@\HLJLnf{chebfft}@*)(*@\HLJLp{(}@*)(*@\HLJLn{v}@*)(*@\HLJLp{))}@*)  (*@\HLJLcs{{\#}}@*) (*@\HLJLcs{second}@*) (*@\HLJLcs{spatial}@*) (*@\HLJLcs{derivative}@*) (*@\HLJLcs{via}@*) (*@\HLJLcs{FFT}@*)
        (*@\HLJLcs{{\#}}@*) (*@\HLJLcs{zero}@*) (*@\HLJLcs{boundary}@*) (*@\HLJLcs{conditions}@*)
        (*@\HLJLn{w}@*)(*@\HLJLp{[}@*)(*@\HLJLni{1}@*)(*@\HLJLp{]}@*)(*@\HLJLoB{=}@*)(*@\HLJLni{0}@*)(*@\HLJLp{;}@*) (*@\HLJLn{w}@*)(*@\HLJLp{[}@*)(*@\HLJLn{nx}@*)(*@\HLJLoB{+}@*)(*@\HLJLni{1}@*)(*@\HLJLp{]}@*) (*@\HLJLoB{=}@*) (*@\HLJLni{0}@*)
        (*@\HLJLcs{{\#}}@*) (*@\HLJLcs{Leapfrog}@*) (*@\HLJLcs{method}@*)
        (*@\HLJLn{vnew}@*) (*@\HLJLoB{=}@*) (*@\HLJLni{2}@*)(*@\HLJLoB{*}@*)(*@\HLJLn{v}@*) (*@\HLJLoB{-}@*) (*@\HLJLn{vold}@*) (*@\HLJLoB{+}@*) (*@\HLJLn{\ensuremath{\tau}}@*)(*@\HLJLoB{{\textasciicircum}}@*)(*@\HLJLni{2}@*)(*@\HLJLoB{*}@*)(*@\HLJLn{w}@*) 
        (*@\HLJLcs{{\#}}@*) (*@\HLJLcs{update}@*)
        (*@\HLJLn{vold}@*) (*@\HLJLoB{=}@*) (*@\HLJLn{v}@*)(*@\HLJLp{;}@*) (*@\HLJLn{v}@*) (*@\HLJLoB{=}@*) (*@\HLJLn{vnew}@*)
        (*@\HLJLcs{{\#}}@*) (*@\HLJLcs{compute}@*) (*@\HLJLcs{error}@*)
        (*@\HLJLn{error}@*)(*@\HLJLp{[}@*)(*@\HLJLn{i}@*)(*@\HLJLp{]}@*) (*@\HLJLoB{=}@*) (*@\HLJLnf{norm}@*)(*@\HLJLp{((}@*)(*@\HLJLn{v}@*) (*@\HLJLoB{-}@*) (*@\HLJLn{u}@*)(*@\HLJLoB{.}@*)(*@\HLJLp{(}@*)(*@\HLJLn{x}@*)(*@\HLJLp{,}@*)(*@\HLJLn{\ensuremath{\tau}}@*)(*@\HLJLoB{*}@*)(*@\HLJLn{i}@*)(*@\HLJLp{)),}@*)(*@\HLJLn{Inf}@*)(*@\HLJLp{)}@*)
    (*@\HLJLk{end}@*)
    (*@\HLJLk{end}@*)
    (*@\HLJLnf{push!}@*)(*@\HLJLp{(}@*)(*@\HLJLn{maxerr}@*)(*@\HLJLp{,}@*)(*@\HLJLnf{norm}@*)(*@\HLJLp{(}@*)(*@\HLJLn{error}@*)(*@\HLJLp{,}@*)(*@\HLJLn{Inf}@*)(*@\HLJLp{))}@*)
(*@\HLJLk{end}@*)
\end{lstlisting}

\begin{lstlisting}
Int64(round(tmax / (*@\ensuremath{\tau}@*))) = 16
  0.001718 seconds (3.74 k allocations: 294.250 KiB)
Int64(round(tmax / (*@\ensuremath{\tau}@*))) = 64
  0.005428 seconds (14.97 k allocations: 1.477 MiB)
Int64(round(tmax / (*@\ensuremath{\tau}@*))) = 256
  0.129767 seconds (59.90 k allocations: 8.574 MiB, 22.24(*@{{\%}}@*) gc time)
Int64(round(tmax / (*@\ensuremath{\tau}@*))) = 1024
  0.344497 seconds (240.64 k allocations: 55.812 MiB)
Int64(round(tmax / (*@\ensuremath{\tau}@*))) = 4096
  1.733067 seconds (965.63 k allocations: 393.296 MiB, 5.95(*@{{\%}}@*) gc time)
Int64(round(tmax / (*@\ensuremath{\tau}@*))) = 16384
  8.285870 seconds (3.87 M allocations: 2.886 GiB, 7.01(*@{{\%}}@*) gc time)
\end{lstlisting}


\begin{lstlisting}
(*@\HLJLnf{scatter}@*)(*@\HLJLp{(}@*)(*@\HLJLn{Nv}@*)(*@\HLJLp{,}@*)(*@\HLJLn{maxerr}@*)(*@\HLJLp{;}@*)(*@\HLJLn{yscale}@*)(*@\HLJLoB{=:}@*)(*@\HLJLn{log10}@*)(*@\HLJLp{,}@*)(*@\HLJLn{xscale}@*)(*@\HLJLoB{=:}@*)(*@\HLJLn{log10}@*)(*@\HLJLp{,}@*)(*@\HLJLn{xlabel}@*)(*@\HLJLoB{=}@*)(*@\HLJLs{"{}n\ensuremath{\_x}"{}}@*)(*@\HLJLp{,}@*)(*@\HLJLn{ylabel}@*)(*@\HLJLoB{=}@*)(*@\HLJLs{"{}max}@*) (*@\HLJLs{error"{}}@*)(*@\HLJLp{,}@*)(*@\HLJLn{label}@*)(*@\HLJLoB{=}@*)(*@\HLJLkc{false}@*)(*@\HLJLp{)}@*)
\end{lstlisting}

\includegraphics[width=\linewidth]{/figures/Chapter3_Exercises_Solutions_7_1.pdf}

To estimate the rate of decay of the error, we estimate the slope of the line through the points (using least squares fitting, for example). Here, is a simple estimate: 


\begin{lstlisting}
(*@\HLJLp{(}@*)(*@\HLJLnf{log}@*)(*@\HLJLp{(}@*)(*@\HLJLn{maxerr}@*)(*@\HLJLp{[}@*)(*@\HLJLk{end}@*)(*@\HLJLp{])}@*)(*@\HLJLoB{-}@*)(*@\HLJLnf{log}@*)(*@\HLJLp{(}@*)(*@\HLJLn{maxerr}@*)(*@\HLJLp{[}@*)(*@\HLJLk{end}@*)(*@\HLJLoB{-}@*)(*@\HLJLni{2}@*)(*@\HLJLp{]))}@*)(*@\HLJLoB{/}@*)(*@\HLJLp{(}@*)(*@\HLJLnf{log}@*)(*@\HLJLp{(}@*)(*@\HLJLn{Nv}@*)(*@\HLJLp{[}@*)(*@\HLJLk{end}@*)(*@\HLJLp{])}@*) (*@\HLJLoB{-}@*) (*@\HLJLnf{log}@*)(*@\HLJLp{(}@*)(*@\HLJLn{Nv}@*)(*@\HLJLp{[}@*)(*@\HLJLk{end}@*)(*@\HLJLoB{-}@*)(*@\HLJLni{2}@*)(*@\HLJLp{]))}@*)
\end{lstlisting}

\begin{lstlisting}
-3.999623238274797
\end{lstlisting}


This suggests the error decays at the rate $\mathcal{O}(n_x^{-4})$ as $n_x \to \infty$.



\end{document}
