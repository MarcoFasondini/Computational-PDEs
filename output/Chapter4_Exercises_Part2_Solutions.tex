\documentclass[12pt,a4paper]{article}

\usepackage[a4paper,text={16.5cm,25.2cm},centering]{geometry}
\usepackage{lmodern}
\usepackage{amssymb,amsmath}
\usepackage{bm}
\usepackage{graphicx}
\usepackage{microtype}
\usepackage{hyperref}
\setlength{\parindent}{0pt}
\setlength{\parskip}{1.2ex}

\hypersetup
       {   pdfauthor = { Marco Fasondini },
           pdftitle={ foo },
           colorlinks=TRUE,
           linkcolor=black,
           citecolor=blue,
           urlcolor=blue
       }




\usepackage{upquote}
\usepackage{listings}
\usepackage{xcolor}
\lstset{
    basicstyle=\ttfamily\footnotesize,
    upquote=true,
    breaklines=true,
    breakindent=0pt,
    keepspaces=true,
    showspaces=false,
    columns=fullflexible,
    showtabs=false,
    showstringspaces=false,
    escapeinside={(*@}{@*)},
    extendedchars=true,
}
\newcommand{\HLJLt}[1]{#1}
\newcommand{\HLJLw}[1]{#1}
\newcommand{\HLJLe}[1]{#1}
\newcommand{\HLJLeB}[1]{#1}
\newcommand{\HLJLo}[1]{#1}
\newcommand{\HLJLk}[1]{\textcolor[RGB]{148,91,176}{\textbf{#1}}}
\newcommand{\HLJLkc}[1]{\textcolor[RGB]{59,151,46}{\textit{#1}}}
\newcommand{\HLJLkd}[1]{\textcolor[RGB]{214,102,97}{\textit{#1}}}
\newcommand{\HLJLkn}[1]{\textcolor[RGB]{148,91,176}{\textbf{#1}}}
\newcommand{\HLJLkp}[1]{\textcolor[RGB]{148,91,176}{\textbf{#1}}}
\newcommand{\HLJLkr}[1]{\textcolor[RGB]{148,91,176}{\textbf{#1}}}
\newcommand{\HLJLkt}[1]{\textcolor[RGB]{148,91,176}{\textbf{#1}}}
\newcommand{\HLJLn}[1]{#1}
\newcommand{\HLJLna}[1]{#1}
\newcommand{\HLJLnb}[1]{#1}
\newcommand{\HLJLnbp}[1]{#1}
\newcommand{\HLJLnc}[1]{#1}
\newcommand{\HLJLncB}[1]{#1}
\newcommand{\HLJLnd}[1]{\textcolor[RGB]{214,102,97}{#1}}
\newcommand{\HLJLne}[1]{#1}
\newcommand{\HLJLneB}[1]{#1}
\newcommand{\HLJLnf}[1]{\textcolor[RGB]{66,102,213}{#1}}
\newcommand{\HLJLnfm}[1]{\textcolor[RGB]{66,102,213}{#1}}
\newcommand{\HLJLnp}[1]{#1}
\newcommand{\HLJLnl}[1]{#1}
\newcommand{\HLJLnn}[1]{#1}
\newcommand{\HLJLno}[1]{#1}
\newcommand{\HLJLnt}[1]{#1}
\newcommand{\HLJLnv}[1]{#1}
\newcommand{\HLJLnvc}[1]{#1}
\newcommand{\HLJLnvg}[1]{#1}
\newcommand{\HLJLnvi}[1]{#1}
\newcommand{\HLJLnvm}[1]{#1}
\newcommand{\HLJLl}[1]{#1}
\newcommand{\HLJLld}[1]{\textcolor[RGB]{148,91,176}{\textit{#1}}}
\newcommand{\HLJLs}[1]{\textcolor[RGB]{201,61,57}{#1}}
\newcommand{\HLJLsa}[1]{\textcolor[RGB]{201,61,57}{#1}}
\newcommand{\HLJLsb}[1]{\textcolor[RGB]{201,61,57}{#1}}
\newcommand{\HLJLsc}[1]{\textcolor[RGB]{201,61,57}{#1}}
\newcommand{\HLJLsd}[1]{\textcolor[RGB]{201,61,57}{#1}}
\newcommand{\HLJLsdB}[1]{\textcolor[RGB]{201,61,57}{#1}}
\newcommand{\HLJLsdC}[1]{\textcolor[RGB]{201,61,57}{#1}}
\newcommand{\HLJLse}[1]{\textcolor[RGB]{59,151,46}{#1}}
\newcommand{\HLJLsh}[1]{\textcolor[RGB]{201,61,57}{#1}}
\newcommand{\HLJLsi}[1]{#1}
\newcommand{\HLJLso}[1]{\textcolor[RGB]{201,61,57}{#1}}
\newcommand{\HLJLsr}[1]{\textcolor[RGB]{201,61,57}{#1}}
\newcommand{\HLJLss}[1]{\textcolor[RGB]{201,61,57}{#1}}
\newcommand{\HLJLssB}[1]{\textcolor[RGB]{201,61,57}{#1}}
\newcommand{\HLJLnB}[1]{\textcolor[RGB]{59,151,46}{#1}}
\newcommand{\HLJLnbB}[1]{\textcolor[RGB]{59,151,46}{#1}}
\newcommand{\HLJLnfB}[1]{\textcolor[RGB]{59,151,46}{#1}}
\newcommand{\HLJLnh}[1]{\textcolor[RGB]{59,151,46}{#1}}
\newcommand{\HLJLni}[1]{\textcolor[RGB]{59,151,46}{#1}}
\newcommand{\HLJLnil}[1]{\textcolor[RGB]{59,151,46}{#1}}
\newcommand{\HLJLnoB}[1]{\textcolor[RGB]{59,151,46}{#1}}
\newcommand{\HLJLoB}[1]{\textcolor[RGB]{102,102,102}{\textbf{#1}}}
\newcommand{\HLJLow}[1]{\textcolor[RGB]{102,102,102}{\textbf{#1}}}
\newcommand{\HLJLp}[1]{#1}
\newcommand{\HLJLc}[1]{\textcolor[RGB]{153,153,119}{\textit{#1}}}
\newcommand{\HLJLch}[1]{\textcolor[RGB]{153,153,119}{\textit{#1}}}
\newcommand{\HLJLcm}[1]{\textcolor[RGB]{153,153,119}{\textit{#1}}}
\newcommand{\HLJLcp}[1]{\textcolor[RGB]{153,153,119}{\textit{#1}}}
\newcommand{\HLJLcpB}[1]{\textcolor[RGB]{153,153,119}{\textit{#1}}}
\newcommand{\HLJLcs}[1]{\textcolor[RGB]{153,153,119}{\textit{#1}}}
\newcommand{\HLJLcsB}[1]{\textcolor[RGB]{153,153,119}{\textit{#1}}}
\newcommand{\HLJLg}[1]{#1}
\newcommand{\HLJLgd}[1]{#1}
\newcommand{\HLJLge}[1]{#1}
\newcommand{\HLJLgeB}[1]{#1}
\newcommand{\HLJLgh}[1]{#1}
\newcommand{\HLJLgi}[1]{#1}
\newcommand{\HLJLgo}[1]{#1}
\newcommand{\HLJLgp}[1]{#1}
\newcommand{\HLJLgs}[1]{#1}
\newcommand{\HLJLgsB}[1]{#1}
\newcommand{\HLJLgt}[1]{#1}



\def\qqand{\qquad\hbox{and}\qquad}
\def\qqfor{\qquad\hbox{for}\qquad}
\def\qqas{\qquad\hbox{as}\qquad}
\def\half{ {1 \over 2} }
\def\D{ {\rm d} }
\def\I{ {\rm i} }
\def\E{ {\rm e} }
\def\C{ {\mathbb C} }
\def\R{ {\mathbb R} }
\def\bbR{ {\mathbb R} }
\def\H{ {\mathbb H} }
\def\Z{ {\mathbb Z} }
\def\CC{ {\cal C} }
\def\FF{ {\cal F} }
\def\HH{ {\cal H} }
\def\LL{ {\cal L} }
\def\vc#1{ {\mathbf #1} }
\def\bbC{ {\mathbb C} }



\def\fR{ f_{\rm R} }
\def\fL{ f_{\rm L} }

\def\qqqquad{\qquad\qquad}
\def\qqwhere{\qquad\hbox{where}\qquad}
\def\Res_#1{\underset{#1}{\rm Res}\,}
\def\sech{ {\rm sech}\, }
\def\acos{ {\rm acos}\, }
\def\asin{ {\rm asin}\, }
\def\atan{ {\rm atan}\, }
\def\Ei{ {\rm Ei}\, }
\def\upepsilon{\varepsilon}


\def\Xint#1{ \mathchoice
   {\XXint\displaystyle\textstyle{#1} }%
   {\XXint\textstyle\scriptstyle{#1} }%
   {\XXint\scriptstyle\scriptscriptstyle{#1} }%
   {\XXint\scriptscriptstyle\scriptscriptstyle{#1} }%
   \!\int}
\def\XXint#1#2#3{ {\setbox0=\hbox{$#1{#2#3}{\int}$}
     \vcenter{\hbox{$#2#3$}}\kern-.5\wd0} }
\def\ddashint{\Xint=}
\def\dashint{\Xint-}
% \def\dashint
\def\infdashint{\dashint_{-\infty}^\infty}




\def\addtab#1={#1\;&=}
\def\ccr{\\\addtab}
\def\ip<#1>{\left\langle{#1}\right\rangle}
\def\dx{\D x}
\def\dt{\D t}
\def\dz{\D z}
\def\ds{\D s}

\def\rR{ {\rm R} }
\def\rL{ {\rm L} }

\def\norm#1{\left\| #1 \right\|}

\def\pr(#1){\left({#1}\right)}
\def\br[#1]{\left[{#1}\right]}

\def\abs#1{\left|{#1}\right|}
\def\fpr(#1){\!\pr({#1})}

\def\sopmatrix#1{ \begin{pmatrix}#1\end{pmatrix} }

\def\endash{–}
\def\emdash{—}
\def\mdblksquare{\blacksquare}
\def\lgblksquare{\blacksquare}
\def\scre{\E}
\def\mapengine#1,#2.{\mapfunction{#1}\ifx\void#2\else\mapengine #2.\fi }

\def\map[#1]{\mapengine #1,\void.}

\def\mapenginesep_#1#2,#3.{\mapfunction{#2}\ifx\void#3\else#1\mapengine #3.\fi }

\def\mapsep_#1[#2]{\mapenginesep_{#1}#2,\void.}


\def\vcbr[#1]{\pr(#1)}


\def\bvect[#1,#2]{
{
\def\dots{\cdots}
\def\mapfunction##1{\ | \  ##1}
	\sopmatrix{
		 \,#1\map[#2]\,
	}
}
}



\def\vect[#1]{
{\def\dots{\ldots}
	\vcbr[{#1}]
} }

\def\vectt[#1]{
{\def\dots{\ldots}
	\vect[{#1}]^{\top}
} }

\def\Vectt[#1]{
{
\def\mapfunction##1{##1 \cr}
\def\dots{\vdots}
	\begin{pmatrix}
		\map[#1]
	\end{pmatrix}
} }

\def\addtab#1={#1\;&=}
\def\ccr{\\\addtab}

\def\questionequals{= \!\!\!\!\!\!{\scriptstyle ? \atop }\,\,\,}

\def\Ei{\rm Ei\,}

\begin{document}

\section{Chapter 4: Solutions to Exercises, Part II}
Consider the Laguerre polynomials

\[
L_n^{(\alpha)}(x) = {(-1)^n \over n! } x^n + O(x^{n-1})
\]
where $\alpha > -1$, which are orthogonal with respect to

\[
\langle f,g\rangle_\alpha = \int_0^\infty f(x) g(x) x^\alpha {\rm e}^{-x}\,{\rm d}x.
\]
\begin{itemize}
\item[1. ] Show that the Rodrigues formula holds:

\end{itemize}
\[
L_n^{(\alpha)}(x) = {x^{-\alpha} {\rm e}^x \over n!} {{\rm d}^n \over \,{\rm d}x^n}\left[x^{\alpha+n}{\rm e}^{-x}\right].
\]
In other words, prove that $L_n^{(\alpha)}(x)$ (defined by the Rodrigues formula) is

(i) a polynomial of degree exactly $n$

(ii) orthogonal to all lower degree polynomials 

(iii) has leading coefficient $\frac{(-1)^n}{n!}$

Hints: For (i) and (iii), it may be helpful first to prove that

\[
{{\rm d} \over \,{\rm d}x}\left[x^{\alpha+1} {\rm e}^{-x} L_n^{(\alpha+1)}(x) \right]  = (n+1)x^{\alpha} {\rm e}^{-x}L_{n+1}^{(\alpha)}(x)
\]
For (ii), use integration by parts.

\textbf{Solution} Following the hint, we have that


\begin{eqnarray*}
{{\rm d} \over \,{\rm d}x}\left[x^{\alpha+1} {\rm e}^{-x} L_n^{(\alpha+1)}(x) \right]  &=& {{\rm d} \over \,{\rm d}x}\left[ {1 \over n!} {{\rm d}^n \over \,{\rm d}x^n}\left[x^{\alpha+1+n}{\rm e}^{-x}\right] \right] \\
&=& (n+1) {1 \over (n+1)!} {{\rm d}^{n+1} \over \,{\rm d}x^{n+1}}\left[x^{\alpha+n+1}{\rm e}^{-x}\right] \\
&=& (n+1)x^{\alpha} {\rm e}^{-x}L_{n+1}^{(\alpha)}(x),
\end{eqnarray*}
hence, differentiating, we find that

\[
\left[(\alpha + 1) - x   \right]x^{\alpha} {\rm e}^{-x} L_n^{(\alpha+1)}(x) + x^{\alpha+1} {\rm e}^{-x} \left(L_n^{(\alpha+1)}\right)'(x) = (n+1)x^{\alpha} {\rm e}^{-x}L_{n+1}^{(\alpha)}(x)
\]
or 

\[
(\alpha+1 -x)L_n^{(\alpha+1)}(x) + x(L_n^{(\alpha+1)})'(x) = (n+1) L_{n+1}^{(\alpha)}(x).
\]
By induction with the fact $L_0^{(\alpha)}(x) = 1$, we therefore get that

\[
L_n^{(\alpha)}(x) = {(\alpha+1 -x)L_{n-1}^{(\alpha+1)}(x) + x(L_{n-1}^{(\alpha+1)})'(x) \over n}
\]
is a degree $n$ polynomial, which proves (i). 

To prove (iii), we note that the leading coefficient (coefficient of the highest degree term) is


\begin{align*}
L_n^{(\alpha)}(x) &= -{x \over n} L_{n-1}^{(\alpha+1)}(x) +O(x^{n-1}) =   {x^2 \over n(n-1)} L_{n-2}^{(\alpha+2)}(x) +O(x^{n-1}) = \cdots 
= {(-1)^n x^n \over n!} L_0^{(\alpha+n)}(x) +O(x^{n-1}) \\
&= {(-1)^n x^n \over n!} +O(x^{n-1})
\end{align*}
We now prove (ii) using integration by parts: let $p_m(x)$ denote a polynomial of degree $m <n$, then integrating by parts $n$ times,


\begin{eqnarray*}
\langle  L_n^{(\alpha)}, p_m \rangle_{\alpha} &=& \int_0^\infty L_n^{(\alpha)}(x) p_m(x) x^\alpha {\rm e}^{-x} {\rm d} x \\
&=&
\int_0^\infty  {1 \over n!} {{\rm d}^n \over {\rm d} x^n}\left[x^{\alpha+n}{\rm e}^{-x}\right]  p_m(x)  {\rm d} x \\
&=&
-\int_0^\infty  {1 \over n!} {{\rm d}^{n-1} \over {\rm d} x^{n-1}}\left[x^{\alpha+n}{\rm e}^{-x}\right]  p_m'(x)  {\rm d} x \\
&\vdots & \\
&=& (-1)^n \int_0^\infty  {1 \over n!} \left[x^{\alpha+n}{\rm e}^{-x}\right]  p_m^{(n)}(x)  {\rm d} x = 0
\end{eqnarray*}
since $p_m^{(n)}(x) = 0$. Note we used the fact that

\[
{{\rm d}^k \over {\rm d} x^k}\left[x^{\alpha+n}{\rm e}^{-x}\right]
\]
vanishes at $x = 0$ and as $x \to \infty$ to ignore the boundary terms in integration by parts.

\begin{itemize}
\item[2. ] Show that

\end{itemize}
\[
(n+1) L_{n+1}^{(\alpha)}(x) = (\alpha+n+1)L_n^{(\alpha)}(x) -   xL_n^{(\alpha+1)}(x)
\]
\textbf{Solution} Applying the product rule once, we find that


\begin{align*}
(n+1) L_{n+1}^{(\alpha)}(x) &= {x^{-\alpha}{\rm e}^x \over n!} {{\rm d}^{n} \over {\rm d} x^{n}} {{\rm d} \over {\rm d} x} \left[x^{\alpha+n+1} {\rm e}^{-x}\right] \\
  &= {x^{-\alpha}{\rm e}^x \over n!} {{\rm d}^{n} \over {\rm d} x^{n}}  \left[(\alpha+n+1)x^{\alpha+n} {\rm e}^{-x}-x^{\alpha+n+1} {\rm e}^{-x}\right]  \\
    &= (\alpha+n+1)L_n^{(\alpha)}(x) -   xL_n^{(\alpha+1)}(x)
\end{align*}
\begin{itemize}
\item[3. ] Show that 

\end{itemize}
\[
L_{n}^{(\alpha+1)}(x) = L_{n-1}^{(\alpha+1)}(x) + L_n^{(\alpha)}(x).
\]
\textbf{Solution} Applying the product rule $n$ times, we find that


\begin{align*}
L_{n}^{(\alpha+1)}(x) &= {x^{-1-\alpha}{\rm e}^x \over n!} {{\rm d}^{n-1} \over {\rm d} x^{n-1}} {{\rm d} \over {\rm d} x} \left[x x^{\alpha+n} {\rm e}^{-x}\right] \\
&= {x^{-1-\alpha}{\rm e}^x \over n!} {{\rm d}^{n-1} \over {\rm d} x^{n-1}} \left[ x^{\alpha+n} {\rm e}^{-x}\right]  + {x^{-1-\alpha}{\rm e}^x \over n!} {{\rm d}^{n-1} \over {\rm d} x^{n-1}} x {{\rm d} \over {\rm d} x} \left[ x^{\alpha+n} {\rm e}^{-x}\right]  \\
&= {2 \over n}L_{n-1}^{(\alpha+1)}(x)   + {x^{-1-\alpha}{\rm e}^x \over n!} {{\rm d}^{n-2} \over {\rm d} x^{n-2}} x {{\rm d}^2 \over {\rm d} x^2} \left[ x^{\alpha+n} {\rm e}^{-x}\right]  \\
&= {3 \over n}L_{n-1}^{(\alpha+1)}(x)   + {x^{-1-\alpha}{\rm e}^x \over n!} {{\rm d}^{n-3} \over {\rm d} x^{n-3}} x {{\rm d}^3 \over {\rm d} x^3} \left[ x^{\alpha+n} {\rm e}^{-x}\right]  \\
&\vdots\\
&={n \over n}L_{n-1}^{(\alpha+1)}(x)   + {x^{-\alpha}{\rm e}^x \over n!} {{\rm d}^{n} \over {\rm d} x^{n}}  \left[ x^{\alpha+n} {\rm e}^{-x}\right] \\
&=L_{n-1}^{(\alpha+1)}(x) + L_n^{(\alpha)}(x)
\end{align*}
\begin{itemize}
\item[4. ] Show that the Laguerre polynomials satisfy the following three-term recurrence:

\end{itemize}
\[
x L_n^{(\alpha)}(x) = - (n+\alpha)L_{n-1}^{(\alpha)}(x) + (2n+\alpha+1) L_n^{(\alpha)}(x) -(n+1)L_{n+1}^{(\alpha)}(x)
\]
\textbf{Solution} Using the results in questions 2 and 3, we have that


\begin{align*}
x L_n^{(\alpha)}(x) &= -(n+1)L_{n+1}^{(\alpha-1)}(x) +(n+\alpha)L_n^{(\alpha-1)}(x) \\
  &= -(n+1)L_{n+1}^{(\alpha)}(x) + (n+1)L_{n}^{(\alpha)}(x) +(n+\alpha)L_{n}^{(\alpha)}(x) - (n+\alpha)L_{n-1}^{(\alpha)}(x) \\
  &= - (n+\alpha)L_{n-1}^{(\alpha)}(x) + (2n+\alpha+1) L_n^{(\alpha)}(x) -(n+1)L_{n+1}^{(\alpha)}(x)
\end{align*}
\begin{itemize}
\item[5. ] Prove that 

\end{itemize}
\[
{{\rm d} L_n^{(\alpha)} \over \,{\rm d}x}  = -L_{n-1}^{(\alpha+1)}(x)
\]
\textbf{Solution} First we show that ${{\rm d} L_n^{(\alpha)} \over \,{\rm d}x}$ is orthogonal wrt $\langle \cdot, \cdot \rangle_{\alpha + 1}$ to polynomials of degree $\leq n-2$ using integration by parts: let $p_m(x)$ be a polynomial of degree $m \leq n-2$, then


\begin{eqnarray*}
\langle \left(L_n^{(\alpha)}\right)', p_m  \rangle_{\alpha+1}  &=&\int_0^\infty {{\rm d} L_n^{(\alpha)}(x) \over {\rm d} x} p_m(x) x^{\alpha+1} {\rm e}^{-x} {\rm d} x  \\
&=& -\int_0^\infty L_n^{(\alpha)}(x) (x p_m'(x) + (\alpha+1) p_m(x) - x p_m(x)) x^{\alpha} {\rm e}^{-x} {\rm d} x  \\
&=& \langle L_n^{(\alpha)}, x p_m + (\alpha+1) p_m - x p_m \rangle_{\alpha} \\
&=& 0 
\end{eqnarray*}
since $(x p_m' + (\alpha+1) p_m - x p_m) $ has degree $m+1 < n$.  We conclude that ${{\rm d} L_n^{(\alpha)} \over \,{\rm d}x} = C L^{(\alpha+1)}_{n-1}$, for some constant $C$ which we can determine by matching leading terms: we have that

\[
{{\rm d} L_n^{(\alpha)} \over \,{\rm d}x} = \frac{(-1)^n}{(n-1)!}x^{n-1} + \mathcal{O}(x^{n-2}) = CL^{(\alpha+1)}_{n-1} = C \frac{(-1)^{n-1}}{(n-1)!}x^{n-1} + \mathcal{O}(x^{n-2}),
\]
which implies that $C = -1$.

\begin{itemize}
\item[6. ] Let 

\end{itemize}
\[
L^{(\alpha)}(x) = \left[L_0^{(\alpha)}(x) |  L_1^{(\alpha)}(x) | \cdots  \right].
\]
Give matrices $D_{\alpha}$ and $S_{\alpha}$ such that

\[
\frac{{\rm d}}{{\rm d}x}L^{(\alpha)}(x) = L^{(\alpha+1)}(x)\mathcal{D}_{\alpha} \quad \text{and} \quad L^{(\alpha)}(x) = L^{(\alpha+1)}(x)S_{\alpha}
\]
\textbf{Solution} It follows from the formulas in questions 5 and 3 that

\[
\mathcal{D}_{\alpha} = \begin{pmatrix}
0 & -1 \\
  &&-1 \\
  &&&\ddots
\end{pmatrix}
\]
and

\[
\mathcal{S}_{\alpha} = \begin{pmatrix}
        1 & -1 \\ & 1 & -1 \\&&\ddots & \ddots
\end{pmatrix}
\]
\begin{itemize}
\item[7. ] Consider the advection equation on the half line:

\end{itemize}
\[
u_t + u_x = 0, \qquad x \in [0, \infty), \qquad t \geq 0.
\]
Suppose the solution has an expansion of the form

\[
u(x,t) =  {\rm e}^{-x/2}\sum_{k=0}^{\infty}u_k(t) L_k^{(0)}(x) = {\rm e}^{-x/2}L^{(0)}(x)\mathbf{u}(t)
\]
where

\[
\mathbf{u}(t) = \begin{bmatrix}   
u_0(t) \\
u_1(t) \\
\vdots
\end{bmatrix}.
\]
Show that 

\[
\mathbf{u}'(t) = A\mathbf{u}(t),
\]
where $A$ is a matrix that is expressible in terms of $\mathcal{D}_0$ and $\mathcal{S}_0$ (defined in question 6).   Use software of your choice to build a $10 \times 10$ version of the matrix $A$.

\textbf{Solution} Since


\begin{eqnarray*}
\frac{\partial}{\partial x}u(x,t) &=& -\frac{1}{2}{\rm e}^{-x/2}L^{(0)}(x)\mathbf{u}(t)  + {\rm e}^{-x/2}\frac{{\rm d}}{{\rm d}x}L^{(0)}(x)\mathbf{u}(t) \\
&=& -\frac{1}{2}{\rm e}^{-x/2}L^{(1)}(x)\mathcal{S}_0\mathbf{u}(t)  + {\rm e}^{-x/2}L^{(1)}(x)\mathcal{D}_0\mathbf{u}(t) \\
&=& {\rm e}^{-x/2}L^{(1)}(x)\left[-\frac{1}{2}\mathcal{S}_0 + \mathcal{D}_0   \right]\mathbf{u}(t) \\
\end{eqnarray*}
and


\begin{eqnarray*}
\frac{\partial}{\partial t}u(x,t) &=& {\rm e}^{-x/2}L^{(0)}(x)\mathbf{u}'(t)  \\
&=& {\rm e}^{-x/2}L^{(1)}(x)\mathcal{S}_0\mathbf{u}'(t)
\end{eqnarray*}
the advection equation implies that

\[
u_t + u_x =
{\rm e}^{-x/2}L^{(1)}(x)\left[\mathcal{S}_0\mathbf{u}'(t)+  \left(-\frac{1}{2}\mathcal{S}_0 + \mathcal{D}_0\right)\mathbf{u}(t)  \right] = 0
\]
hence

\[
\mathcal{S}_0\mathbf{u}'(t) = \left(\frac{1}{2}\mathcal{S}_0 - \mathcal{D}_0\right)\mathbf{u}(t)
\]
or

\[
\mathbf{u}'(t) = A\mathbf{u}(t)
\]
where $A = \frac{1}{2}\mathcal{I} - \mathcal{S}_0^{-1}\mathcal{D}_0$ and $\mathcal{I}$ is an infinite identity matrix.


\begin{lstlisting}
(*@\HLJLk{using}@*) (*@\HLJLn{ApproxFun}@*)(*@\HLJLp{,}@*) (*@\HLJLn{LinearAlgebra}@*)
\end{lstlisting}


\begin{lstlisting}
(*@\HLJLn{D0}@*) (*@\HLJLoB{=}@*) (*@\HLJLnf{Derivative}@*)(*@\HLJLp{()}@*) (*@\HLJLoB{:}@*)  (*@\HLJLnf{Laguerre}@*)(*@\HLJLp{(}@*)(*@\HLJLni{0}@*)(*@\HLJLp{)}@*) (*@\HLJLoB{\ensuremath{\rightarrow}}@*) (*@\HLJLnf{Laguerre}@*)(*@\HLJLp{(}@*)(*@\HLJLni{1}@*)(*@\HLJLp{)}@*)
(*@\HLJLnf{real}@*)(*@\HLJLp{(}@*)(*@\HLJLn{D0}@*)(*@\HLJLp{)}@*)
\end{lstlisting}

\begin{lstlisting}
ReOperator : ApproxFunOrthogonalPolynomials.Laguerre(*@{{\{}}@*)Int64, ApproxFunOrthog
onalPolynomials.Ray(*@{{\{}}@*)false, Float64(*@{{\}}}@*)(*@{{\}}}@*)(0, 【0.0,(*@\ensuremath{\infty}@*)❫) (*@\ensuremath{\rightarrow}@*) ApproxFunOrthogonalPolyn
omials.Laguerre(*@{{\{}}@*)Int64, ApproxFunOrthogonalPolynomials.Ray(*@{{\{}}@*)false, Float64(*@{{\}}}@*)(*@{{\}}}@*)(
1, 【0.0,(*@\ensuremath{\infty}@*)❫)
 0.0  -1.0    (*@\ensuremath{\cdot}@*)     (*@\ensuremath{\cdot}@*)     (*@\ensuremath{\cdot}@*)     (*@\ensuremath{\cdot}@*)     (*@\ensuremath{\cdot}@*)     (*@\ensuremath{\cdot}@*)     (*@\ensuremath{\cdot}@*)     (*@\ensuremath{\cdot}@*)   (*@\ensuremath{\cdot}@*)
  (*@\ensuremath{\cdot}@*)    0.0  -1.0    (*@\ensuremath{\cdot}@*)     (*@\ensuremath{\cdot}@*)     (*@\ensuremath{\cdot}@*)     (*@\ensuremath{\cdot}@*)     (*@\ensuremath{\cdot}@*)     (*@\ensuremath{\cdot}@*)     (*@\ensuremath{\cdot}@*)   (*@\ensuremath{\cdot}@*)
  (*@\ensuremath{\cdot}@*)     (*@\ensuremath{\cdot}@*)    0.0  -1.0    (*@\ensuremath{\cdot}@*)     (*@\ensuremath{\cdot}@*)     (*@\ensuremath{\cdot}@*)     (*@\ensuremath{\cdot}@*)     (*@\ensuremath{\cdot}@*)     (*@\ensuremath{\cdot}@*)   (*@\ensuremath{\cdot}@*)
  (*@\ensuremath{\cdot}@*)     (*@\ensuremath{\cdot}@*)     (*@\ensuremath{\cdot}@*)    0.0  -1.0    (*@\ensuremath{\cdot}@*)     (*@\ensuremath{\cdot}@*)     (*@\ensuremath{\cdot}@*)     (*@\ensuremath{\cdot}@*)     (*@\ensuremath{\cdot}@*)   (*@\ensuremath{\cdot}@*)
  (*@\ensuremath{\cdot}@*)     (*@\ensuremath{\cdot}@*)     (*@\ensuremath{\cdot}@*)     (*@\ensuremath{\cdot}@*)    0.0  -1.0    (*@\ensuremath{\cdot}@*)     (*@\ensuremath{\cdot}@*)     (*@\ensuremath{\cdot}@*)     (*@\ensuremath{\cdot}@*)   (*@\ensuremath{\cdot}@*)
  (*@\ensuremath{\cdot}@*)     (*@\ensuremath{\cdot}@*)     (*@\ensuremath{\cdot}@*)     (*@\ensuremath{\cdot}@*)     (*@\ensuremath{\cdot}@*)    0.0  -1.0    (*@\ensuremath{\cdot}@*)     (*@\ensuremath{\cdot}@*)     (*@\ensuremath{\cdot}@*)   (*@\ensuremath{\cdot}@*)
  (*@\ensuremath{\cdot}@*)     (*@\ensuremath{\cdot}@*)     (*@\ensuremath{\cdot}@*)     (*@\ensuremath{\cdot}@*)     (*@\ensuremath{\cdot}@*)     (*@\ensuremath{\cdot}@*)    0.0  -1.0    (*@\ensuremath{\cdot}@*)     (*@\ensuremath{\cdot}@*)   (*@\ensuremath{\cdot}@*)
  (*@\ensuremath{\cdot}@*)     (*@\ensuremath{\cdot}@*)     (*@\ensuremath{\cdot}@*)     (*@\ensuremath{\cdot}@*)     (*@\ensuremath{\cdot}@*)     (*@\ensuremath{\cdot}@*)     (*@\ensuremath{\cdot}@*)    0.0  -1.0    (*@\ensuremath{\cdot}@*)   (*@\ensuremath{\cdot}@*)
  (*@\ensuremath{\cdot}@*)     (*@\ensuremath{\cdot}@*)     (*@\ensuremath{\cdot}@*)     (*@\ensuremath{\cdot}@*)     (*@\ensuremath{\cdot}@*)     (*@\ensuremath{\cdot}@*)     (*@\ensuremath{\cdot}@*)     (*@\ensuremath{\cdot}@*)    0.0  -1.0  (*@\ensuremath{\cdot}@*)
  (*@\ensuremath{\cdot}@*)     (*@\ensuremath{\cdot}@*)     (*@\ensuremath{\cdot}@*)     (*@\ensuremath{\cdot}@*)     (*@\ensuremath{\cdot}@*)     (*@\ensuremath{\cdot}@*)     (*@\ensuremath{\cdot}@*)     (*@\ensuremath{\cdot}@*)     (*@\ensuremath{\cdot}@*)    0.0  (*@\ensuremath{\ddots}@*)
  (*@\ensuremath{\cdot}@*)     (*@\ensuremath{\cdot}@*)     (*@\ensuremath{\cdot}@*)     (*@\ensuremath{\cdot}@*)     (*@\ensuremath{\cdot}@*)     (*@\ensuremath{\cdot}@*)     (*@\ensuremath{\cdot}@*)     (*@\ensuremath{\cdot}@*)     (*@\ensuremath{\cdot}@*)     (*@\ensuremath{\cdot}@*)   (*@\ensuremath{\ddots}@*)
\end{lstlisting}


\begin{lstlisting}
(*@\HLJLn{S0}@*) (*@\HLJLoB{=}@*) (*@\HLJLn{I}@*) (*@\HLJLoB{:}@*) (*@\HLJLnf{Laguerre}@*)(*@\HLJLp{(}@*)(*@\HLJLni{0}@*)(*@\HLJLp{)}@*) (*@\HLJLoB{\ensuremath{\rightarrow}}@*) (*@\HLJLnf{Laguerre}@*)(*@\HLJLp{(}@*)(*@\HLJLni{1}@*)(*@\HLJLp{)}@*)
\end{lstlisting}

\begin{lstlisting}
ConstantTimesOperator : ApproxFunOrthogonalPolynomials.Laguerre(*@{{\{}}@*)Int64, Appr
oxFunOrthogonalPolynomials.Ray(*@{{\{}}@*)false, Float64(*@{{\}}}@*)(*@{{\}}}@*)(0, 【0.0,(*@\ensuremath{\infty}@*)❫) (*@\ensuremath{\rightarrow}@*) ApproxFunOrth
ogonalPolynomials.Laguerre(*@{{\{}}@*)Int64, ApproxFunOrthogonalPolynomials.Ray(*@{{\{}}@*)false,
 Float64(*@{{\}}}@*)(*@{{\}}}@*)(1, 【0.0,(*@\ensuremath{\infty}@*)❫)
 1.0  -1.0    (*@\ensuremath{\cdot}@*)     (*@\ensuremath{\cdot}@*)     (*@\ensuremath{\cdot}@*)     (*@\ensuremath{\cdot}@*)     (*@\ensuremath{\cdot}@*)     (*@\ensuremath{\cdot}@*)     (*@\ensuremath{\cdot}@*)     (*@\ensuremath{\cdot}@*)   (*@\ensuremath{\cdot}@*)
  (*@\ensuremath{\cdot}@*)    1.0  -1.0    (*@\ensuremath{\cdot}@*)     (*@\ensuremath{\cdot}@*)     (*@\ensuremath{\cdot}@*)     (*@\ensuremath{\cdot}@*)     (*@\ensuremath{\cdot}@*)     (*@\ensuremath{\cdot}@*)     (*@\ensuremath{\cdot}@*)   (*@\ensuremath{\cdot}@*)
  (*@\ensuremath{\cdot}@*)     (*@\ensuremath{\cdot}@*)    1.0  -1.0    (*@\ensuremath{\cdot}@*)     (*@\ensuremath{\cdot}@*)     (*@\ensuremath{\cdot}@*)     (*@\ensuremath{\cdot}@*)     (*@\ensuremath{\cdot}@*)     (*@\ensuremath{\cdot}@*)   (*@\ensuremath{\cdot}@*)
  (*@\ensuremath{\cdot}@*)     (*@\ensuremath{\cdot}@*)     (*@\ensuremath{\cdot}@*)    1.0  -1.0    (*@\ensuremath{\cdot}@*)     (*@\ensuremath{\cdot}@*)     (*@\ensuremath{\cdot}@*)     (*@\ensuremath{\cdot}@*)     (*@\ensuremath{\cdot}@*)   (*@\ensuremath{\cdot}@*)
  (*@\ensuremath{\cdot}@*)     (*@\ensuremath{\cdot}@*)     (*@\ensuremath{\cdot}@*)     (*@\ensuremath{\cdot}@*)    1.0  -1.0    (*@\ensuremath{\cdot}@*)     (*@\ensuremath{\cdot}@*)     (*@\ensuremath{\cdot}@*)     (*@\ensuremath{\cdot}@*)   (*@\ensuremath{\cdot}@*)
  (*@\ensuremath{\cdot}@*)     (*@\ensuremath{\cdot}@*)     (*@\ensuremath{\cdot}@*)     (*@\ensuremath{\cdot}@*)     (*@\ensuremath{\cdot}@*)    1.0  -1.0    (*@\ensuremath{\cdot}@*)     (*@\ensuremath{\cdot}@*)     (*@\ensuremath{\cdot}@*)   (*@\ensuremath{\cdot}@*)
  (*@\ensuremath{\cdot}@*)     (*@\ensuremath{\cdot}@*)     (*@\ensuremath{\cdot}@*)     (*@\ensuremath{\cdot}@*)     (*@\ensuremath{\cdot}@*)     (*@\ensuremath{\cdot}@*)    1.0  -1.0    (*@\ensuremath{\cdot}@*)     (*@\ensuremath{\cdot}@*)   (*@\ensuremath{\cdot}@*)
  (*@\ensuremath{\cdot}@*)     (*@\ensuremath{\cdot}@*)     (*@\ensuremath{\cdot}@*)     (*@\ensuremath{\cdot}@*)     (*@\ensuremath{\cdot}@*)     (*@\ensuremath{\cdot}@*)     (*@\ensuremath{\cdot}@*)    1.0  -1.0    (*@\ensuremath{\cdot}@*)   (*@\ensuremath{\cdot}@*)
  (*@\ensuremath{\cdot}@*)     (*@\ensuremath{\cdot}@*)     (*@\ensuremath{\cdot}@*)     (*@\ensuremath{\cdot}@*)     (*@\ensuremath{\cdot}@*)     (*@\ensuremath{\cdot}@*)     (*@\ensuremath{\cdot}@*)     (*@\ensuremath{\cdot}@*)    1.0  -1.0  (*@\ensuremath{\cdot}@*)
  (*@\ensuremath{\cdot}@*)     (*@\ensuremath{\cdot}@*)     (*@\ensuremath{\cdot}@*)     (*@\ensuremath{\cdot}@*)     (*@\ensuremath{\cdot}@*)     (*@\ensuremath{\cdot}@*)     (*@\ensuremath{\cdot}@*)     (*@\ensuremath{\cdot}@*)     (*@\ensuremath{\cdot}@*)    1.0  (*@\ensuremath{\ddots}@*)
  (*@\ensuremath{\cdot}@*)     (*@\ensuremath{\cdot}@*)     (*@\ensuremath{\cdot}@*)     (*@\ensuremath{\cdot}@*)     (*@\ensuremath{\cdot}@*)     (*@\ensuremath{\cdot}@*)     (*@\ensuremath{\cdot}@*)     (*@\ensuremath{\cdot}@*)     (*@\ensuremath{\cdot}@*)     (*@\ensuremath{\cdot}@*)   (*@\ensuremath{\ddots}@*)
\end{lstlisting}


\begin{lstlisting}
(*@\HLJLn{n}@*) (*@\HLJLoB{=}@*) (*@\HLJLni{10}@*)
(*@\HLJLn{A}@*) (*@\HLJLoB{=}@*) (*@\HLJLn{I}@*)(*@\HLJLoB{/}@*)(*@\HLJLni{2}@*) (*@\HLJLoB{-}@*) (*@\HLJLn{S0}@*)(*@\HLJLp{[}@*)(*@\HLJLni{1}@*)(*@\HLJLoB{:}@*)(*@\HLJLn{n}@*)(*@\HLJLp{,}@*)(*@\HLJLni{1}@*)(*@\HLJLoB{:}@*)(*@\HLJLn{n}@*)(*@\HLJLp{]}@*)(*@\HLJLoB{{\textbackslash}}@*)(*@\HLJLnf{real}@*)(*@\HLJLp{(}@*)(*@\HLJLn{D0}@*)(*@\HLJLp{[}@*)(*@\HLJLni{1}@*)(*@\HLJLoB{:}@*)(*@\HLJLn{n}@*)(*@\HLJLp{,}@*)(*@\HLJLni{1}@*)(*@\HLJLoB{:}@*)(*@\HLJLn{n}@*)(*@\HLJLp{])}@*)
\end{lstlisting}

\begin{lstlisting}
10(*@\ensuremath{\times}@*)10 Matrix(*@{{\{}}@*)Float64(*@{{\}}}@*):
  0.5   1.0   1.0   1.0   1.0   1.0   1.0   1.0   1.0  1.0
 -0.0   0.5   1.0   1.0   1.0   1.0   1.0   1.0   1.0  1.0
 -0.0  -0.0   0.5   1.0   1.0   1.0   1.0   1.0   1.0  1.0
 -0.0  -0.0  -0.0   0.5   1.0   1.0   1.0   1.0   1.0  1.0
 -0.0  -0.0  -0.0  -0.0   0.5   1.0   1.0   1.0   1.0  1.0
 -0.0  -0.0  -0.0  -0.0  -0.0   0.5   1.0   1.0   1.0  1.0
 -0.0  -0.0  -0.0  -0.0  -0.0  -0.0   0.5   1.0   1.0  1.0
 -0.0  -0.0  -0.0  -0.0  -0.0  -0.0  -0.0   0.5   1.0  1.0
 -0.0  -0.0  -0.0  -0.0  -0.0  -0.0  -0.0  -0.0   0.5  1.0
 -0.0  -0.0  -0.0  -0.0  -0.0  -0.0  -0.0  -0.0  -0.0  0.5
\end{lstlisting}



\end{document}
