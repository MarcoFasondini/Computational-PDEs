\documentclass[12pt,a4paper]{article}

\usepackage[a4paper,text={16.5cm,25.2cm},centering]{geometry}
\usepackage{lmodern}
\usepackage{amssymb,amsmath}
\usepackage{bm}
\usepackage{graphicx}
\usepackage{microtype}
\usepackage{hyperref}
\setlength{\parindent}{0pt}
\setlength{\parskip}{1.2ex}

\hypersetup
       {   pdfauthor = { Marco Fasondini },
           pdftitle={ foo },
           colorlinks=TRUE,
           linkcolor=black,
           citecolor=blue,
           urlcolor=blue
       }




\usepackage{upquote}
\usepackage{listings}
\usepackage{xcolor}
\lstset{
    basicstyle=\ttfamily\footnotesize,
    upquote=true,
    breaklines=true,
    breakindent=0pt,
    keepspaces=true,
    showspaces=false,
    columns=fullflexible,
    showtabs=false,
    showstringspaces=false,
    escapeinside={(*@}{@*)},
    extendedchars=true,
}
\newcommand{\HLJLt}[1]{#1}
\newcommand{\HLJLw}[1]{#1}
\newcommand{\HLJLe}[1]{#1}
\newcommand{\HLJLeB}[1]{#1}
\newcommand{\HLJLo}[1]{#1}
\newcommand{\HLJLk}[1]{\textcolor[RGB]{148,91,176}{\textbf{#1}}}
\newcommand{\HLJLkc}[1]{\textcolor[RGB]{59,151,46}{\textit{#1}}}
\newcommand{\HLJLkd}[1]{\textcolor[RGB]{214,102,97}{\textit{#1}}}
\newcommand{\HLJLkn}[1]{\textcolor[RGB]{148,91,176}{\textbf{#1}}}
\newcommand{\HLJLkp}[1]{\textcolor[RGB]{148,91,176}{\textbf{#1}}}
\newcommand{\HLJLkr}[1]{\textcolor[RGB]{148,91,176}{\textbf{#1}}}
\newcommand{\HLJLkt}[1]{\textcolor[RGB]{148,91,176}{\textbf{#1}}}
\newcommand{\HLJLn}[1]{#1}
\newcommand{\HLJLna}[1]{#1}
\newcommand{\HLJLnb}[1]{#1}
\newcommand{\HLJLnbp}[1]{#1}
\newcommand{\HLJLnc}[1]{#1}
\newcommand{\HLJLncB}[1]{#1}
\newcommand{\HLJLnd}[1]{\textcolor[RGB]{214,102,97}{#1}}
\newcommand{\HLJLne}[1]{#1}
\newcommand{\HLJLneB}[1]{#1}
\newcommand{\HLJLnf}[1]{\textcolor[RGB]{66,102,213}{#1}}
\newcommand{\HLJLnfm}[1]{\textcolor[RGB]{66,102,213}{#1}}
\newcommand{\HLJLnp}[1]{#1}
\newcommand{\HLJLnl}[1]{#1}
\newcommand{\HLJLnn}[1]{#1}
\newcommand{\HLJLno}[1]{#1}
\newcommand{\HLJLnt}[1]{#1}
\newcommand{\HLJLnv}[1]{#1}
\newcommand{\HLJLnvc}[1]{#1}
\newcommand{\HLJLnvg}[1]{#1}
\newcommand{\HLJLnvi}[1]{#1}
\newcommand{\HLJLnvm}[1]{#1}
\newcommand{\HLJLl}[1]{#1}
\newcommand{\HLJLld}[1]{\textcolor[RGB]{148,91,176}{\textit{#1}}}
\newcommand{\HLJLs}[1]{\textcolor[RGB]{201,61,57}{#1}}
\newcommand{\HLJLsa}[1]{\textcolor[RGB]{201,61,57}{#1}}
\newcommand{\HLJLsb}[1]{\textcolor[RGB]{201,61,57}{#1}}
\newcommand{\HLJLsc}[1]{\textcolor[RGB]{201,61,57}{#1}}
\newcommand{\HLJLsd}[1]{\textcolor[RGB]{201,61,57}{#1}}
\newcommand{\HLJLsdB}[1]{\textcolor[RGB]{201,61,57}{#1}}
\newcommand{\HLJLsdC}[1]{\textcolor[RGB]{201,61,57}{#1}}
\newcommand{\HLJLse}[1]{\textcolor[RGB]{59,151,46}{#1}}
\newcommand{\HLJLsh}[1]{\textcolor[RGB]{201,61,57}{#1}}
\newcommand{\HLJLsi}[1]{#1}
\newcommand{\HLJLso}[1]{\textcolor[RGB]{201,61,57}{#1}}
\newcommand{\HLJLsr}[1]{\textcolor[RGB]{201,61,57}{#1}}
\newcommand{\HLJLss}[1]{\textcolor[RGB]{201,61,57}{#1}}
\newcommand{\HLJLssB}[1]{\textcolor[RGB]{201,61,57}{#1}}
\newcommand{\HLJLnB}[1]{\textcolor[RGB]{59,151,46}{#1}}
\newcommand{\HLJLnbB}[1]{\textcolor[RGB]{59,151,46}{#1}}
\newcommand{\HLJLnfB}[1]{\textcolor[RGB]{59,151,46}{#1}}
\newcommand{\HLJLnh}[1]{\textcolor[RGB]{59,151,46}{#1}}
\newcommand{\HLJLni}[1]{\textcolor[RGB]{59,151,46}{#1}}
\newcommand{\HLJLnil}[1]{\textcolor[RGB]{59,151,46}{#1}}
\newcommand{\HLJLnoB}[1]{\textcolor[RGB]{59,151,46}{#1}}
\newcommand{\HLJLoB}[1]{\textcolor[RGB]{102,102,102}{\textbf{#1}}}
\newcommand{\HLJLow}[1]{\textcolor[RGB]{102,102,102}{\textbf{#1}}}
\newcommand{\HLJLp}[1]{#1}
\newcommand{\HLJLc}[1]{\textcolor[RGB]{153,153,119}{\textit{#1}}}
\newcommand{\HLJLch}[1]{\textcolor[RGB]{153,153,119}{\textit{#1}}}
\newcommand{\HLJLcm}[1]{\textcolor[RGB]{153,153,119}{\textit{#1}}}
\newcommand{\HLJLcp}[1]{\textcolor[RGB]{153,153,119}{\textit{#1}}}
\newcommand{\HLJLcpB}[1]{\textcolor[RGB]{153,153,119}{\textit{#1}}}
\newcommand{\HLJLcs}[1]{\textcolor[RGB]{153,153,119}{\textit{#1}}}
\newcommand{\HLJLcsB}[1]{\textcolor[RGB]{153,153,119}{\textit{#1}}}
\newcommand{\HLJLg}[1]{#1}
\newcommand{\HLJLgd}[1]{#1}
\newcommand{\HLJLge}[1]{#1}
\newcommand{\HLJLgeB}[1]{#1}
\newcommand{\HLJLgh}[1]{#1}
\newcommand{\HLJLgi}[1]{#1}
\newcommand{\HLJLgo}[1]{#1}
\newcommand{\HLJLgp}[1]{#1}
\newcommand{\HLJLgs}[1]{#1}
\newcommand{\HLJLgsB}[1]{#1}
\newcommand{\HLJLgt}[1]{#1}



\def\qqand{\qquad\hbox{and}\qquad}
\def\qqfor{\qquad\hbox{for}\qquad}
\def\qqas{\qquad\hbox{as}\qquad}
\def\half{ {1 \over 2} }
\def\D{ {\rm d} }
\def\I{ {\rm i} }
\def\E{ {\rm e} }
\def\C{ {\mathbb C} }
\def\R{ {\mathbb R} }
\def\bbR{ {\mathbb R} }
\def\H{ {\mathbb H} }
\def\Z{ {\mathbb Z} }
\def\CC{ {\cal C} }
\def\FF{ {\cal F} }
\def\HH{ {\cal H} }
\def\LL{ {\cal L} }
\def\vc#1{ {\mathbf #1} }
\def\bbC{ {\mathbb C} }



\def\fR{ f_{\rm R} }
\def\fL{ f_{\rm L} }

\def\qqqquad{\qquad\qquad}
\def\qqwhere{\qquad\hbox{where}\qquad}
\def\Res_#1{\underset{#1}{\rm Res}\,}
\def\sech{ {\rm sech}\, }
\def\acos{ {\rm acos}\, }
\def\asin{ {\rm asin}\, }
\def\atan{ {\rm atan}\, }
\def\Ei{ {\rm Ei}\, }
\def\upepsilon{\varepsilon}


\def\Xint#1{ \mathchoice
   {\XXint\displaystyle\textstyle{#1} }%
   {\XXint\textstyle\scriptstyle{#1} }%
   {\XXint\scriptstyle\scriptscriptstyle{#1} }%
   {\XXint\scriptscriptstyle\scriptscriptstyle{#1} }%
   \!\int}
\def\XXint#1#2#3{ {\setbox0=\hbox{$#1{#2#3}{\int}$}
     \vcenter{\hbox{$#2#3$}}\kern-.5\wd0} }
\def\ddashint{\Xint=}
\def\dashint{\Xint-}
% \def\dashint
\def\infdashint{\dashint_{-\infty}^\infty}




\def\addtab#1={#1\;&=}
\def\ccr{\\\addtab}
\def\ip<#1>{\left\langle{#1}\right\rangle}
\def\dx{\D x}
\def\dt{\D t}
\def\dz{\D z}
\def\ds{\D s}

\def\rR{ {\rm R} }
\def\rL{ {\rm L} }

\def\norm#1{\left\| #1 \right\|}

\def\pr(#1){\left({#1}\right)}
\def\br[#1]{\left[{#1}\right]}

\def\abs#1{\left|{#1}\right|}
\def\fpr(#1){\!\pr({#1})}

\def\sopmatrix#1{ \begin{pmatrix}#1\end{pmatrix} }

\def\endash{–}
\def\emdash{—}
\def\mdblksquare{\blacksquare}
\def\lgblksquare{\blacksquare}
\def\scre{\E}
\def\mapengine#1,#2.{\mapfunction{#1}\ifx\void#2\else\mapengine #2.\fi }

\def\map[#1]{\mapengine #1,\void.}

\def\mapenginesep_#1#2,#3.{\mapfunction{#2}\ifx\void#3\else#1\mapengine #3.\fi }

\def\mapsep_#1[#2]{\mapenginesep_{#1}#2,\void.}


\def\vcbr[#1]{\pr(#1)}


\def\bvect[#1,#2]{
{
\def\dots{\cdots}
\def\mapfunction##1{\ | \  ##1}
	\sopmatrix{
		 \,#1\map[#2]\,
	}
}
}



\def\vect[#1]{
{\def\dots{\ldots}
	\vcbr[{#1}]
} }

\def\vectt[#1]{
{\def\dots{\ldots}
	\vect[{#1}]^{\top}
} }

\def\Vectt[#1]{
{
\def\mapfunction##1{##1 \cr}
\def\dots{\vdots}
	\begin{pmatrix}
		\map[#1]
	\end{pmatrix}
} }

\def\addtab#1={#1\;&=}
\def\ccr{\\\addtab}

\def\questionequals{= \!\!\!\!\!\!{\scriptstyle ? \atop }\,\,\,}

\def\Ei{\rm Ei\,}

\begin{document}

\section{Chapter 1: Solutions to Exercises}
\begin{itemize}
\item[1. ] \textbf{(Second-order central difference approximation to the second derivative)} Show that if $f \in C^{4}[x_{j-1}, x_{j+1}]$, then

\end{itemize}
\[
f''(x_j) = \frac{f(x_{j+1}) -  2f(x_{j}) + f(x_{j-1})}{h^2} + \mathcal{O}(h^2), \qquad h \to 0.
\]
It follows from Taylor's theorem that there exists an $\xi_1 \in [x_{j},x_{j+1}]$ such that

\[
f(x_{j+1}) = f(x_{j}) + hf'(x_j) + \frac{h^2}{2}f''(x_j) +  \frac{h^3}{6}f'''(x_j)  +  \frac{h^4}{24}f^{(4)}(\xi_1)
\]
where we used the fact that  $x_{j+1} - x_j = h$. Similarly, there exists an $\xi_2 \in [x_{j-1},x_{j}]$ such that

\[
f(x_{j-1}) = f(x_{j}) - hf'(x_j) + \frac{h^2}{2}f''(x_j) -  \frac{h^3}{6}f'''(x_j)  +  \frac{h^4}{24}f^{(4)}(\xi_2)
\]
Adding these two equations and solving for $f''(x_j)$, we have that

\[
f''(x_j) = \frac{f(x_{j+1}) -  2f(x_{j}) + f(x_{j-1})}{h^2} + \frac{h^2}{24}\left(f^{(4)}(\xi_1) + f^{(4)}(\xi_2)\right)
\]
Since $f \in C^{4}[x_{j-1}, x_{j+1}]$, there exists an $M < \infty$ such that 

\[
M = \sup_{x \in [x_{j-1}, x_{j+1}]} \vert f^{(4)}(x)\vert
\]
and therefore

\[
\left\vert f''(x_j) - \frac{f(x_{j+1}) -  2f(x_{j}) + f(x_{j-1})}{h^2}\right\vert  \leq \frac{ M}{12}h^2
\]
and hence the result follows.

\begin{itemize}
\item[2. ] Run the Julia code and / or the Matlab code for this chapter (see the Blackboard page of this module) on your own machine.  To run the Julia code, see the document titled Julia on Blackboard.  To run the Matlab code, you'll need to use Chebfun (see the instructions for how to do this \href{https://www.chebfun.org/download/}{here}). 


\item[3. ] Construct, by hand, a polynomial that interpolates $f(x) = {\rm e}^{x}$ at $x = 0, 1, 2$.

\end{itemize}
Using the formula for the Lagrange interpolating polynomial, we have that

\[
p_3(x) = \frac{(x-1)(x-2)}{2}{\rm e}^0 - (x-0)(x-2){\rm e}^1 + \frac{(x-0)(x-1)}{2}{\rm e}^2
\]
\begin{itemize}
\item[4. ] \textbf{(Second-order backward difference approximation)} Construct a Lagrange interpolating polynomial $p_3(x)$ that interpolates $f(x)$ at $x_{j-2}$, $x_{j-1}$ and $x_{j}$, where $x_{j}-x_{j-1} = x_{j-1}-x_{j-2} = h$, and derive the following one-sided finite difference approximation:  

\end{itemize}
\[
f'(x_j) \approx p_3'(x_j) = \frac{3f(x_j) - 4f(x_{j-1}) + f(x_{j-2})}{2h}
\]
Using the formula for the Lagrange interpolating polynomial, it follows that

\[
p_3(x) = \frac{(x-x_{j-1})(x-x_{j})}{2h^2} f(x_{j-2}) - \frac{(x-x_{j-2})(x-x_{j})}{h^2} f(x_{j-1}) +  \frac{(x-x_{j-2})(x-x_{j-1})}{2h^2} f(x_{j}).
\]
Since

\[
p_3'(x) = \frac{(x-x_{j-1}) + (x-x_{j})}{2h^2} f(x_{j-2}) - \frac{(x-x_{j-2})+ (x-x_{j})}{h^2} f(x_{j-1}) +  \frac{(x-x_{j-2}) + (x-x_{j-1})}{2h^2} f(x_{j})
\]
we have that

\[
p_3'(x_j) = \frac{h}{2h^2} f(x_{j-2}) - \frac{2h}{h^2} f(x_{j-1}) +  \frac{2h + h}{2h^2} f(x_{j})
\]
and the result follows.

\begin{itemize}
\item[5. ] In Julia or Matlab, compute the maximum error of the finite difference approximation 

\end{itemize}
\[
f'(x_j) \approx \frac{f(x_{j-2}) -8 f(x_{j-1}) + 8 f(x_{j+1}) - f(x_{j+2})}{12h},
\]
at the points $x_j = jh$ with $j = 0, \ldots, n-1$ and $h = 2\pi/n$ for the function $f(x) = 1/(2 + \cos(x))$ for various values of $n$.  Then plot the errors on a log-log plot. Comment on your results.

Since $f(x)$ is periodic, the differentiation matrix is as follows:

\[
\left(
\begin{array}{c}
f'(x_0) \\
f'(x_1) \\
  \\
\vdots  \\
  \\
f'(x_{n-2}) \\ 
f'(x_{n-1})
\end{array}
\right)
\approx 
\left(
\begin{array}{c}
p'(x_0) \\
p'(x_1) \\
  \\
\vdots  \\
  \\
p'(x_{n-2}) \\
p'(x_{n-1})
\end{array}
\right) = \frac{1}{h}\begin{bmatrix}
 0 & \frac{2}{3} & -\frac{1}{12} &   &  &  \frac{1}{12} & -\frac{2}{3}    \\
-\frac{2}{3} & 0 &  \frac{2}{3} & \ddots   & & & \frac{1}{12}   \\
\frac{1}{12} & -\frac{2}{3} & 0 & \ddots  & \ddots  & &  \\
& \ddots    & \ddots   & \ddots & \ddots &\ddots  & \\
&     & \frac{1}{12} &-\frac{2}{3} & 0 & \frac{2}{3} & -\frac{1}{12} \\
 -\frac{1}{12} &   &  & \frac{1}{12} &-\frac{2}{3} & 0 & \frac{2}{3} \\
 \frac{2}{3}&  -\frac{1}{12} &   &   & \frac{1}{12} &-\frac{2}{3} & 0 
\end{bmatrix}
\left(
\begin{array}{c}
f(x_0) \\
f(x_1) \\
  \\
\vdots  \\
  \\
f(x_{n-2}) \\
f(x_{n-1})
\end{array}
\right).
\]
Here is how we can build the differentiation matrix:


\begin{lstlisting}
(*@\HLJLk{using}@*) (*@\HLJLn{LinearAlgebra}@*)(*@\HLJLp{,}@*) (*@\HLJLn{Plots}@*)(*@\HLJLp{,}@*) (*@\HLJLn{SparseArrays}@*)
(*@\HLJLn{n}@*) (*@\HLJLoB{=}@*) (*@\HLJLni{10}@*)
(*@\HLJLn{Dn}@*) (*@\HLJLoB{=}@*) (*@\HLJLnf{spdiagm}@*)(*@\HLJLp{(}@*)(*@\HLJLni{1}@*)(*@\HLJLoB{=>}@*)(*@\HLJLnf{fill}@*)(*@\HLJLp{(}@*)(*@\HLJLni{2}@*)(*@\HLJLoB{/}@*)(*@\HLJLni{3}@*)(*@\HLJLp{,}@*)(*@\HLJLn{n}@*)(*@\HLJLoB{-}@*)(*@\HLJLni{1}@*)(*@\HLJLp{),}@*)(*@\HLJLni{2}@*)(*@\HLJLoB{=>}@*)(*@\HLJLnf{fill}@*)(*@\HLJLp{(}@*)(*@\HLJLoB{-}@*)(*@\HLJLni{1}@*)(*@\HLJLoB{/}@*)(*@\HLJLni{12}@*)(*@\HLJLp{,}@*)(*@\HLJLn{n}@*)(*@\HLJLoB{-}@*)(*@\HLJLni{2}@*)(*@\HLJLp{),}@*)(*@\HLJLn{n}@*)(*@\HLJLoB{-}@*)(*@\HLJLni{2}@*)(*@\HLJLoB{=>}@*)(*@\HLJLnf{fill}@*)(*@\HLJLp{(}@*)(*@\HLJLni{1}@*)(*@\HLJLoB{/}@*)(*@\HLJLni{12}@*)(*@\HLJLp{,}@*)(*@\HLJLni{2}@*)(*@\HLJLp{),}@*)(*@\HLJLn{n}@*)(*@\HLJLoB{-}@*)(*@\HLJLni{1}@*)(*@\HLJLoB{=>}@*)(*@\HLJLnf{fill}@*)(*@\HLJLp{(}@*)(*@\HLJLoB{-}@*)(*@\HLJLni{2}@*)(*@\HLJLoB{/}@*)(*@\HLJLni{3}@*)(*@\HLJLp{,}@*)(*@\HLJLni{1}@*)(*@\HLJLp{))}@*)
(*@\HLJLn{Dn}@*) (*@\HLJLoB{=}@*) (*@\HLJLn{Dn}@*) (*@\HLJLoB{-}@*) (*@\HLJLn{Dn}@*)(*@\HLJLoB{{\textquotesingle}}@*)
\end{lstlisting}

\begin{lstlisting}
10(*@\ensuremath{\times}@*)10 SparseArrays.SparseMatrixCSC(*@{{\{}}@*)Float64, Int64(*@{{\}}}@*) with 40 stored entries:
   (*@\ensuremath{\cdot}@*)          0.666667   -0.0833333  (*@\ensuremath{\ldots}@*)    (*@\ensuremath{\cdot}@*)          0.0833333  -0.666667
 -0.666667     (*@\ensuremath{\cdot}@*)          0.666667        (*@\ensuremath{\cdot}@*)           (*@\ensuremath{\cdot}@*)          0.0833333
  0.0833333  -0.666667     (*@\ensuremath{\cdot}@*)              (*@\ensuremath{\cdot}@*)           (*@\ensuremath{\cdot}@*)           (*@\ensuremath{\cdot}@*) 
   (*@\ensuremath{\cdot}@*)          0.0833333  -0.666667        (*@\ensuremath{\cdot}@*)           (*@\ensuremath{\cdot}@*)           (*@\ensuremath{\cdot}@*) 
   (*@\ensuremath{\cdot}@*)           (*@\ensuremath{\cdot}@*)          0.0833333       (*@\ensuremath{\cdot}@*)           (*@\ensuremath{\cdot}@*)           (*@\ensuremath{\cdot}@*) 
   (*@\ensuremath{\cdot}@*)           (*@\ensuremath{\cdot}@*)           (*@\ensuremath{\cdot}@*)         (*@\ensuremath{\ldots}@*)  -0.0833333    (*@\ensuremath{\cdot}@*)           (*@\ensuremath{\cdot}@*) 
   (*@\ensuremath{\cdot}@*)           (*@\ensuremath{\cdot}@*)           (*@\ensuremath{\cdot}@*)             0.666667   -0.0833333    (*@\ensuremath{\cdot}@*) 
   (*@\ensuremath{\cdot}@*)           (*@\ensuremath{\cdot}@*)           (*@\ensuremath{\cdot}@*)              (*@\ensuremath{\cdot}@*)          0.666667   -0.0833333
 -0.0833333    (*@\ensuremath{\cdot}@*)           (*@\ensuremath{\cdot}@*)            -0.666667     (*@\ensuremath{\cdot}@*)          0.666667
  0.666667   -0.0833333    (*@\ensuremath{\cdot}@*)             0.0833333  -0.666667     (*@\ensuremath{\cdot}@*)
\end{lstlisting}


\begin{lstlisting}
(*@\HLJLn{f}@*) (*@\HLJLoB{=}@*) (*@\HLJLn{x}@*) (*@\HLJLoB{->}@*) (*@\HLJLni{1}@*)(*@\HLJLoB{/}@*)(*@\HLJLp{(}@*)(*@\HLJLni{2}@*) (*@\HLJLoB{+}@*) (*@\HLJLnf{cos}@*)(*@\HLJLp{(}@*)(*@\HLJLn{x}@*)(*@\HLJLp{))}@*)
(*@\HLJLn{Df}@*) (*@\HLJLoB{=}@*) (*@\HLJLn{x}@*) (*@\HLJLoB{->}@*) (*@\HLJLnf{sin}@*)(*@\HLJLp{(}@*)(*@\HLJLn{x}@*)(*@\HLJLp{)}@*)(*@\HLJLoB{/}@*)(*@\HLJLp{(}@*)(*@\HLJLni{2}@*) (*@\HLJLoB{+}@*) (*@\HLJLnf{cos}@*)(*@\HLJLp{(}@*)(*@\HLJLn{x}@*)(*@\HLJLp{))}@*)(*@\HLJLoB{{\textasciicircum}}@*)(*@\HLJLni{2}@*)  (*@\HLJLcs{{\#}}@*) (*@\HLJLcs{derivative}@*)
(*@\HLJLcs{{\#}}@*) (*@\HLJLcs{compute}@*) (*@\HLJLcs{the}@*) (*@\HLJLcs{errors}@*)
(*@\HLJLn{nv}@*) (*@\HLJLoB{=}@*) (*@\HLJLn{Int64}@*)(*@\HLJLoB{.}@*)(*@\HLJLp{(}@*)(*@\HLJLn{round}@*)(*@\HLJLoB{.}@*)(*@\HLJLp{(}@*)(*@\HLJLni{10}@*) (*@\HLJLoB{.{\textasciicircum}}@*)(*@\HLJLp{(}@*)(*@\HLJLni{1}@*)(*@\HLJLoB{:}@*)(*@\HLJLnfB{0.5}@*)(*@\HLJLoB{:}@*)(*@\HLJLni{6}@*)(*@\HLJLp{)))}@*)
(*@\HLJLnd{@time}@*) (*@\HLJLk{begin}@*)
(*@\HLJLn{errs}@*) (*@\HLJLoB{=}@*) 
(*@\HLJLp{[(}@*) (*@\HLJLn{h}@*) (*@\HLJLoB{=}@*) (*@\HLJLni{2}@*)(*@\HLJLn{\ensuremath{\pi}}@*)(*@\HLJLoB{/}@*)(*@\HLJLn{n}@*)(*@\HLJLp{;}@*)
   (*@\HLJLcs{{\#}}@*) (*@\HLJLcs{differentiation}@*) (*@\HLJLcs{matrix:}@*)
   (*@\HLJLn{Dn}@*) (*@\HLJLoB{=}@*) (*@\HLJLni{1}@*)(*@\HLJLoB{/}@*)(*@\HLJLn{h}@*)(*@\HLJLoB{*}@*)(*@\HLJLnf{spdiagm}@*)(*@\HLJLp{(}@*)(*@\HLJLni{1}@*)(*@\HLJLoB{=>}@*)(*@\HLJLnf{fill}@*)(*@\HLJLp{(}@*)(*@\HLJLni{2}@*)(*@\HLJLoB{/}@*)(*@\HLJLni{3}@*)(*@\HLJLp{,}@*)(*@\HLJLn{n}@*)(*@\HLJLoB{-}@*)(*@\HLJLni{1}@*)(*@\HLJLp{),}@*)(*@\HLJLni{2}@*)(*@\HLJLoB{=>}@*)(*@\HLJLnf{fill}@*)(*@\HLJLp{(}@*)(*@\HLJLoB{-}@*)(*@\HLJLni{1}@*)(*@\HLJLoB{/}@*)(*@\HLJLni{12}@*)(*@\HLJLp{,}@*)(*@\HLJLn{n}@*)(*@\HLJLoB{-}@*)(*@\HLJLni{2}@*)(*@\HLJLp{),}@*)(*@\HLJLn{n}@*)(*@\HLJLoB{-}@*)(*@\HLJLni{2}@*)(*@\HLJLoB{=>}@*)(*@\HLJLnf{fill}@*)(*@\HLJLp{(}@*)(*@\HLJLni{1}@*)(*@\HLJLoB{/}@*)(*@\HLJLni{12}@*)(*@\HLJLp{,}@*)(*@\HLJLni{2}@*)(*@\HLJLp{),}@*)(*@\HLJLn{n}@*)(*@\HLJLoB{-}@*)(*@\HLJLni{1}@*)(*@\HLJLoB{=>}@*)(*@\HLJLnf{fill}@*)(*@\HLJLp{(}@*)(*@\HLJLoB{-}@*)(*@\HLJLni{2}@*)(*@\HLJLoB{/}@*)(*@\HLJLni{3}@*)(*@\HLJLp{,}@*)(*@\HLJLni{1}@*)(*@\HLJLp{));}@*)
   (*@\HLJLn{Dn}@*) (*@\HLJLoB{=}@*) (*@\HLJLn{Dn}@*) (*@\HLJLoB{-}@*) (*@\HLJLn{Dn}@*)(*@\HLJLoB{{\textquotesingle}}@*)(*@\HLJLp{;}@*)
   (*@\HLJLn{x}@*) (*@\HLJLoB{=}@*) (*@\HLJLnf{range}@*)(*@\HLJLp{(}@*)(*@\HLJLni{0}@*)(*@\HLJLp{,}@*)(*@\HLJLni{2}@*)(*@\HLJLn{\ensuremath{\pi}}@*)(*@\HLJLp{;}@*)(*@\HLJLn{length}@*)(*@\HLJLoB{=}@*)(*@\HLJLn{n}@*)(*@\HLJLoB{+}@*)(*@\HLJLni{1}@*)(*@\HLJLp{)[}@*)(*@\HLJLni{1}@*)(*@\HLJLoB{:}@*)(*@\HLJLk{end}@*)(*@\HLJLoB{-}@*)(*@\HLJLni{1}@*)(*@\HLJLp{];}@*) (*@\HLJLcs{{\#}}@*) (*@\HLJLcs{the}@*) (*@\HLJLcs{nodes}@*) (*@\HLJLcs{x\ensuremath{\_j},}@*) (*@\HLJLcs{j}@*) (*@\HLJLcs{=}@*) (*@\HLJLcs{0,}@*) (*@\HLJLcs{...,}@*) (*@\HLJLcs{n-1}@*)
   (*@\HLJLcs{{\#}}@*) (*@\HLJLcs{compute}@*) (*@\HLJLcs{the}@*) (*@\HLJLcs{maximum}@*) (*@\HLJLcs{error}@*) (*@\HLJLcs{at}@*) (*@\HLJLcs{the}@*) (*@\HLJLcs{nodes}@*)
   (*@\HLJLnf{norm}@*)(*@\HLJLp{(}@*)(*@\HLJLn{Dn}@*)(*@\HLJLoB{*}@*)(*@\HLJLn{f}@*)(*@\HLJLoB{.}@*)(*@\HLJLp{(}@*)(*@\HLJLn{x}@*)(*@\HLJLp{)}@*) (*@\HLJLoB{-}@*) (*@\HLJLn{Df}@*)(*@\HLJLoB{.}@*)(*@\HLJLp{(}@*)(*@\HLJLn{x}@*)(*@\HLJLp{),}@*)(*@\HLJLn{Inf}@*)(*@\HLJLp{)}@*) (*@\HLJLp{)}@*) (*@\HLJLk{for}@*) (*@\HLJLn{n}@*) (*@\HLJLoB{=}@*) (*@\HLJLn{nv}@*)(*@\HLJLp{]}@*)
(*@\HLJLk{end}@*)(*@\HLJLp{;}@*)
(*@\HLJLcs{{\#}}@*) (*@\HLJLcs{plot}@*) (*@\HLJLcs{the}@*) (*@\HLJLcs{errors}@*) (*@\HLJLcs{on}@*) (*@\HLJLcs{a}@*) (*@\HLJLcs{log-log}@*) (*@\HLJLcs{scale}@*)
(*@\HLJLnf{scatter}@*)(*@\HLJLp{(}@*)(*@\HLJLn{nv}@*)(*@\HLJLp{,}@*)(*@\HLJLn{errs}@*)(*@\HLJLp{;}@*)(*@\HLJLn{xscale}@*)(*@\HLJLoB{=:}@*)(*@\HLJLn{log10}@*)(*@\HLJLp{,}@*)(*@\HLJLn{yscale}@*)(*@\HLJLoB{=:}@*)(*@\HLJLn{log10}@*)(*@\HLJLp{,}@*)(*@\HLJLn{label}@*)(*@\HLJLoB{=}@*)(*@\HLJLs{"{}Finite}@*) (*@\HLJLs{difference}@*) (*@\HLJLs{error"{}}@*)(*@\HLJLp{,}@*)(*@\HLJLn{xlabel}@*)(*@\HLJLoB{=}@*)(*@\HLJLs{"{}n"{}}@*)(*@\HLJLp{)}@*)
\end{lstlisting}

\begin{lstlisting}
1.643177 seconds (514.34 k allocations: 497.255 MiB, 52.42(*@{{\%}}@*) gc time, 24.6
7(*@{{\%}}@*) compilation time)
\end{lstlisting}

\includegraphics[width=\linewidth]{/figures/Chapter1_Exercises_Solutions_2_1.pdf}

The errors decrease as $n$ is increased but for large enough $n$, the errors increase due to cancellation error in double precision.   To estimate the rate of convergence, we need to estimate the slope of the line along which the errors decrease.  Here is an estimate of the slope:


\begin{lstlisting}
(*@\HLJLp{(}@*)(*@\HLJLnf{log}@*)(*@\HLJLp{(}@*)(*@\HLJLn{errs}@*)(*@\HLJLp{[}@*)(*@\HLJLni{6}@*)(*@\HLJLp{])}@*)(*@\HLJLoB{-}@*)(*@\HLJLnf{log}@*)(*@\HLJLp{(}@*)(*@\HLJLn{errs}@*)(*@\HLJLp{[}@*)(*@\HLJLni{4}@*)(*@\HLJLp{]))}@*)(*@\HLJLoB{/}@*)(*@\HLJLp{(}@*)(*@\HLJLnf{log}@*)(*@\HLJLp{(}@*)(*@\HLJLn{nv}@*)(*@\HLJLp{[}@*)(*@\HLJLni{6}@*)(*@\HLJLp{])}@*) (*@\HLJLoB{-}@*) (*@\HLJLnf{log}@*)(*@\HLJLp{(}@*)(*@\HLJLn{nv}@*)(*@\HLJLp{[}@*)(*@\HLJLni{4}@*)(*@\HLJLp{]))}@*)
\end{lstlisting}

\begin{lstlisting}
-3.9935549986049828
\end{lstlisting}


The slope is approximately $-4$, hence the error decreases as $\mathcal{O}(n^{-4})$, $n \to \infty$ or $\mathcal{O}(h^{4})$, $h \to 0$.



\end{document}
