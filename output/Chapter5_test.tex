\documentclass[12pt,a4paper]{article}

\usepackage[a4paper,text={16.5cm,25.2cm},centering]{geometry}
\usepackage{lmodern}
\usepackage{amssymb,amsmath}
\usepackage{bm}
\usepackage{graphicx}
\usepackage{microtype}
\usepackage{hyperref}
\setlength{\parindent}{0pt}
\setlength{\parskip}{1.2ex}

\hypersetup
       {   pdfauthor = { Marco Fasondini },
           pdftitle={ foo },
           colorlinks=TRUE,
           linkcolor=black,
           citecolor=blue,
           urlcolor=blue
       }




\usepackage{upquote}
\usepackage{listings}
\usepackage{xcolor}
\lstset{
    basicstyle=\ttfamily\footnotesize,
    upquote=true,
    breaklines=true,
    breakindent=0pt,
    keepspaces=true,
    showspaces=false,
    columns=fullflexible,
    showtabs=false,
    showstringspaces=false,
    escapeinside={(*@}{@*)},
    extendedchars=true,
}
\newcommand{\HLJLt}[1]{#1}
\newcommand{\HLJLw}[1]{#1}
\newcommand{\HLJLe}[1]{#1}
\newcommand{\HLJLeB}[1]{#1}
\newcommand{\HLJLo}[1]{#1}
\newcommand{\HLJLk}[1]{\textcolor[RGB]{148,91,176}{\textbf{#1}}}
\newcommand{\HLJLkc}[1]{\textcolor[RGB]{59,151,46}{\textit{#1}}}
\newcommand{\HLJLkd}[1]{\textcolor[RGB]{214,102,97}{\textit{#1}}}
\newcommand{\HLJLkn}[1]{\textcolor[RGB]{148,91,176}{\textbf{#1}}}
\newcommand{\HLJLkp}[1]{\textcolor[RGB]{148,91,176}{\textbf{#1}}}
\newcommand{\HLJLkr}[1]{\textcolor[RGB]{148,91,176}{\textbf{#1}}}
\newcommand{\HLJLkt}[1]{\textcolor[RGB]{148,91,176}{\textbf{#1}}}
\newcommand{\HLJLn}[1]{#1}
\newcommand{\HLJLna}[1]{#1}
\newcommand{\HLJLnb}[1]{#1}
\newcommand{\HLJLnbp}[1]{#1}
\newcommand{\HLJLnc}[1]{#1}
\newcommand{\HLJLncB}[1]{#1}
\newcommand{\HLJLnd}[1]{\textcolor[RGB]{214,102,97}{#1}}
\newcommand{\HLJLne}[1]{#1}
\newcommand{\HLJLneB}[1]{#1}
\newcommand{\HLJLnf}[1]{\textcolor[RGB]{66,102,213}{#1}}
\newcommand{\HLJLnfm}[1]{\textcolor[RGB]{66,102,213}{#1}}
\newcommand{\HLJLnp}[1]{#1}
\newcommand{\HLJLnl}[1]{#1}
\newcommand{\HLJLnn}[1]{#1}
\newcommand{\HLJLno}[1]{#1}
\newcommand{\HLJLnt}[1]{#1}
\newcommand{\HLJLnv}[1]{#1}
\newcommand{\HLJLnvc}[1]{#1}
\newcommand{\HLJLnvg}[1]{#1}
\newcommand{\HLJLnvi}[1]{#1}
\newcommand{\HLJLnvm}[1]{#1}
\newcommand{\HLJLl}[1]{#1}
\newcommand{\HLJLld}[1]{\textcolor[RGB]{148,91,176}{\textit{#1}}}
\newcommand{\HLJLs}[1]{\textcolor[RGB]{201,61,57}{#1}}
\newcommand{\HLJLsa}[1]{\textcolor[RGB]{201,61,57}{#1}}
\newcommand{\HLJLsb}[1]{\textcolor[RGB]{201,61,57}{#1}}
\newcommand{\HLJLsc}[1]{\textcolor[RGB]{201,61,57}{#1}}
\newcommand{\HLJLsd}[1]{\textcolor[RGB]{201,61,57}{#1}}
\newcommand{\HLJLsdB}[1]{\textcolor[RGB]{201,61,57}{#1}}
\newcommand{\HLJLsdC}[1]{\textcolor[RGB]{201,61,57}{#1}}
\newcommand{\HLJLse}[1]{\textcolor[RGB]{59,151,46}{#1}}
\newcommand{\HLJLsh}[1]{\textcolor[RGB]{201,61,57}{#1}}
\newcommand{\HLJLsi}[1]{#1}
\newcommand{\HLJLso}[1]{\textcolor[RGB]{201,61,57}{#1}}
\newcommand{\HLJLsr}[1]{\textcolor[RGB]{201,61,57}{#1}}
\newcommand{\HLJLss}[1]{\textcolor[RGB]{201,61,57}{#1}}
\newcommand{\HLJLssB}[1]{\textcolor[RGB]{201,61,57}{#1}}
\newcommand{\HLJLnB}[1]{\textcolor[RGB]{59,151,46}{#1}}
\newcommand{\HLJLnbB}[1]{\textcolor[RGB]{59,151,46}{#1}}
\newcommand{\HLJLnfB}[1]{\textcolor[RGB]{59,151,46}{#1}}
\newcommand{\HLJLnh}[1]{\textcolor[RGB]{59,151,46}{#1}}
\newcommand{\HLJLni}[1]{\textcolor[RGB]{59,151,46}{#1}}
\newcommand{\HLJLnil}[1]{\textcolor[RGB]{59,151,46}{#1}}
\newcommand{\HLJLnoB}[1]{\textcolor[RGB]{59,151,46}{#1}}
\newcommand{\HLJLoB}[1]{\textcolor[RGB]{102,102,102}{\textbf{#1}}}
\newcommand{\HLJLow}[1]{\textcolor[RGB]{102,102,102}{\textbf{#1}}}
\newcommand{\HLJLp}[1]{#1}
\newcommand{\HLJLc}[1]{\textcolor[RGB]{153,153,119}{\textit{#1}}}
\newcommand{\HLJLch}[1]{\textcolor[RGB]{153,153,119}{\textit{#1}}}
\newcommand{\HLJLcm}[1]{\textcolor[RGB]{153,153,119}{\textit{#1}}}
\newcommand{\HLJLcp}[1]{\textcolor[RGB]{153,153,119}{\textit{#1}}}
\newcommand{\HLJLcpB}[1]{\textcolor[RGB]{153,153,119}{\textit{#1}}}
\newcommand{\HLJLcs}[1]{\textcolor[RGB]{153,153,119}{\textit{#1}}}
\newcommand{\HLJLcsB}[1]{\textcolor[RGB]{153,153,119}{\textit{#1}}}
\newcommand{\HLJLg}[1]{#1}
\newcommand{\HLJLgd}[1]{#1}
\newcommand{\HLJLge}[1]{#1}
\newcommand{\HLJLgeB}[1]{#1}
\newcommand{\HLJLgh}[1]{#1}
\newcommand{\HLJLgi}[1]{#1}
\newcommand{\HLJLgo}[1]{#1}
\newcommand{\HLJLgp}[1]{#1}
\newcommand{\HLJLgs}[1]{#1}
\newcommand{\HLJLgsB}[1]{#1}
\newcommand{\HLJLgt}[1]{#1}



\def\qqand{\qquad\hbox{and}\qquad}
\def\qqfor{\qquad\hbox{for}\qquad}
\def\qqas{\qquad\hbox{as}\qquad}
\def\half{ {1 \over 2} }
\def\D{ {\rm d} }
\def\I{ {\rm i} }
\def\E{ {\rm e} }
\def\C{ {\mathbb C} }
\def\R{ {\mathbb R} }
\def\bbR{ {\mathbb R} }
\def\H{ {\mathbb H} }
\def\Z{ {\mathbb Z} }
\def\CC{ {\cal C} }
\def\FF{ {\cal F} }
\def\HH{ {\cal H} }
\def\LL{ {\cal L} }
\def\vc#1{ {\mathbf #1} }
\def\bbC{ {\mathbb C} }



\def\fR{ f_{\rm R} }
\def\fL{ f_{\rm L} }

\def\qqqquad{\qquad\qquad}
\def\qqwhere{\qquad\hbox{where}\qquad}
\def\Res_#1{\underset{#1}{\rm Res}\,}
\def\sech{ {\rm sech}\, }
\def\acos{ {\rm acos}\, }
\def\asin{ {\rm asin}\, }
\def\atan{ {\rm atan}\, }
\def\Ei{ {\rm Ei}\, }
\def\upepsilon{\varepsilon}


\def\Xint#1{ \mathchoice
   {\XXint\displaystyle\textstyle{#1} }%
   {\XXint\textstyle\scriptstyle{#1} }%
   {\XXint\scriptstyle\scriptscriptstyle{#1} }%
   {\XXint\scriptscriptstyle\scriptscriptstyle{#1} }%
   \!\int}
\def\XXint#1#2#3{ {\setbox0=\hbox{$#1{#2#3}{\int}$}
     \vcenter{\hbox{$#2#3$}}\kern-.5\wd0} }
\def\ddashint{\Xint=}
\def\dashint{\Xint-}
% \def\dashint
\def\infdashint{\dashint_{-\infty}^\infty}




\def\addtab#1={#1\;&=}
\def\ccr{\\\addtab}
\def\ip<#1>{\left\langle{#1}\right\rangle}
\def\dx{\D x}
\def\dt{\D t}
\def\dz{\D z}
\def\ds{\D s}

\def\rR{ {\rm R} }
\def\rL{ {\rm L} }

\def\norm#1{\left\| #1 \right\|}

\def\pr(#1){\left({#1}\right)}
\def\br[#1]{\left[{#1}\right]}

\def\abs#1{\left|{#1}\right|}
\def\fpr(#1){\!\pr({#1})}

\def\sopmatrix#1{ \begin{pmatrix}#1\end{pmatrix} }

\def\endash{–}
\def\emdash{—}
\def\mdblksquare{\blacksquare}
\def\lgblksquare{\blacksquare}
\def\scre{\E}
\def\mapengine#1,#2.{\mapfunction{#1}\ifx\void#2\else\mapengine #2.\fi }

\def\map[#1]{\mapengine #1,\void.}

\def\mapenginesep_#1#2,#3.{\mapfunction{#2}\ifx\void#3\else#1\mapengine #3.\fi }

\def\mapsep_#1[#2]{\mapenginesep_{#1}#2,\void.}


\def\vcbr[#1]{\pr(#1)}


\def\bvect[#1,#2]{
{
\def\dots{\cdots}
\def\mapfunction##1{\ | \  ##1}
	\sopmatrix{
		 \,#1\map[#2]\,
	}
}
}



\def\vect[#1]{
{\def\dots{\ldots}
	\vcbr[{#1}]
} }

\def\vectt[#1]{
{\def\dots{\ldots}
	\vect[{#1}]^{\top}
} }

\def\Vectt[#1]{
{
\def\mapfunction##1{##1 \cr}
\def\dots{\vdots}
	\begin{pmatrix}
		\map[#1]
	\end{pmatrix}
} }

\def\addtab#1={#1\;&=}
\def\ccr{\\\addtab}

\def\questionequals{= \!\!\!\!\!\!{\scriptstyle ? \atop }\,\,\,}

\def\Ei{\rm Ei\,}

\begin{document}

\section{Chapter 5: Consistency, stability and convergence of methods for evolutionary PDEs}
In this chapter we 

\begin{itemize}
\item[1. ] 
\item[2. ] 
\item[3. ] 
\item[4. ] 
\item[5. ] \end{itemize}
Before we looked at function approximation methods... here we analyse methods for PDEs directly

\subsection{Parabolic equations?}
\subsubsection{A finite difference method for the diffusion equation}
Consider the $(1+1)$-dimensional diffusion equation,

\[
\frac{\partial u}{\partial t}=\frac{\partial^2 u}{\partial x^2}, \qquad x \in (0, 1),\qquad t \in (0, T]
\]
subject to the initial and boundary data

\[
u(x,0) = f(x), \qquad 0 \leq x \leq 1, \qquad u(0,t) = \varphi_0(t), \qquad u(1,t) = \varphi_1(t), \qquad t \geq 0.
\]
We shall analyse a finite difference method for this PDE initial / boundary value problem.

We discretise the spatial variable as follows,

\[
x_j = j h, \qquad h = \frac{1}{n_x + 1}, \qquad j = 0, \ldots, n_x + 1,
\]
and the temporal variable,

\[
t_i = i\tau, \qquad \tau = \frac{T}{n_t}, \qquad i = 0, 1, 2, \ldots, n_t.
\]
Let 

\[
u(x_j, t_i) := \tilde{u}^i_j \approx u^{i}_j.
\]
We set

\[
u(x_j, 0) = \tilde{u}^0_j =  f(x_j) =  u^{0}_j, \qquad j = 1, \ldots, n_x
\]
and

\[
u(0,t_i)  = \tilde{u}^i_0 = \varphi_0(t_i) =  u^i_0, \qquad u(1,t_i) =  \tilde{u}^i_{n_x + 1} = \varphi_1(t_i) =  u^i_{n_x +1}, \qquad i \geq 0.
\]
\subsubsection{Order and convergence}
Let's replace the time derivative with a (first-order) forward difference and the second spatial derivative with a (second-order) central difference approximation: from Taylor's theorem (see Chapter 1), there exists a $\xi \in [t_i, t_{i+1}]$ such that

\[
\frac{u(x_j,t_{i+1}) - u(x_j,t_{i})}{\tau} =\frac{\tilde{u}^{i+1}_j - \tilde{u}^i_j}{\tau} = u_t(x_j,t_i) + \frac{\tau}{2}u_{tt}(x_j,\xi).
\]
Let's assume that $u_{tt}(x,t)$ exists and is bounded on $(x,t) \in [0,1]\times[0,T]$, then we have

\[
\frac{\tilde{u}^{i+1}_j - \tilde{u}^i_j}{\tau} = u_t(x_j,t_i) + \mathcal{O}(\tau), \qquad \tau \to 0.
\]
Similarly, if we assume that $u_{xxxx}(x,t)$ exists and is bounded on $(x,t) \in [0,1]\times[0,T]$, then we have (see Chapter 1 Exercises)

\[
\frac{\tilde{u}^{i}_{j+1} - 2\tilde{u}^i_j + \tilde{u}^i_{j-1}}{h^2} = u_{xx}(x_j,t_i) + \mathcal{O}(h^2), \qquad h \to 0.
\]
We define

\[
\mu = \frac{\tau}{h^2},
\]
which is known as the Courant number and specify that $\mu$ be held constant as $\tau, h \to 0$. Hence, we have that

\[
\frac{\tilde{u}^{i+1}_j - \tilde{u}^i_j}{\tau} - \frac{\tilde{u}^{i}_{j+1} - 2\tilde{u}^i_j + \tilde{u}^i_{j-1}}{h^2} = u_t(x_j,t_i) - u_{xx}(x_j,t_i) + \mathcal{O}(h^2) =  \mathcal{O}(h^2), \qquad h \to 0.
\]
We shall approximate the solution to the diffusion equation with the finite difference method

\[
\frac{u^{i+1}_j - u^i_j}{\tau} - \frac{u^{i}_{j+1} - 2u^i_j + u^i_{j-1}}{h^2} = 0
\]
or

\[
u^{i+1}_j = u^i_j + \mu \left( u^{i}_{j+1} - 2u^i_j + u^i_{j-1}  \right).
\]
This method is known as \textbf{Euler's method}. Notice we have shown that if the exact solution is substituted into the finite difference method, then the PDE is recovered exactly and the local error tends to zero as the discretisation parameters tend to zero.  The method is \textbf{second order} since the local error approaches zero at the rate $\mathcal{O}(h^2)$ as $h \to 0$. A finite difference method of order $p \geq 1$ is said to be \textbf{consistent}.  Notice that for consistency, we only require the error at $(x_j,t_i)$  to go to zero as the discretisation parameters tend to zero, hence the order of a method measures the local error.  A method is said to be \textbf{convergent} if the \emph{global} error tends to zero as the discretisation parameters tend to zero, i.e., if

\[
\lim_{h \to 0}\left[\lim_{j \to x/h}\left( \lim_{i \to t/\tau} u^i_j \right)   \right] = u(x,t), \qquad x \in [0, 1], \qquad t \in [0, T],
\]
where $\mu = \tau/h^2$ is kept constant as $h \to 0$.

\subsubsection{Numerical experiments with Euler's method}
Before we analyse the convergence or otherwise of Euler's method, let's perform some numerical experiments.  First, we note that we can express Euler's method as

\[
\underbrace{\begin{bmatrix}
u^{i+1}_{1} \\
\vdots \\
\vdots \\
\vdots \\
u^{i+1}_{n_x}
\end{bmatrix}}_{\mathbf{u}^{i+1}} = 
\underbrace{\begin{bmatrix}
1 - 2\mu & \mu & & & \\
\mu  & 1-2\mu & \mu  & & \\
      & \ddots & \ddots & \ddots & \\
      &        & \mu    & 1- 2\mu & \mu \\
      &        &        &\mu      & 1-2\mu
\end{bmatrix}}_{A}
\underbrace{\begin{bmatrix}
u^{i}_{1} \\
\vdots \\
\vdots \\
\vdots \\
u^{i}_{n_x}
\end{bmatrix}}_{\mathbf{u}^i}
+
\underbrace{\begin{bmatrix}
\mu\varphi_0(t_i) \\
0 \\
\vdots \\
0 \\
\mu \varphi_1(t_i)
\end{bmatrix}}_{\mathbf{k}^i}
\]
i.e., 

\[
\mathbf{u}^{i+1} = A\mathbf{u}^i + \mathbf{k}^i, \qquad i = 0, \ldots, n_t-1.
\]

\begin{lstlisting}
(*@\HLJLk{using}@*) (*@\HLJLn{LinearAlgebra}@*)(*@\HLJLp{,}@*) (*@\HLJLn{Plots}@*)
\end{lstlisting}


\begin{lstlisting}
(*@\HLJLk{function}@*) (*@\HLJLnf{Euler}@*)(*@\HLJLp{(}@*)(*@\HLJLn{f}@*)(*@\HLJLp{,}@*)(*@\HLJLn{\ensuremath{\phi}0}@*)(*@\HLJLp{,}@*)(*@\HLJLn{\ensuremath{\phi}1}@*)(*@\HLJLp{,}@*)(*@\HLJLn{nx}@*)(*@\HLJLp{,}@*)(*@\HLJLn{\ensuremath{\mu}}@*)(*@\HLJLp{,}@*)(*@\HLJLn{T}@*)(*@\HLJLp{)}@*)
    
    (*@\HLJLn{x}@*) (*@\HLJLoB{=}@*) (*@\HLJLnf{range}@*)(*@\HLJLp{(}@*)(*@\HLJLni{0}@*)(*@\HLJLp{,}@*)(*@\HLJLni{1}@*)(*@\HLJLp{,}@*)(*@\HLJLn{nx}@*) (*@\HLJLoB{+}@*) (*@\HLJLni{2}@*)(*@\HLJLp{)}@*)
    (*@\HLJLn{h}@*) (*@\HLJLoB{=}@*) (*@\HLJLni{1}@*)(*@\HLJLoB{/}@*)(*@\HLJLp{(}@*)(*@\HLJLn{nx}@*)(*@\HLJLoB{+}@*)(*@\HLJLni{1}@*)(*@\HLJLp{)}@*)
    (*@\HLJLn{\ensuremath{\tau}}@*) (*@\HLJLoB{=}@*) (*@\HLJLn{\ensuremath{\mu}}@*)(*@\HLJLoB{*}@*)(*@\HLJLn{h}@*)(*@\HLJLoB{{\textasciicircum}}@*)(*@\HLJLni{2}@*)
    (*@\HLJLn{t}@*) (*@\HLJLoB{=}@*) (*@\HLJLni{0}@*)(*@\HLJLoB{:}@*)(*@\HLJLn{\ensuremath{\tau}}@*)(*@\HLJLoB{:}@*)(*@\HLJLn{T}@*)
    (*@\HLJLn{nt}@*) (*@\HLJLoB{=}@*) (*@\HLJLnf{length}@*)(*@\HLJLp{(}@*)(*@\HLJLn{t}@*)(*@\HLJLp{)}@*)(*@\HLJLoB{-}@*)(*@\HLJLni{1}@*)
    (*@\HLJLn{A}@*) (*@\HLJLoB{=}@*) (*@\HLJLnf{SymTridiagonal}@*)(*@\HLJLp{(}@*)(*@\HLJLnf{fill}@*)(*@\HLJLp{((}@*)(*@\HLJLni{1}@*) (*@\HLJLoB{-}@*) (*@\HLJLni{2}@*)(*@\HLJLn{\ensuremath{\mu}}@*)(*@\HLJLp{),}@*)(*@\HLJLn{nx}@*)(*@\HLJLp{),}@*)(*@\HLJLnf{fill}@*)(*@\HLJLp{(}@*)(*@\HLJLn{\ensuremath{\mu}}@*)(*@\HLJLp{,}@*)(*@\HLJLn{nx}@*)(*@\HLJLoB{-}@*)(*@\HLJLni{1}@*)(*@\HLJLp{))}@*)
    (*@\HLJLn{u}@*) (*@\HLJLoB{=}@*) (*@\HLJLnf{zeros}@*)(*@\HLJLp{(}@*)(*@\HLJLn{nt}@*)(*@\HLJLoB{+}@*)(*@\HLJLni{1}@*)(*@\HLJLp{,}@*)(*@\HLJLn{nx}@*)(*@\HLJLoB{+}@*)(*@\HLJLni{2}@*)(*@\HLJLp{)}@*)
    (*@\HLJLn{u}@*)(*@\HLJLp{[}@*)(*@\HLJLoB{:}@*)(*@\HLJLp{,}@*)(*@\HLJLni{1}@*)(*@\HLJLp{]}@*) (*@\HLJLoB{=}@*) (*@\HLJLn{\ensuremath{\phi}0}@*)(*@\HLJLoB{.}@*)(*@\HLJLp{(}@*)(*@\HLJLn{t}@*)(*@\HLJLp{)}@*)
    (*@\HLJLn{u}@*)(*@\HLJLp{[}@*)(*@\HLJLoB{:}@*)(*@\HLJLp{,}@*)(*@\HLJLn{nx}@*)(*@\HLJLoB{+}@*)(*@\HLJLni{2}@*)(*@\HLJLp{]}@*) (*@\HLJLoB{=}@*) (*@\HLJLn{\ensuremath{\phi}1}@*)(*@\HLJLoB{.}@*)(*@\HLJLp{(}@*)(*@\HLJLn{t}@*)(*@\HLJLp{)}@*)
    (*@\HLJLn{u}@*)(*@\HLJLp{[}@*)(*@\HLJLni{1}@*)(*@\HLJLp{,}@*)(*@\HLJLni{2}@*)(*@\HLJLoB{:}@*)(*@\HLJLn{nx}@*)(*@\HLJLoB{+}@*)(*@\HLJLni{1}@*)(*@\HLJLp{]}@*) (*@\HLJLoB{=}@*) (*@\HLJLn{f}@*)(*@\HLJLoB{.}@*)(*@\HLJLp{(}@*)(*@\HLJLn{x}@*)(*@\HLJLp{[}@*)(*@\HLJLni{2}@*)(*@\HLJLoB{:}@*)(*@\HLJLn{nx}@*)(*@\HLJLoB{+}@*)(*@\HLJLni{1}@*)(*@\HLJLp{])}@*)

    (*@\HLJLk{for}@*) (*@\HLJLn{i}@*) (*@\HLJLoB{=}@*) (*@\HLJLni{1}@*)(*@\HLJLoB{:}@*)(*@\HLJLn{nt}@*)
        (*@\HLJLn{u}@*)(*@\HLJLp{[}@*)(*@\HLJLn{i}@*)(*@\HLJLoB{+}@*)(*@\HLJLni{1}@*)(*@\HLJLp{,}@*)(*@\HLJLni{2}@*)(*@\HLJLoB{:}@*)(*@\HLJLn{nx}@*)(*@\HLJLoB{+}@*)(*@\HLJLni{1}@*)(*@\HLJLp{]}@*) (*@\HLJLoB{=}@*) (*@\HLJLn{A}@*)(*@\HLJLoB{*}@*)(*@\HLJLn{u}@*)(*@\HLJLp{[}@*)(*@\HLJLn{i}@*)(*@\HLJLp{,}@*)(*@\HLJLni{2}@*)(*@\HLJLoB{:}@*)(*@\HLJLn{nx}@*)(*@\HLJLoB{+}@*)(*@\HLJLni{1}@*)(*@\HLJLp{]}@*) 
        (*@\HLJLn{u}@*)(*@\HLJLp{[}@*)(*@\HLJLn{i}@*)(*@\HLJLoB{+}@*)(*@\HLJLni{1}@*)(*@\HLJLp{,}@*)(*@\HLJLni{2}@*)(*@\HLJLp{]}@*) (*@\HLJLoB{+=}@*) (*@\HLJLn{\ensuremath{\mu}}@*)(*@\HLJLoB{*}@*)(*@\HLJLnf{\ensuremath{\phi}0}@*)(*@\HLJLp{(}@*)(*@\HLJLn{t}@*)(*@\HLJLp{[}@*)(*@\HLJLn{i}@*)(*@\HLJLp{])}@*)
        (*@\HLJLn{u}@*)(*@\HLJLp{[}@*)(*@\HLJLn{i}@*)(*@\HLJLoB{+}@*)(*@\HLJLni{1}@*)(*@\HLJLp{,}@*)(*@\HLJLn{nx}@*)(*@\HLJLoB{+}@*)(*@\HLJLni{1}@*)(*@\HLJLp{]}@*) (*@\HLJLoB{+=}@*) (*@\HLJLn{\ensuremath{\mu}}@*)(*@\HLJLoB{*}@*)(*@\HLJLnf{\ensuremath{\phi}1}@*)(*@\HLJLp{(}@*)(*@\HLJLn{t}@*)(*@\HLJLp{[}@*)(*@\HLJLn{i}@*)(*@\HLJLp{])}@*)
    (*@\HLJLk{end}@*)
    
    (*@\HLJLn{u}@*)(*@\HLJLp{,}@*) (*@\HLJLn{x}@*)(*@\HLJLp{,}@*) (*@\HLJLn{t}@*)
(*@\HLJLk{end}@*)
\end{lstlisting}

\begin{lstlisting}
Euler (generic function with 1 method)
\end{lstlisting}


The diffusion initial / boundary value problem with 

\[
u(x,0) = f(x) = \sin \frac{1}{2}\pi x + \frac{1}{2}\sin 2\pi x, \qquad 0 \leq x \leq 1, 
\]
and

\[
u(0,t) = \varphi_0(t) = 0, \qquad  u(1,t) = \varphi_1(t) = {\rm e}^{-\pi^2 t/4}, \qquad t \geq 0,
\]
has the exact solution

\[
u(x,t) = {\rm e}^{-\pi^2 t/4}\sin \frac{1}{2}\pi x + \frac{1}{2}{\rm e}^{-4\pi^2 t}\sin 2\pi x, \qquad 0 \leq x \leq 1, \qquad t \geq 0.
\]

\begin{lstlisting}
(*@\HLJLn{f}@*) (*@\HLJLoB{=}@*) (*@\HLJLn{x}@*) (*@\HLJLoB{->}@*) (*@\HLJLnf{sin}@*)(*@\HLJLp{(}@*)(*@\HLJLn{\ensuremath{\pi}}@*)(*@\HLJLoB{*}@*)(*@\HLJLn{x}@*)(*@\HLJLoB{/}@*)(*@\HLJLni{2}@*)(*@\HLJLp{)}@*) (*@\HLJLoB{+}@*) (*@\HLJLnfB{0.5}@*)(*@\HLJLoB{*}@*)(*@\HLJLnf{sin}@*)(*@\HLJLp{(}@*)(*@\HLJLni{2}@*)(*@\HLJLn{\ensuremath{\pi}}@*)(*@\HLJLoB{*}@*)(*@\HLJLn{x}@*)(*@\HLJLp{)}@*)
(*@\HLJLn{\ensuremath{\phi}1}@*) (*@\HLJLoB{=}@*) (*@\HLJLn{t}@*) (*@\HLJLoB{->}@*) (*@\HLJLnf{exp}@*)(*@\HLJLp{(}@*)(*@\HLJLoB{-}@*)(*@\HLJLn{\ensuremath{\pi}}@*)(*@\HLJLoB{{\textasciicircum}}@*)(*@\HLJLni{2}@*)(*@\HLJLoB{*}@*)(*@\HLJLn{t}@*)(*@\HLJLoB{/}@*)(*@\HLJLni{4}@*)(*@\HLJLp{)}@*)
(*@\HLJLn{\ensuremath{\phi}0}@*) (*@\HLJLoB{=}@*) (*@\HLJLn{t}@*) (*@\HLJLoB{->}@*) (*@\HLJLni{0}@*)
(*@\HLJLn{nx}@*) (*@\HLJLoB{=}@*) (*@\HLJLni{50}@*)
(*@\HLJLn{\ensuremath{\mu}}@*) (*@\HLJLoB{=}@*) (*@\HLJLnfB{0.50}@*)
(*@\HLJLn{T}@*) (*@\HLJLoB{=}@*) (*@\HLJLni{1}@*)
(*@\HLJLn{u}@*)(*@\HLJLp{,}@*)(*@\HLJLn{x}@*)(*@\HLJLp{,}@*)(*@\HLJLn{t}@*) (*@\HLJLoB{=}@*) (*@\HLJLnf{Euler}@*)(*@\HLJLp{(}@*)(*@\HLJLn{f}@*)(*@\HLJLp{,}@*)(*@\HLJLn{\ensuremath{\phi}0}@*)(*@\HLJLp{,}@*)(*@\HLJLn{\ensuremath{\phi}1}@*)(*@\HLJLp{,}@*)(*@\HLJLn{nx}@*)(*@\HLJLp{,}@*)(*@\HLJLn{\ensuremath{\mu}}@*)(*@\HLJLp{,}@*)(*@\HLJLn{T}@*)(*@\HLJLp{)}@*)
(*@\HLJLn{nt}@*) (*@\HLJLoB{=}@*) (*@\HLJLnf{length}@*)(*@\HLJLp{(}@*)(*@\HLJLn{t}@*)(*@\HLJLp{)}@*) (*@\HLJLoB{-}@*)(*@\HLJLni{1}@*)
\end{lstlisting}

\begin{lstlisting}
5202
\end{lstlisting}


\begin{lstlisting}
(*@\HLJLn{inc}@*) (*@\HLJLoB{=}@*) (*@\HLJLni{100}@*)
(*@\HLJLnf{surface}@*)(*@\HLJLp{(}@*)(*@\HLJLn{x}@*)(*@\HLJLp{,}@*)(*@\HLJLn{t}@*)(*@\HLJLp{[}@*)(*@\HLJLni{1}@*)(*@\HLJLoB{:}@*)(*@\HLJLn{inc}@*)(*@\HLJLoB{:}@*)(*@\HLJLk{end}@*)(*@\HLJLp{],}@*)(*@\HLJLn{u}@*)(*@\HLJLp{[}@*)(*@\HLJLni{1}@*)(*@\HLJLoB{:}@*)(*@\HLJLn{inc}@*)(*@\HLJLoB{:}@*)(*@\HLJLk{end}@*)(*@\HLJLp{,}@*)(*@\HLJLoB{:}@*)(*@\HLJLp{],}@*)(*@\HLJLn{seriescolor}@*)(*@\HLJLoB{=:}@*)(*@\HLJLn{redsblues}@*)(*@\HLJLp{,}@*) (*@\HLJLn{camera}@*)(*@\HLJLoB{=}@*)(*@\HLJLp{(}@*)(*@\HLJLni{30}@*)(*@\HLJLp{,}@*)(*@\HLJLni{40}@*)(*@\HLJLp{))}@*)
\end{lstlisting}

\includegraphics[width=\linewidth]{/figures/Chapter5_test_4_1.pdf}

\begin{lstlisting}
(*@\HLJLn{xx}@*) (*@\HLJLoB{=}@*) (*@\HLJLn{x}@*)(*@\HLJLoB{{\textquotesingle}}@*) (*@\HLJLoB{.*}@*) (*@\HLJLnf{ones}@*)(*@\HLJLp{(}@*)(*@\HLJLn{nt}@*)(*@\HLJLoB{+}@*)(*@\HLJLni{1}@*)(*@\HLJLp{)}@*)
(*@\HLJLn{tt}@*) (*@\HLJLoB{=}@*) (*@\HLJLnf{ones}@*)(*@\HLJLp{(}@*)(*@\HLJLn{nx}@*)(*@\HLJLoB{+}@*)(*@\HLJLni{2}@*)(*@\HLJLp{)}@*)(*@\HLJLoB{{\textquotesingle}}@*) (*@\HLJLoB{.*}@*) (*@\HLJLn{t}@*)
(*@\HLJLn{ue}@*) (*@\HLJLoB{=}@*) (*@\HLJLp{(}@*)(*@\HLJLn{x}@*)(*@\HLJLp{,}@*)(*@\HLJLn{t}@*)(*@\HLJLp{)}@*) (*@\HLJLoB{->}@*) (*@\HLJLnf{exp}@*)(*@\HLJLp{(}@*)(*@\HLJLoB{-}@*)(*@\HLJLn{\ensuremath{\pi}}@*)(*@\HLJLoB{{\textasciicircum}}@*)(*@\HLJLni{2}@*)(*@\HLJLoB{*}@*)(*@\HLJLn{t}@*)(*@\HLJLoB{/}@*)(*@\HLJLni{4}@*)(*@\HLJLp{)}@*)(*@\HLJLoB{*}@*)(*@\HLJLnf{sin}@*)(*@\HLJLp{(}@*)(*@\HLJLn{\ensuremath{\pi}}@*)(*@\HLJLoB{*}@*)(*@\HLJLn{x}@*)(*@\HLJLoB{/}@*)(*@\HLJLni{2}@*)(*@\HLJLp{)}@*) (*@\HLJLoB{+}@*) (*@\HLJLnfB{0.5}@*)(*@\HLJLoB{*}@*)(*@\HLJLnf{exp}@*)(*@\HLJLp{(}@*)(*@\HLJLoB{-}@*)(*@\HLJLni{4}@*)(*@\HLJLoB{*}@*)(*@\HLJLn{\ensuremath{\pi}}@*)(*@\HLJLoB{{\textasciicircum}}@*)(*@\HLJLni{2}@*)(*@\HLJLoB{*}@*)(*@\HLJLn{t}@*)(*@\HLJLp{)}@*)(*@\HLJLoB{*}@*)(*@\HLJLnf{sin}@*)(*@\HLJLp{(}@*)(*@\HLJLni{2}@*)(*@\HLJLn{\ensuremath{\pi}}@*)(*@\HLJLoB{*}@*)(*@\HLJLn{x}@*)(*@\HLJLp{)}@*)
(*@\HLJLn{error}@*) (*@\HLJLoB{=}@*) (*@\HLJLn{ue}@*)(*@\HLJLoB{.}@*)(*@\HLJLp{(}@*)(*@\HLJLn{xx}@*)(*@\HLJLp{,}@*)(*@\HLJLn{tt}@*)(*@\HLJLp{)}@*) (*@\HLJLoB{-}@*) (*@\HLJLn{u}@*)
(*@\HLJLn{e1}@*) (*@\HLJLoB{=}@*) (*@\HLJLnf{maximum}@*)(*@\HLJLp{(}@*)(*@\HLJLn{error}@*)(*@\HLJLp{)}@*)
\end{lstlisting}

\begin{lstlisting}
0.00047001331601359553
\end{lstlisting}


If we double $n_x$, the the error decreases by roughly a factor of $4$.


\begin{lstlisting}
(*@\HLJLn{nx}@*) (*@\HLJLoB{*=}@*) (*@\HLJLni{2}@*)
(*@\HLJLn{u}@*)(*@\HLJLp{,}@*)(*@\HLJLn{x}@*)(*@\HLJLp{,}@*)(*@\HLJLn{t}@*) (*@\HLJLoB{=}@*) (*@\HLJLnf{Euler}@*)(*@\HLJLp{(}@*)(*@\HLJLn{f}@*)(*@\HLJLp{,}@*)(*@\HLJLn{\ensuremath{\phi}0}@*)(*@\HLJLp{,}@*)(*@\HLJLn{\ensuremath{\phi}1}@*)(*@\HLJLp{,}@*)(*@\HLJLn{nx}@*)(*@\HLJLp{,}@*)(*@\HLJLn{\ensuremath{\mu}}@*)(*@\HLJLp{,}@*)(*@\HLJLn{T}@*)(*@\HLJLp{)}@*)
(*@\HLJLnd{@show}@*) (*@\HLJLn{nt}@*) (*@\HLJLoB{=}@*) (*@\HLJLnf{length}@*)(*@\HLJLp{(}@*)(*@\HLJLn{t}@*)(*@\HLJLp{)}@*) (*@\HLJLoB{-}@*) (*@\HLJLni{1}@*)
(*@\HLJLn{xx}@*) (*@\HLJLoB{=}@*) (*@\HLJLn{x}@*)(*@\HLJLoB{{\textquotesingle}}@*) (*@\HLJLoB{.*}@*) (*@\HLJLnf{ones}@*)(*@\HLJLp{(}@*)(*@\HLJLn{nt}@*)(*@\HLJLoB{+}@*)(*@\HLJLni{1}@*)(*@\HLJLp{)}@*)
(*@\HLJLn{tt}@*) (*@\HLJLoB{=}@*) (*@\HLJLnf{ones}@*)(*@\HLJLp{(}@*)(*@\HLJLn{nx}@*)(*@\HLJLoB{+}@*)(*@\HLJLni{2}@*)(*@\HLJLp{)}@*)(*@\HLJLoB{{\textquotesingle}}@*) (*@\HLJLoB{.*}@*) (*@\HLJLn{t}@*)
(*@\HLJLn{error}@*) (*@\HLJLoB{=}@*) (*@\HLJLn{ue}@*)(*@\HLJLoB{.}@*)(*@\HLJLp{(}@*)(*@\HLJLn{xx}@*)(*@\HLJLp{,}@*)(*@\HLJLn{tt}@*)(*@\HLJLp{)}@*) (*@\HLJLoB{-}@*) (*@\HLJLn{u}@*)
(*@\HLJLn{e2}@*) (*@\HLJLoB{=}@*) (*@\HLJLnf{maximum}@*)(*@\HLJLp{(}@*)(*@\HLJLn{error}@*)(*@\HLJLp{)}@*)
(*@\HLJLn{e1}@*)(*@\HLJLoB{/}@*)(*@\HLJLn{e2}@*)
\end{lstlisting}

\begin{lstlisting}
nt = length(t) - 1 = 20402
3.929469183388294
\end{lstlisting}


This suggests the error decays as $\mathcal{O}(n_x^{-2})$, $n_x \to \infty$.


\begin{lstlisting}
(*@\HLJLn{nx}@*) (*@\HLJLoB{=}@*) (*@\HLJLni{30}@*)
(*@\HLJLn{\ensuremath{\mu}}@*) (*@\HLJLoB{=}@*) (*@\HLJLnfB{0.51}@*)
(*@\HLJLn{u}@*)(*@\HLJLp{,}@*)(*@\HLJLn{x}@*)(*@\HLJLp{,}@*)(*@\HLJLn{t}@*) (*@\HLJLoB{=}@*) (*@\HLJLnf{Euler}@*)(*@\HLJLp{(}@*)(*@\HLJLn{f}@*)(*@\HLJLp{,}@*)(*@\HLJLn{\ensuremath{\phi}0}@*)(*@\HLJLp{,}@*)(*@\HLJLn{\ensuremath{\phi}1}@*)(*@\HLJLp{,}@*)(*@\HLJLn{nx}@*)(*@\HLJLp{,}@*)(*@\HLJLn{\ensuremath{\mu}}@*)(*@\HLJLp{,}@*)(*@\HLJLn{T}@*)(*@\HLJLp{);}@*)
\end{lstlisting}


\begin{lstlisting}
(*@\HLJLnf{surface}@*)(*@\HLJLp{(}@*)(*@\HLJLn{x}@*)(*@\HLJLp{,}@*)(*@\HLJLn{t}@*)(*@\HLJLp{[}@*)(*@\HLJLni{1}@*)(*@\HLJLoB{:}@*)(*@\HLJLni{30}@*)(*@\HLJLoB{:}@*)(*@\HLJLk{end}@*)(*@\HLJLp{],}@*)(*@\HLJLn{u}@*)(*@\HLJLp{[}@*)(*@\HLJLni{1}@*)(*@\HLJLoB{:}@*)(*@\HLJLni{30}@*)(*@\HLJLoB{:}@*)(*@\HLJLk{end}@*)(*@\HLJLp{,}@*)(*@\HLJLoB{:}@*)(*@\HLJLp{],}@*)(*@\HLJLn{seriescolor}@*)(*@\HLJLoB{=:}@*)(*@\HLJLn{redsblues}@*)(*@\HLJLp{,}@*) (*@\HLJLn{camera}@*)(*@\HLJLoB{=}@*)(*@\HLJLp{(}@*)(*@\HLJLni{30}@*)(*@\HLJLp{,}@*)(*@\HLJLni{40}@*)(*@\HLJLp{))}@*)
\end{lstlisting}

\includegraphics[width=\linewidth]{/figures/Chapter5_test_8_1.pdf}

Based on these experiments, we conjecture that Euler's method converges if $\mu \leq \frac{1}{2}$.

The initial condition is "smoothed out" by the diffusion equation.  Here is another example:


\begin{lstlisting}
(*@\HLJLn{f}@*) (*@\HLJLoB{=}@*) (*@\HLJLn{x}@*) (*@\HLJLoB{->}@*) (*@\HLJLn{x}@*) (*@\HLJLoB{<=}@*) (*@\HLJLnfB{0.5}@*) (*@\HLJLoB{?}@*) (*@\HLJLni{2}@*)(*@\HLJLoB{*}@*)(*@\HLJLn{x}@*) (*@\HLJLoB{:}@*) (*@\HLJLni{2}@*)(*@\HLJLoB{-}@*)(*@\HLJLni{2}@*)(*@\HLJLoB{*}@*)(*@\HLJLn{x}@*)
(*@\HLJLn{\ensuremath{\phi}1}@*) (*@\HLJLoB{=}@*) (*@\HLJLn{t}@*) (*@\HLJLoB{->}@*) (*@\HLJLni{0}@*)
(*@\HLJLn{\ensuremath{\phi}0}@*) (*@\HLJLoB{=}@*) (*@\HLJLn{t}@*) (*@\HLJLoB{->}@*) (*@\HLJLni{0}@*)
(*@\HLJLn{nx}@*) (*@\HLJLoB{=}@*) (*@\HLJLni{19}@*)
(*@\HLJLn{\ensuremath{\mu}}@*) (*@\HLJLoB{=}@*) (*@\HLJLnfB{0.50}@*)
(*@\HLJLn{T}@*) (*@\HLJLoB{=}@*) (*@\HLJLnfB{0.25}@*)
(*@\HLJLn{u}@*)(*@\HLJLp{,}@*)(*@\HLJLn{x}@*)(*@\HLJLp{,}@*)(*@\HLJLn{t}@*) (*@\HLJLoB{=}@*) (*@\HLJLnf{Euler}@*)(*@\HLJLp{(}@*)(*@\HLJLn{f}@*)(*@\HLJLp{,}@*)(*@\HLJLn{\ensuremath{\phi}0}@*)(*@\HLJLp{,}@*)(*@\HLJLn{\ensuremath{\phi}1}@*)(*@\HLJLp{,}@*)(*@\HLJLn{nx}@*)(*@\HLJLp{,}@*)(*@\HLJLn{\ensuremath{\mu}}@*)(*@\HLJLp{,}@*)(*@\HLJLn{T}@*)(*@\HLJLp{);}@*)
\end{lstlisting}


\begin{lstlisting}
(*@\HLJLnf{surface}@*)(*@\HLJLp{(}@*)(*@\HLJLn{x}@*)(*@\HLJLp{,}@*)(*@\HLJLn{t}@*)(*@\HLJLp{[}@*)(*@\HLJLni{1}@*)(*@\HLJLoB{:}@*)(*@\HLJLni{30}@*)(*@\HLJLoB{:}@*)(*@\HLJLk{end}@*)(*@\HLJLp{],}@*)(*@\HLJLn{u}@*)(*@\HLJLp{[}@*)(*@\HLJLni{1}@*)(*@\HLJLoB{:}@*)(*@\HLJLni{30}@*)(*@\HLJLoB{:}@*)(*@\HLJLk{end}@*)(*@\HLJLp{,}@*)(*@\HLJLoB{:}@*)(*@\HLJLp{],}@*)(*@\HLJLn{seriescolor}@*)(*@\HLJLoB{=:}@*)(*@\HLJLn{redsblues}@*)(*@\HLJLp{,}@*) (*@\HLJLn{camera}@*)(*@\HLJLoB{=}@*)(*@\HLJLp{(}@*)(*@\HLJLni{30}@*)(*@\HLJLp{,}@*)(*@\HLJLni{40}@*)(*@\HLJLp{))}@*)
\end{lstlisting}

\includegraphics[width=\linewidth]{/figures/Chapter5_test_10_1.pdf}

\subsubsection{Convergence of Euler's method}
\textbf{Proposition} If $\mu \leq \frac{1}{2}$, then Euler's method converges.

\textbf{Proof} Recall that

\[
\frac{\tilde{u}^{i+1}_j - \tilde{u}^i_j}{\tau} - \frac{\tilde{u}^{i}_{j+1} - 2\tilde{u}^i_j + \tilde{u}^i_{j-1}}{h^2}  =  \mathcal{O}(h^2), \qquad h \to 0,
\]
which implies

\[
\tilde{u}^{i+1}_j = \tilde{u}^i_j + \mu \left(\tilde{u}^{i}_{j+1} - 2\tilde{u}^i_j + \tilde{u}^i_{j-1}\right)  +  \mathcal{O}(h^4), \qquad h \to 0.
\]
Let $e^i_j = \tilde{u}^i_j - u^i_j$, then since

\[
u^{i+1}_j = u^i_j + \mu \left( u^{i}_{j+1} - 2u^i_j + u^i_{j-1}  \right),
\]
we have

\[
e^{i+1}_j = e^i_j + \mu \left( e^{i}_{j+1} - 2e^i_j + e^i_{j-1}  \right) + \mathcal{O}(h^4), \qquad h \to 0.
\]
This means that for sufficiently small $h > 0$, there exists a constant $c$ such that 

\[
\left\vert e^{i+1}_j -  e^i_j - \mu \left( e^{i}_{j+1} - 2e^i_j + e^i_{j-1}  \right) \right\vert  \leq ch^4.
\]
Hence, for sufficiently small $h$, we have


\begin{eqnarray*}
\vert e^{i+1}_j \vert &\leq& \left\vert e^i_j + \mu ( e^{i}_{j+1} - 2e^i_j + e^i_{j-1} ) \right\vert + ch^4\\
 & \leq & \mu \vert e^i_{j-1} \vert + \vert 1 - 2\mu \vert \vert e^i_{j} \vert  + \mu \vert e^i_{j+1} \vert + ch^4  \\
 & \leq & \left(2\mu  + \vert 1 - 2\mu \vert \right) \eta^i  + ch^4  
\end{eqnarray*}
where 

\[
\eta^{i} = \max_{j = 0, \ldots, n_x+1}\vert e^i_j \vert.
\]
The above inequality holds for $\vert e^{i+1}_j \vert$, with $j = 0, \ldots, n_x+1$ and therefore it also holds for $\vert  \eta^{i+1} \vert$. Since $0 < \mu \leq 1/2$ and $u^0_j = \tilde{u}^0_j$, which implies that $\eta^0$, we have that

\[
\eta^{i+1} \leq \eta^i  + ch^4 \leq \eta^{i-1} + 2ch^4 \leq \cdots \leq \eta^{0} + (i+1)ch^4 = (i+1)ch^4.
\]
Therefore

\[
\eta^{n_t} \leq c n_th^4 = \frac{c}{\mu} n_t\mu h^4 = \frac{c}{\mu} n_t\tau h^2 = \frac{c}{\mu}T h^2 
\]
We conclude that $\eta^i \to 0$ for $i = 0, \ldots, n_t$ as $h \to 0$ and therefore $e^i_j \to 0$ as $h \to 0$ for $j = 0, \ldots, n_x + 1$ and $i = 0, \ldots, n_t$ as $h \to 0$, which implies that the method converges.   $\blacksquare$

\textbf{Remark:} Notice that the maximum error of the Euler method (assuming all relevant partial derivatives exist and are bounded) is bounded by $\frac{c}{\mu}T h^2$ for sufficiently small $h$. This confirms our numerical observation above that the method converges at the rate $\mathcal{O}(h^2)$ as $h \to 0$, or equivalently, at the rate $\mathcal{O}(n_x^{-2})$ as $n_x \to \infty$ (i.e., the method is second order).

\subsubsection{Well-posedness and ill-posedness of PDE problems}
A PDE intial and / or boundary value problem is \textbf{well-posed} if there exists a unique solution that depends continuously on the initial and boundary data.  Otherwise, the PDE problem is said to be \textbf{ill-posed}.

\textbf{Example:}  Suppose $\varphi_0(t) = \varphi_1(t) = 0$ (i.e., zero Dirichlet boundary conditions), then using the method of separation of variables, it can be shown that if

\[
u(x,0) = f(x) = \sum_{m = 1}^{\infty}\alpha_m \sin \pi m x, \qquad 0 \leq x \leq 1,
\]
then the solution to the diffusion equation is

\[
u(x,t) = \sum_{m = 1}^{\infty}\alpha_m {\rm e}^{-\pi^2m^2 t} \sin \pi m x, \qquad 0 \leq x \leq 1, \qquad t \geq 0.
\]
The $2$-norm of a function $g$ on $[0, 1]$ is defined as

\[
\| g \|_2 = \left( \int_0^1 \left[g(x)\right]^2\,{\rm d}x   \right)^{1/2}
\]
Hence,


\begin{eqnarray*}
\|u(x,t)\|_2^2 &=& \int_0^1 \left[u(x,t)\right]^2\,{\rm d}x  = \int_0^1 \left(\sum_{m = 1}^{\infty}\alpha_m {\rm e}^{-\pi^2m^2 t} \sin \pi m x  \right)^2\,{\rm d}x \\
&=& \sum_{m=1}^{\infty}\sum_{j=1}^{\infty} \alpha_m \alpha_j{\rm e}^{-\pi^2(m^2 + j^2) t}\int_0^1 \sin\pi m x\sin \pi j x\, {\rm d}x
\end{eqnarray*}
and since

\[
\int_0^1 \sin \pi m x \sin \pi j x\, {\rm d}x = \begin{cases}
\frac{1}{2}, & m  = j, \\
0, & \text{otherwise}
\end{cases}
\]
we have

\[
\|u(x,t)\|_2^2 =  \frac{1}{2}\sum_{m=1}^{\infty} \alpha_m^2 {\rm e}^{-2\pi^2 m^2 t} \leq \frac{1}{2}\sum_{m=1}^{\infty} \alpha_m^2 = \| u(x,0) \|_2^2.
\]
Suppose that $\tilde{u}(x,t)$ is the solution to the diffusion equation with zero boundary conditions and a slightly perturbed initial condition, i.e.,

\[
\tilde{u}(x,0) = \tilde{f}(x) = \sum_{m = 1}^{\infty}\tilde{\alpha}_m \sin \pi m x, \qquad 0 \leq x \leq 1,
\]
where

\[
\| \tilde{u}(x,0) - u(x,0) \|_2^2 = \| \tilde{f}(x) - f(x) \|_2^2 = \frac{1}{2}\sum_{m=1}^{\infty} \left(  \tilde{\alpha}_m -    \alpha_m  \right)^2   \leq \epsilon^2 \ll 1
\]
then

\[
\tilde{u}(x,t) = \sum_{m = 1}^{\infty}\tilde{\alpha}_m {\rm e}^{-\pi^2m^2 t} \sin \pi m x, \qquad 0 \leq x \leq 1, \qquad t \geq 0
\]
and therefore

\[
\|\tilde{u}(x,t) - u(x,t)\|_2^2   \leq  \| \tilde{u}(x,0) -  u(x,0) \|_2^2 = \epsilon^2.
\]
This shows that if $\| \tilde{u}(x,0) -  u(x,0) \|_2$ is small, then $\|\tilde{u}(x,t) - u(x,t)\|_2$ is also small, which shows that the solution depends continuously on the initial data, i.e., the diffusion equation with zero Dirichlet boundary conditions is well-posed.



\end{document}
